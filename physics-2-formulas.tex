\documentclass[10pt]{extarticle}

\title{Physics 2 Formulas}
\author{Giacomo Ellero}
\date{a.y. 2024/2025}

\usepackage[arrowdel]{physics}
\usepackage{siunitx}
\usepackage{preamble_base}
\usepackage{preamble_math}
\usepackage{multicol}

% \renewcommand{\vec}[1]{\uvec{#1}}
\setlength{\columnsep}{1cm}
\setlength{\columnseprule}{1pt}
\def\columnseprulecolor{\color{blue}}

\numberwithin{equation}{section}

\begin{document}

\begin{multicols}{2}


    \section{Change of coordinates}

    \subsection{Spherical}
    \begin{equation}
        \begin{cases}
            x = r \sin \theta \cos \varphi & r \in [0, \infty)     \\
            y = r \sin \theta \sin \varphi & \theta \in [0, \pi]   \\
            z = r \cos \theta              & \varphi \in [0, 2\pi]
        \end{cases}
    \end{equation}

    Jacobian:
    \begin{equation}
        \dd{\tau} = r^2 \sin \theta \dd{r} \dd{\theta} \dd{\varphi}
    \end{equation}

    \subsection{Cylindrical}
    \begin{equation}
        \begin{cases}
            x = s \cos \varphi & s \in [0, \infty)       \\
            y = s \sin \varphi & \varphi \in [0, \pi]    \\
            z = z              & z \in (-\infty, \infty)
        \end{cases}
    \end{equation}

    Jacobian:
    \begin{equation}
        \dd{\tau} = s \dd{s} \dd{\varphi} \dd{z}
    \end{equation}

    \section{Electric Field}

    \textbf{Coulomb law}
    \begin{equation}
        \vec F = \oneover{4\pi \varepsilon_0} \frac{Q_1 Q_2}{\Delta r^2} \hat{\Delta r}
    \end{equation}

    \textbf{Electric field}
    \begin{equation}
        \vec E =
        \oneover{4 \pi \varepsilon_0}
        \int_{R^3} \frac{\rho (\vec{r'})}{\Delta r} \hat{\Delta r} \dd{\tau'}
    \end{equation}
    where
    \begin{itemize}
        \item $\vec r$ is the position of the point where we want to compute the field
        \item $\vec{r'}$ is the position of the space we are integrating over
        \item $\Delta r = \vec r - \vec{r'}$
        \item $\dd \tau'$ is the portion of space at $\vec{r'}$
    \end{itemize}

    \textbf{Gauss law}
    \begin{equation}
        \oint_S \vec E \cdot \vec{\dd S} = \frac{Q_\text{enc}}{\varepsilon_0}
    \end{equation}

    \textbf{Maxwell equation (Gauss in local form)}
    \begin{equation}
        \div \vec E = \frac{\rho}{\varepsilon_0}
    \end{equation}

    \section{Potential}

    \textbf{Definition}
    \begin{equation}
        \vec E = - \grad V \implies \int_C \vec E \cdot \vec{\dd \ell} = V(A) - V(B)
    \end{equation}
    assume that $V(\infty) = 0$.

    \textbf{Point charge}
    \begin{equation}
        V(\vec r) = \oneover{4 \pi \varepsilon_0} \frac{q}{\Delta r}
    \end{equation}

    \textbf{Continuous charge distribution}
    \begin{equation}
        V(\vec{r}) = \oneover{4 \pi \varepsilon_0} \int_{R^3} \frac{\rho(\vec{r'})}{\Delta r} \dd{\tau'}
    \end{equation}

    \textbf{Potential}
    \begin{equation}
        \oneover{4 \pi \varepsilon_0} \frac{\vec p \cdot \hat r}{r^2}
    \end{equation}

    \section{Energy}

    \textbf{Potential energy}
    \begin{equation}
        U(\vec r) = q V(\vec r)
    \end{equation}

    \textbf{Energy stored}
    \begin{equation}
        W = \half \int_{\mathcal V} \rho V \dd \tau = \half \varepsilon_0 \int_{\R^3} E^2 \dd \tau
    \end{equation}

    \section{Dipoles}

    \textbf{Ideal dipole}
    \begin{equation}
        r \gg d
    \end{equation}

    \textbf{Dipole moment}
    \begin{equation}
        \vec p = q \vec d
    \end{equation}

    \textbf{Torque}
    \begin{equation}
        \vec N = \vec p \cross \vec E
    \end{equation}

    \textbf{Force}
    \begin{equation}
        \vec F = (\vec p \cdot \grad) \vec E
    \end{equation}

    \textbf{Energy}
    \begin{equation}
        U = - \vec p \cdot \vec E
    \end{equation}

    \section{Conductors}

    \textbf{Properties}
    \begin{enumerate}
        \item $\vec E_\text{inside} = 0$
        \item The induced charged are located on the surface of the conductor.
        \item $\vec E^\text{just outside} = \frac{\sigma}{\varepsilon_0} \hat n$.
        \item $V_\text{inside} = \text{const} \implies$ conductors are \emph{equipotential}.
    \end{enumerate}

    \subsection{Capacitors}

    \textbf{Capacitance}
    \begin{equation}
        C = \frac{Q }{\Delta V} = \varepsilon_0 \frac{A}{d}
    \end{equation}

    \textbf{Parallel} (same potential)
    \begin{equation}
        C = C_1 + C_2
    \end{equation}

    \textbf{Series} (same charge)
    \begin{equation}
        C = \frac{C_1 C_2}{C_1 + C_2}
    \end{equation}

    \textbf{Energy}
    \begin{equation}
        W = \half Q \Delta V
    \end{equation}

    \textbf{Force}
    \begin{equation}
        F = \frac{Q}{2 \varepsilon_0 A}
    \end{equation}

    \textbf{Pressure}
    \begin{equation}
        P = \frac{Q}{2 \varepsilon_0 A^2} = u
    \end{equation}

    \section{Insulators}

    \textbf{Bound densities}
    \begin{equation}
        \begin{cases}
            \rho_b = -\div \vec P \\
            \sigma_b = \vec P \cdot \hat n
        \end{cases}
    \end{equation}

    \textbf{Electric displacement}
    \begin{gather}
        \vec D = \varepsilon_0 \varepsilon_r \vec E \\
        \div D = \rho_f \\
        \oint_S \vec D \cdot \vec{\dd S} = Q_\text{enc}^f
    \end{gather}

    \textbf{Polarization}
    \begin{equation}
        \vec P = \varepsilon_0 (\varepsilon_r-1) \vec E
    \end{equation}

\end{multicols}
\end{document}

