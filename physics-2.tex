\documentclass[12pt]{extarticle}

\setlength{\headheight}{16pt} % ??? we do what fancyhdr tells us to do  

\title{Physics 2}
\author{Giacomo Ellero}
\date{a.y. 2024/2025}

\usepackage[arrowdel]{physics}
\usepackage{siunitx}
\usepackage{preamble}
\usepackage{preamble_svg}

% \renewcommand{\vec}[1]{\uvec{#1}}

\begin{document}

\firstpage

\section{Mathematical tools}

\subsection{Scalar and vector fields}

\begin{definition}{Scalar field}{scalar-field}
    A scalar field is a function of type $f:\R^d \to \R$.
\end{definition}

Some examples of scalar fields are temperature or pressure.

\begin{definition}{Level curve of a scalar field}{level-curve-scalar}
    Let $f$ be a scalar field and $k \in \R$ constant.
    Then a level curve of $f$ at $k$ is the set
    \begin{equation}
        \left\{ \vec x \in \R^d : f(\vec x) = k  \right\}
    \end{equation}
\end{definition}


\begin{definition}{Vector field}{vector-field}
    A vector field is a function of type $\vec v: \R^d \to \R^d$.
\end{definition}

Some examples of vector fields are gravitational fields or the velocity field of fluids.

We will always work with \say{well behaved} fields: this means the functions are \say{almost always infinite} (of class $C^\infty$ with only some points as exceptions).

\begin{example}{Vector field}{vec-field}
    Let $\vec v(x, y, z) = x \hat x + y \hat y + z \hat z$.

    \begin{figure}[H]
        \centering
        \includegraphics[width=0.6\textwidth]{assets/physics-2/vector-field-example.png}
        \caption{The arrow representation of $\vec v$ when $z = 0$.}
        \label{fig:field_source_0}
    \end{figure}
\end{example}

Some vector fields have \say{special points} where there are an infinite number of field lines going through it:
\begin{itemize}
    \item If all the lines go into that point we call such point \textbf{source} (the point $(0,0,0)$ is a source for \Cref{ex:vec-field});
    \item If all the lines go out of that point we call such point \textbf{sink} (use the field $\vec v(x, y, z) = -x \hat x - y \hat y - z \hat z$ as an example).
\end{itemize}
It is possible for some vector fields to have no sources or sinks. Some patterns that can arise are \textit{loops} or \textit{straight lines}.

\subsection{Operations over fields}

\begin{definition}{Gradient of a scalar field}{gradient-scalar}
    We define the gradient as
    \begin{equation}
        \grad f = \pdv{f}{x} \hat{x} + \pdv{f}{y} \hat{y} + \pdv{f}{z} \hat{z}
    \end{equation}

    We can also define the differential of a field as follows (where $\dd{l} = \dd{x} \hat{x} + \dd{y} \hat{y} + \dd{z} \hat{z} $):
    \begin{equation}
        \dd{f} = \pdv{f}{x} \dd{x} + \pdv{f}{y} \dd{y} + \pdv{f}{z} \dd{z} = \grad f \cdot \dd{\vec l}
    \end{equation}
\end{definition}

The gradient of a field is somewhat the equivalent of the derivative for 1D functions.

\begin{proposition}{Properties of the gradient}{props-gradient}
    \begin{enumerate}[label=\roman*.]
        \item $\grad f$ is orthogonal to the level curves;
        \item $\grad f$ points in the direction of the steepest ascent;
    \end{enumerate}
\end{proposition}

\begin{definition}{Directional derivative}{directional-derivative}
    The directional derivative is the slope of the field in the direction $\vec l$.
    \begin{equation}
        \dv{f}{l} = \grad \cdot \dd{\vec l}
    \end{equation}
\end{definition}

\begin{definition}{Gradient (nabla) operator}{nabla-operator}
    We define the nabla operator as
    \begin{equation}
        \grad = \pdv{}{x} \hat{x} + \pdv{}{y} \hat{y} + \pdv{}{z} \hat{z}
    \end{equation}
\end{definition}

This operator is particularly useful because we can treat it basically like a vector: when we multiply the $\grad$ with a vector field $f$ we get the gradient field of $f$.

\begin{definition}{Divergence}{divergence}
    Let $\vec v$. Then
    \begin{align}
        \div \vec v & = \left( \pdv{}{x} \hat{x} + \pdv{}{y} \hat{y} + \pdv{}{z} \hat{z} \right) \cdot \left( v_x \hat x + v_y \hat y + v_z \hat z\right) \\
                    & = \pdv{v_x}{x} + \pdv{v_y}{y} + \pdv{v_z}{z}
    \end{align}
\end{definition}

The intuition behind divergence is to measure how much a field \say{spreads out} from the point where it is measured.

\begin{example}{}{}
    Take $\vec v$ as in \Cref{fig:field_source_0}. Then
    \begin{equation}
        \div \vec v = \pdv{x}{x} + \pdv{y}{y} + \pdv{z}{z} = 3
    \end{equation}

    We have that the divergence for this field is always positive therefore the arrows always go further away from each other.
\end{example}

\begin{example}{Solenoidal field}{solenoidal-field}
    Let $\vec v = \hat z$. We have $\div \vec v = 0$.

    Fields with zero divergence are called solenoidals.
\end{example}

\begin{definition}{Curl}{curl}
    Let $\vec v$ be a vector field. Then
    \begin{equation}
        \curl \vec f = \det\begin{vmatrix}
            \hat x    & \hat y    & \hat z    \\
            \pdv{}{x} & \pdv{}{y} & \pdv{}{z} \\
            v_x       & v_y       & v_z
        \end{vmatrix}
    \end{equation}
\end{definition}

The intuition behind curl is to measure how much a vector field rotates or spins around the point where we compute it.

\begin{proposition}{Linearity}{linearity-of-grad}
    $\grad(\mathord{\cdot})$, $\div(\mathord{\cdot})$, and $\curl(\mathord{\cdot})$ are linear operators.
\end{proposition}

\begin{proposition}{Leibniz rule}{leibniz-rule}
    \begin{equation}
        \grad{fg} = (\grad f)g + f (\grad g)
    \end{equation}
\end{proposition}

\subsubsection{Higher order derivatives}
\label{sec:higher-order-derivatives}

\begin{enumerate}
    \item \textit{Divergence of gradient}
          (also called \textit{laplacian}):

          \begin{align}
              \div (\grad \vec f) & = \left( \pdv{}{x} \hat{x} + \pdv{}{y} \hat{y} + \pdv{}{z} \hat{z} \right) \cdot \left( \pdv{f}{x} \hat{x} + \pdv{f}{y} \hat{y} + \pdv{f}{z} \hat{z} \right) \\
                                  & = \pdv[2]{f}{x} + \pdv[2]{f}{y} + \pdv[2]{f}{z} = \grad^2(f)
          \end{align}

    \item \textit{Curl of gradient}: $\curl(\grad f) = 0$ (Schwartz theorem holds).
    \item \textit{Gradient of divergence}: $\grad (\div \vec V)$ is usually not useful.
    \item \textit{Divergence of curl}: $\div (\curl \vec v) = 0$
    \item \textit{Curl of curl}: (not so frequent)
          \begin{equation}
              \curl (\curl \vec f) = \grad (\div \vec v) - \grad^2(\vec v)
          \end{equation}
          where $\grad^2(\vec v) = (\grad^2 v_x \hat x + \grad^2 v_y \hat y + \grad^2 v_z \hat z)$ is called the laplacian vector operator.
\end{enumerate}

\subsection{Integrals}

\subsubsection{Line integrals}

\begin{definition}{Line integral}{line-integral}
    Let $f$ be a scalar field and $\vec v$ a vector field.
    \begin{enumerate}
        \item $\int_C f \dd{s}$
        \item $\int_C \vec v \dd{\vec l}$
    \end{enumerate}
    where $\dd{\vec l}$ is tangent to $C$ at every point.
\end{definition}

Note that if $C$ is closed we will write $\oint$ instead of $\int$.

Moreover, some fields $\vec v$ are such that $\int_C \vec v \dd{\vec l}$ does not depend on $C$ but only on its endpoints.
These are called \textbf{conservative fields}.

\subsubsection{Surface integral}

Let a vector field $\vec v$ and an open surface $S$.
We define $\dd{\vec S}$ the infinitesimal area oriented normal to $S$: $\dd{\vec S} = \dd{S} \vec n$.

Then define the \textit{infinitesimal flux} over $S$ as
\begin{equation}
    \dd{\Phi_S(\vec v)} = \vec v \cdot \dd{\vec S}
\end{equation}

\begin{definition}{Flux}{flux}
    The flux of $\vec v$ over $S$ is defined as
    \begin{equation}
        \Phi_S(\vec v) = \int_S \dd{\Phi_S(\vec v)} = \int \vec v \cdot \dd{\vec S}
    \end{equation}
\end{definition}

Again, if $S$ is a closed surface (like a sphere) we use $\oint$ instead of $\int$.
By convention, if $S$ is closed $\vec n$ goes outwards.

Moreover, we will commonly refer to $C$ as the boundary of $S$.

\subsubsection{Volume integral}

\begin{definition}{Volume integral}{volume-integral}
    Let $f$ be a scalar field and $V \in \R^3$.
    We want to define $I = \int_V f \dd{\tau}$ where $\dd{\tau} = \dd{x} \dd{y} \dd{z}$.

    If $\vec g$ is a vector field we define the integral as
    \begin{equation}
        \vec I = \int_V \vec g(x, y, z) \dd{\tau} = \int_V g_x (x, y, z) \dd{x} \hat x + \int_V g_y (x, y, z) \dd{y} \hat y + \int_V g_z (x, y, z) \dd{z} \hat z
    \end{equation}
\end{definition}

\subsection{Theorems for \texorpdfstring{$\grad(\mathord{\cdot})$, $\div(\mathord{\cdot})$, and $\curl(\mathord{\cdot}) $}{gradient, divergence and curl}}

\subsubsection{Gradient}

\begin{theorem}{Fundamental theorem of gradient}{fundamental-gradient}
    Let $f$ be a gradient field and $C$ be a path. We have
    \begin{equation}
        \int_C \dd{f} = \int_A^B \grad f \cdot \dd{\vec l} = f(B)- f(A)
    \end{equation}
\end{theorem}

\begin{corollary}{}{}
    If $\vec v = \grad f$ then $\vec v$ is conservative.
\end{corollary}

\begin{corollary}{}{}
    If $C$ is closed and $\vec v = \grad f$ we have
    \begin{equation}
        \oint \vec v \cdot \dd{\vec l} = \oint \grad f \cdot \dd{l} = 0
    \end{equation}
\end{corollary}

\subsubsection{Divergence}

\begin{theorem}{Divergence theorem}{divergence-theorem}
    Let $V$ be a volume, $S$ be a surface corresponding to that volume and $\vec v$ a vector field.
    Then
    \begin{equation}
        \int_V \left(\div \vec v\right) = \oint_S \vec v \cdot \dd{\vec S}
    \end{equation}
\end{theorem}

\begin{proof}
    We have done a partial proof in the case of a cube in Physics 1.
\end{proof}

\begin{corollary}{}{}
    If $\vec v$ is solenoidal
    \begin{enumerate}
        \item $\oint_S \vec v \cdot \dd{\vec S} = 0 \enspace \forall S$
        \item $\int_S \vec v \cdot \dd{S}$ is independent of $S$ but only on the boundary line of $S$
        \item Let $\vec v = \curl \vec A$ where $\vec A$ is a vector field. Then $\div \vec v = \div (\curl \vec A) = 0$
    \end{enumerate}
\end{corollary}

\begin{proof}
    \skiplineafterproof
    \begin{enumerate}
        \item Trivial from theorem and definition of solenoidal.
        \item
              For any $S_1$ open we can find another $S_2$ open with the same boundary line.
              Let $S = S_1 \cup S_2$, then $S$ is closed.
              By the divergence theorem we can write
              \begin{equation}
                  \oint_S \vec v \cdot \dd{\vec S} = \int_V \left(\div \vec v\right) = 0 = \int_{S_1} \vec v \cdot \dd{\vec S} + \int_{S_2} \vec v \cdot \dd{\vec S}
              \end{equation}
              This means that $\abs{\Phi_{S_1}(\vec v)} = \abs{\Phi_{S_2} (\vec v)}$.
        \item This is a consequence of the theorem and the results of \cref{sec:higher-order-derivatives}.
    \end{enumerate}
\end{proof}

\subsubsection{Curl}

\begin{theorem}{Stokes' theorem}{stokes}
    Let $\vec v$ be a vector field, $S$ an open surface, $C$ the boundary line of $S$.
    Then
    \begin{equation}
        \int_S (\curl \vec v) \cdot \dd{\vec S} = \oint_C \vec v \cdot \dd{\vec l}
    \end{equation}
    and the direction of $C$ depends on the direction of the normals of $S$ by the right-hand rule.
\end{theorem}

The intuition behind the theorem is that if we consider the total \say{rotation} in the surface it will be equal to the total \say{diraction} of the vectors at the boundary.

\begin{proof}
    We will consider a surface that can be decomposed in rectangles.
    Consider a rectangle with vertices $P_1,P_2,P_3,P_4$ on the $yz$ plane. Let $\dd{y}$ be the \say{base} of the rectangle, $\dd{z}$ its height and $A = (x, y, z)$ the center of the rectangle.

    Now we want to compute the line integral over the boundary of the rectangle going counterclockwise:
    \begin{align}
        P_1 P_2: \quad & \dd{\vec l} = \dd{y} \hat y \implies \vec v \cdot \dd{\vec l} = v_y\left(x, y, z - \frac{\dd{z}}{2}\right)\dd{y} \approx \left[v_y(x, y, z) \dd{y} - \pdv{v_y(x, y, z)}{z} \frac{\dd{z}}{2}\right] \dd{y}    \\
        P_3 P_4: \quad & \dd{\vec l} = -\dd{y} \hat y \implies \vec v \cdot \dd{\vec l} = -v_y\left(x, y, z - \frac{\dd{z}}{2}\right)\dd{y} \approx -\left[v_y(x, y, z) \dd{y} - \pdv{v_y(x, y, z)}{z} \frac{\dd{z}}{2}\right] \dd{y} \\
        P_2 P_3: \quad & \dd{\vec l} = \dd{z} \hat z \implies \vec v \cdot \dd{\vec l} = v_z\left(x, y - \frac{\dd{y}}{2}, z\right)\dd{y} \approx \left[v_z(x, y, z) \dd{z} - \pdv{v_z(x, y, z)}{y} \frac{\dd{y}}{2}\right] \dd{z}    \\
        P_4 P_1: \quad & \dd{\vec l} = -\dd{z} \hat z \implies \vec v \cdot \dd{\vec l} = -v_z\left(x, y - \frac{\dd{y}}{2}, z\right)\dd{y} \approx -\left[v_z(x, y, z) \dd{z} - \pdv{v_z(x, y, z)}{y} \frac{\dd{y}}{2}\right] \dd{z}
    \end{align}
    where the $\approx$ are due to a taylor expansion.

    By summing up all the terms we get that many terms cancel out:
    \begin{equation}
        P_1P_2 + P_2P_3 + P_3P_4 + P_4P_1 = \left( \pdv{v_z}{y} - \pdv{v_y}{z} \right) \dd{y} \dd{z} = \left( \curl \vec v \right)_x \dd{y} \dd{z}
    \end{equation}

    \textit{A complete proof would require the same procedure on the $xy$ and $xz$ planes.}
\end{proof}

\begin{corollary}{}{}
    The theorem doesn't say anything about $S$ itself, just the boundary $C$.
    Therefore we can say
    \begin{equation}
        \int_S (\curl \vec v) \cdot \dd{\vec S} \text{ only depends on the boundary line of } S
    \end{equation}

    Moreover
    \begin{equation}
        \oint_S (\curl \vec v) \cdot \dd{\vec S} = 0
    \end{equation}
\end{corollary}

\begin{theorem}{Fundamental theorem for irrotational fields}{irrotational-fields}
    Let $\vec v$ such that $\curl \vec v = 0$ everywhere.
    Then:
    \begin{enumerate}
        \item $\oint_C \vec v \cdot \dd{\vec l} \enspace \forall C \text{ closed}$.
        \item $\int_A^B \vec v \cdot \dd{\vec l}$ is independent of the slope of the path.
        \item There exists $f$ such that $\vec v = \grad f$.
    \end{enumerate}
\end{theorem}

\subsection{Dirac delta function}

\subsubsection{Motivation}

Let $\vec v$ be a vector field defined as
\begin{equation}
    \vec v = \frac{\vec r}{r^3} = \frac{x \hat x + y \hat y + z \hat z}{\left( x^2 + y^2 + z^2\right)^{\frac{3}{2}}} \in \R^3 \setminus (0,0,0)
\end{equation}

We want to compute its divergence.
\begin{equation}
    \pdv{v_x}{x} = \frac{1}{\left( x^2 + y^2 + z^2\right)^{\frac{3}{2}}} - \frac{3}{2}\frac{2 x^2}{\left( x^2 + y^2 + z^2\right)^{\frac{5}{2}}} = \frac{-2x^2 + y^2 + z^2}{\left( x^2 + y^2 + z^2\right)^{\frac{5}{2}}}
\end{equation}

Similarly we get
\begin{align}
    \pdv{v_x}{x} & = \frac{x^2 - 2y^2 + z^2}{\left( x^2 + y^2 + z^2\right)^{\frac{5}{2}}} \\
    \pdv{v_x}{x} & = \frac{x^2 + y^2 - 2z^2}{\left( x^2 + y^2 + z^2\right)^{\frac{5}{2}}}
\end{align}
and by adding up all the partial derivatives we get that $\div \vec v = 0$.

Now let's use the divergence theorem to compute the flux of $\vec v$ over a sphere:
\begin{equation}
    \oint_S \vec v \cdot \dd{\vec S} = \int_V \div \vec v \dd{\tau} {\color{red}= 0}
    \label{eq:dirac-wrong}
\end{equation}

This would look correct but let's also compute this without the divergence theorem.
Using spherical coordinates we write
\begin{equation}
    \dd{\vec S} = \dd{S}\hat r = R^2 \sin\theta \dd{\theta} \dd{\phi} \hat r
\end{equation}
Now we notice that $\dd{S}$ is always at the same distance from the center of the sphere and
\begin{equation}
    \oint_S \vec v \cdot \dd{\vec S} = \int \frac{R}{R^3} \dd{S} = \frac{1}{R^2} \int \dd{S} = \frac{1}{R^2} \cdot 4 \pi R^2
\end{equation}
where $\int \dd{S}$ is the area of the sphere, therefore we get a result of $4\pi$ which conflicts with the result we got from the divergence theorem.

The issue is that we have a singularity at $(0,0,0)$ where the divergence is not $0$.

\subsubsection{1D Dirac delta function}

To simplify we start by the 1D case.
\begin{definition}{1D Dirac delta function}{1d-dirac-delta}
    \begin{equation}
        \delta(x) = \begin{cases}
            0      & x \ne 0          \\
            \infty & \text{if } x = 0
        \end{cases}  \quad, \quad \int_{-\infty}^{\infty} \delta(x) \dd{x} = 1
    \end{equation}
\end{definition}

A way to obtain this function is as the limit of a sequence of functions. Define
\begin{equation}
    R_n(x) = n \cdot 1_{[-\frac{1}{2n}, \frac{1}{2n}]}(x)
\end{equation}
the rectangle of width $\frac{1}{n}$ and height $n$.
Indeed the integral of this function is $1$ and as we take $\lim_{n \to \infty} R_n(x) = \delta(x)$.

Another way to obtain $\delta(x)$ is taking a gaussian distribution with $\mu = 0$ and take $\lim_{\sigma^2 \to 0}$.

\begin{proposition}{Properties of the Dirac delta functions}{prop-1D-dirac-delta}
    We have that $f(x)\delta(x) = f(0) \delta(x)$.
    Moreover, under the integral we get
    \begin{equation}
        \int_{-\infty}^{\infty} f(x) \delta(x - x_0) \dd{x} = \int_{-\infty}^{\infty} f(x_0) \delta(x) \dd{x} = f(x_0)
    \end{equation}
    for a fixed $x_0$.
\end{proposition}

To be able to apply this property we have to make sure that the \say{spike} is within the boundary of integration.

\subsubsection{3D Dirac delta function}

Now we are ready to go back to the 3D version:
\begin{definition}{3D Dirac delta function}{dirac-delta}
    \begin{equation}
        \delta^{(3)}(\vec r) = \delta(\vec r) = \delta(x) \delta(y) \delta(z)
    \end{equation}
    that is, the product of the 1D version of the dirac delta as in \cref{def:1d-dirac-delta}.
\end{definition}

Here we have as well that the integral is $1$:
\begin{equation}
    \int_{\R^3} \delta(\vec r) \dd{x} \dd{y} \dd{z} = \int_{\R^3} \delta(x) \delta(y) \delta(z) \dd{x} \dd{y} \dd{z} = 1 \cdot 1 \cdot 1 = 1
\end{equation}

Now we can solve the mystery of \cref{eq:dirac-wrong}
\begin{equation}
    \label{eq:dirac-delta-ok}
    \div \left( \frac{\hat r}{r^3} \right) = 4 \pi \delta(\vec r)
\end{equation}

\subsection{Change of coordinates}

\subsubsection{Spherical coordiante}

\begin{definition}{Spherical coordinates}{spherical-coordiantes}
    Let $\vec r = x \hat x + y \hat y + z \hat z$.
    We use the following transformation:
    \begin{equation}
        \begin{cases}
            x = r \sin \theta \cos \varphi & r \in [0, \infty)     \\
            y = r \sin \theta \sin \varphi & \theta \in [0, \pi]   \\
            z = r \cos \theta              & \varphi \in [0, 2\pi]
        \end{cases}
    \end{equation}
    where $r$ is the distance of the point with the origin, $\theta$ is the angle that $\vec r$ makes with the $z$ axis, and $\varphi$ is the angle on the $xy$-plane that the projection makes with the $y$ axis.
\end{definition}

To define directions in this coordinate system we fix two coordinates and we slightly increase the other: the \say{movement} we obtain is the direction we want:
\begin{itemize}
    \item $\hat r$ is the radial direction, obtained by slightly increasing $r$
    \item $\hat \theta$ is obtained by increasing $\theta$
    \item $\hat \varphi$ is obtained by increasing $\varphi$
\end{itemize}

\begin{proposition}{Jacobian of spherical change of coordinates}{jacobian-spherical}
    We want to compute $\dd{\vec l}$ given $\dd{x} + \dd{y} + \dd{z}$:
    \begin{itemize}
        \item $\dd{l}_r = \dd{r}$
        \item $\dd{l}_\theta = r \dd{\theta}$
        \item $\dd{l}_\varphi = r \sin \theta \dd{\varphi}$
    \end{itemize}

    Therefore
    \begin{equation}
        \dd{\tau} = \dd{x} \dd{y} \dd{z} = r^2 \sin \theta \dd{r} \dd{\theta} \dd{\varphi}
    \end{equation}
\end{proposition}

\subsubsection{Cylindircal coordinates}

\begin{definition}{Cylindircal coordinates}
    Let $\vec r = x \hat x + y \hat y + z \hat z$.
    We use the following transformation:
    \begin{equation}
        \begin{cases}
            x = s \cos \varphi & s \in [0, \infty)       \\
            y = s \sin \varphi & \varphi \in [0, \pi]    \\
            z = z              & z \in (-\infty, \infty)
        \end{cases}
    \end{equation}
\end{definition}

\begin{proposition}{Jacobian of cylindircal change of coordinates}{jacobian-cylindircal}
    \begin{equation}
        \dd{\tau} = \dd{x} \dd{y} \dd{z} = s \dd{s} \dd{\varphi} \dd{z}
    \end{equation}
\end{proposition}

\section{Electrostatics}

\subsection{Introduction}

Franklin was the first one to show that rubbing a piece of glass with fur you get a positive charge, while if you do it with amber you get a negative charge.
Fur would always get the opposite charge (\emph{conservation of charge}).
He noticed that same charge repel while different charges attract.

Charge is measured in Coulomb ($\si{\coulomb}$).
To measure charge we use the \emph{electroscope}:
this instrument will get charged by contact and since alike charges repel we can measure the angle the gold foil create to measure the charge.

\begin{figure}[H]
    \centering
    \includesvg[width=0.4\textwidth]{assets/physics-2/electroscope.svg}
    \caption{Drawing of an electroscope}
\end{figure}

The \textbf{general problem of electrostatics} is to compute the force due to some \emph{source charges} $q_1, \dots, q_n$ on a \emph{test charge} $Q$.

\begin{proposition}{Charge is quantized}{quantization-charge}
    The smallest charge possible is the charge of an electron:
    \begin{equation}
        Q_e = 1.6 \cdot 10^{-19} \si{\coulomb}
    \end{equation}
\end{proposition}

\begin{proposition}{Superposition principle}{superposition-principle}
    If we have $n$ charges, the total force applied by those particles is
    \begin{equation}
        \vec F = \vec F_1 + \dots + \vec F_n
    \end{equation}
\end{proposition}

But even in the case of $n = 1$ it is not easy to compute $\vec F$ because it not only depends on $\vec r$ but also $\vec v_q$ and $\vec a_q$.
This is why we invented electrostatics, where charges do not move.

\subsection{Coulomb's law}

\begin{theorem}{Coulomb's law}{coulomb-law}
    Let the distance between the two charges be $ \vec{\Delta r} = \vec {r'} - \vec r$.
    Then the force applied on the two charges is
    \begin{equation}
        \vec F = k \frac{qQ}{\Delta r^2} \hat{\Delta r}
    \end{equation}
    where $k$ is a constant:
    \begin{equation}
        k = \frac{1}{4\pi \varepsilon_0} \simeq 9 \cdot 10^9 \si{\newton \meter \squared \per \coulomb \squared}
    \end{equation}
    and $\varepsilon_0$ is another constant called \textbf{permittivity in vacuum}.
\end{theorem}

Note that, compared to gravity, electric force is in the order of $10^{39}$ times stronger.

\begin{definition}{Electric field (of a point charge)}{electric-field-point}
    \begin{equation}
        \vec E(\vec r) = \frac{\vec F}{Q} = k \frac{q}{\Delta r^2} \hat{\Delta r}
    \end{equation}

    This is particularly useful because it is independent of the test charge.
\end{definition}

If the charge is positive the field will have positive divergence, if negative it will have negative divergence.
Moreover, note that the electric field gets its field lines from $\frac{r^2}{\hat r}$ as in \Cref{eq:dirac-delta-ok}.

\begin{definition}{Electric field (continuous charge distribution)}{electric-field-cont}
    Consider a small $\dd{\tau}$ where we can consider to have a constant point charge of $\dd{q}$.
    Then
    \begin{align}
        \dd{\vec E} & = \frac{1}{4\pi \varepsilon_0} \frac{\dd{q}}{\Delta r^2} \hat{\Delta r}                             \\
        \vec E      & = \int_R \dd{\vec E} = \frac{1}{4\pi \varepsilon_0} \int_R \frac{\dd{q}}{\Delta r^2} \hat{\Delta r}
    \end{align}
    where $R$ is the region where the charge is.
\end{definition}

We now have three cases.
\begin{enumerate}
    \item \emph{$R$ is one-dimensional}.
          Let $\lambda$ be the 1D density of charge:
          \begin{equation}
              \lambda(\vec {r'}) = \dv{q}{l} \implies \dd{q} = \lambda (\vec {r'}) \dd{l}
          \end{equation}
    \item \emph{$R$ is a surface of charge}.
          \begin{equation}
              \sigma(\vec {r'}) = \dv{q}{S}
          \end{equation}
    \item \emph{$R$ is a volume density of charge}.
          \begin{equation}
              \rho(\vec {r'}) = \dv{q}{\tau}
          \end{equation}
\end{enumerate}
Similarly to the first case, in the other two we define the density of charge in the respective dimension and substitute $\dd{q}$ in the field equation.

\begin{example}{Staight line of constant charge}{straight-line-const-charge}
    Consider a straight line of length $2L$ and of constant charge $\lambda(\vec r) = \lambda$.
    Compute the electric field at $P(0,z)$.

    \begin{figure}[H]
        \centering
        \includesvg[width=0.5\textwidth]{assets/physics-2/ex-electric-field-straight-line.svg}
        \caption{Illustration of the problem}
    \end{figure}
\end{example}

\begin{proof}[Solution]
    First, consider the infinitesimal change in the electric field $\dd{\vec E}$ due to a small $\dd{l}$ difference length of the line.
    As we know, a length of $\dd{l}$ carries a charge of $\dd{q} = \lambda \dd{l}$, hence adding $\dd{l}$ on each side of the line will result in $\dd{\vec E_1}$ and $\dd{\vec E_2}$.
    We can compute the total $\dd{\vec E}$ as
    \begin{align}
        \dd{\vec E} & = \dd{\vec E_1} + \dd{\vec E_2}                                                \\
                    & = 2 \dd{E_1} \cos \theta \hat z                                                \\
                    & = 2\oneover{4 \pi \varepsilon_0} \frac{\dd{q}}{r^2} \cos \theta \hat z         \\
                    & = 2\oneover{4 \pi \varepsilon_0} \frac{\lambda \dd{l}}{r^2} \cos \theta \hat z
    \end{align}

    Next we want to write everything in terms of $\theta$ such that we can integrate over it.
    We have
    \begin{gather}
        r         = \frac{z}{\cos \theta}                                      \\
        l(\theta) = z \tan \theta                                              \\
        \dd{l}    = \dd{z \tan \theta} = \frac{z}{\cos^2 \theta} \dd{\theta}
    \end{gather}
    and rewrite $\dd{E}$ accordingly
    \begin{align}
        \dd{E} & = 2\oneover{4 \pi \varepsilon_0} \frac{\lambda \frac{z}{\cos^2 \theta} \dd{\theta}}{\frac{z^2}{\cos^2 \theta}} \cos \theta \hat z \\
               & = \oneover{2 \pi \varepsilon_0} \frac{\lambda}{z} \dd{\theta} \cos \theta \hat z
    \end{align}

    Now we can integrate in order to find the field
    \begin{align}
        \vec E & = \int \dd{\vec E}                                                                                                       \\
               & = \oneover{2 \pi \varepsilon_0} \frac{\lambda}{z} \hat z \underbrace{\int_0^{\frac{\pi}{2}} \dd{\theta} \cos \theta}_{1} \\
               & = \oneover{2 \pi \varepsilon_0} \frac{\lambda}{z} \hat z
    \end{align}
\end{proof}

\begin{example}{Uniformely charged ring}{uniformely-charged-ring}
    Consider a ring of radius $R$ and of constant charge $\lambda(\vec r) = \lambda$.
    Compute the electric field at $P(0,z)$.

    \begin{figure}[H]
        \centering
        \includesvg[width=0.5\textwidth]{assets/physics-2/ex-electric-field-ring.svg}
        \caption{Illustration of the problem}
    \end{figure}
\end{example}

\begin{proof}[Solution]
    We proceed similarly to \Cref{ex:straight-line-const-charge}: first we need to compute $\dd{\vec E}$.
    \begin{align}
        \dd{\vec E} & = \dd{\vec E_1} + \dd{\vec E_2} = 2 \dd{E_1} \cos \theta \hat z               \\
                    & = \frac{2}{4 \pi \varepsilon_0} \frac{\lambda \dd{l}}{r^2} \cos \theta \hat z
    \end{align}

    This time we notice that $\theta$ does not depend on $\dd{l}$, therefore we can compute the integral quite easily as follows:
    \begin{align}
        \vec E & = \int \dd{\vec E} = \frac{2}{4 \pi \varepsilon_0} \frac{\lambda}{r^2} \underbrace{\cos \theta}_{z/r} \hat z \underbrace{\int \dd{l}}_{\pi R} \\
               & = \frac{1}{4 \pi \varepsilon_0} \frac{\lambda (2\pi R) z}{\left(\sqrt{R^2 + z^2}\right)^3} \hat z
    \end{align}
\end{proof}

\begin{example}{Uniformely charged disk}{uniformely-charged-disk}
    Consider a disk of radius $R$ and of constant charge $\sigma(r') = \sigma$.
    Compute the electric field at $P(0,z)$.

    \begin{figure}[H]
        \centering
        \includesvg[width=0.5\textwidth]{assets/physics-2/ex-electric-field-disk.svg}
        \caption{Illustration of the problem}
    \end{figure}
\end{example}

\begin{proof}{Solution}
    The trick is to divide the disk into rings of radius $r \in [0, R]$ with thickness $\dd{r}$, therefore the charge of each ring is
    \begin{equation}
        \dd{q}_\text{ring} = \sigma (2 \pi r) \dd{r}
    \end{equation}
    and we can take the result of \Cref{ex:uniformely-charged-ring} and substitute
    \begin{equation}
        \dd{\vec E}_\text{ring} = \frac{1}{4 \pi \varepsilon_0} \frac{\sigma (2 \pi r) z}{\left(\sqrt{r^2 + z^2}\right)^3} \dd{r} \hat z
    \end{equation}

    Finally, we can integrate in order to get the final result
    \begin{align}
        \vec E & = \int \dd{\vec E}_\text{ring}                                                                                    \\
               & = \frac{1}{4 \pi \varepsilon_0} \sigma(2 \pi) z  \hat z \int_0^R \frac{r}{\left(\sqrt{r^2 + z^2}\right)^3} \dd{r} \\
               & = \frac{\sigma z}{2 \varepsilon_0} \hat z \int_0^R \frac{r}{\left(r^2 + z^2\right)^{\frac{3}{2}}} \dd{r}          \\
               & = \frac{\sigma z}{2 \varepsilon_0} \hat z \left[ -\oneover{\sqrt{z^2 + r^2}} \right]_0^R                          \\
               & = \frac{\sigma z}{2 \varepsilon_0} \left[ \oneover{z} - \oneover{\sqrt{z^2 + R^2}} \right] \hat z
    \end{align}

    Notably, it is interesting the case where $R \to \infty$, meaning the disk becomes a plane.
    In this situation we have that the above equation becomes
    \begin{equation}
        \vec E = \frac{\sigma}{2 \varepsilon_0} \hat z
    \end{equation}
\end{proof}


\end{document}
