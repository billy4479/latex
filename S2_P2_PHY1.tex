\documentclass[12pt]{extarticle}

\usepackage{preamble}

\title{Mathematical Analysis 2 Notes, Partial 2}
\date{Semester 2, 2023/2024}

\setlength{\headheight}{15pt} % ??? we do what fancyhdr tells us to do

\begin{document}

\maketitle
\tableofcontents
\clearpage

% Class of 21/03/2024
\section{Rotational Dynamics}

\subsection{Introduction}

We define a \emph{rigid body} as an object that never changes its shape.
Mathematically we can say that given any $i, j$ inside the body, the distance $d_{ij}$ will stay constant no matter what.

\subsubsection{Types of rigid bodies}

We can differentiate between continuos rigid body and discrete rigid bodies:
\begin{itemize}
    \item Continuous rigid bodies have a density mass function of the form $\rho(\vec{r})$.
    \item Discrete rigid bodies have a finite number of particles, each with a mass $m_i$ and a position $\vec{r}_i$, which might all be connected by some massless structure.
\end{itemize}

\subsubsection{Decomposing displacements}

We want to describe the motion of a rigid body in space. This is usually a hard task.

It turns out that a general displacement of a rigid body can be decomposed into the sum of two motions: a \textbf{translation} of the center of mass and a \textbf{rotation} around the center of mass.
This means that we need 6 parameters to describe any displacement: 3 for the translation and 3 for the rotation.

\subsubsection{The simplest case: 2D}

The simplest case we will consider is a 2D rigid body that lies on the $xy$ plane and can rotate around the $z$ axis.

\subsection{Angular variables}

Consider a simple case of a 2D rigid body that can rotate around the $z$ axis and has its center of mass fixed at the origin.

In this case we can describe the position of any point $p$ in the RB in polar coordinates as $(r, \theta)$. Now, knowing $r$ and $\theta$ we can compute the \textbf{arc length} $s = r\theta$.

Note that specifying $\theta$ is enough to get the position of all the points in the RB because $r$ will stay constant.

\subsubsection{Angular velocity}

Let $\omega_{\text{avg}} \coloneq \frac{\theta(t_2) - \theta(t_1)}{t_2 - t_1}$ be the \textbf{average angular velocity} between times $t_1$ and $t_2$. As $\Delta t = t_2 - t_1 \to 0$, we define the \textbf{instantaneous angular velocity} as

$$
    \omega \coloneq \lim_{\Delta t \to 0} \omega_{\text{avg}} = \dv{\theta}{t}
$$

The angular velocity is a vector, therefore we can decompose it in direction and magnitude. We have $\vec{\omega} = \omega \hat{\omega}$, where $\hat{\omega}$ is either $\hat{z}$ if the motion is counterclockwise or $-\hat{z}$ if the motion is clockwise.

Moreover note that the angular velocity stays constant for every point in the RB, but the linear velocity does not, in fact $v = r\omega$.

\subsubsection{Angular acceleration}

We define the \textbf{average angular acceleration} as $\alpha_{\text{avg}} \coloneq \frac{\omega(t_2) - \omega(t_1)}{t_2 - t_1}$ and the \textbf{instantaneous angular acceleration} as

$$
    \alpha \coloneq \lim_{\Delta t \to 0} \alpha_{\text{avg}} = \dv{\omega}{t} = \dv[2]{\theta}{t}
$$

If we want to calculate the linear acceleration of a point in the RB, we have to combine the centripetal acceleration and the tangential acceleration, that is

$$
    a = \begin{cases}
        a_c = \frac{v^2}{r} = r\omega^2 \\
        a_t = \dv{v}{t} = \dv{t}(\omega r) = r \dv{\omega}{t} = r\alpha
    \end{cases}
$$

\subsubsection{Summary}

We can compile a table with all the variables we have defined so far:

\begin{table}[H]
    \centering
    \begin{tabular}{|c|c|c|}
        \hline
        Variable     & 1D                 & 2D rotation                  \\
        \hline
        Position     & $x$                & $\theta$                     \\
        Velocity     & $v = \dv{x}{t}$    & $\omega = \dv{\theta}{t}$    \\
        Acceleration & $a = \dv[2]{x}{t}$ & $\alpha = \dv[2]{\theta}{t}$ \\
        \hline
    \end{tabular}
    \caption{Summary of angular variables}
    \label{tab:angular-variables}
\end{table}

and if $\alpha = \text{const}$ we have

$$
    \begin{cases}
        \theta(t) & = \theta_0 + \omega_0 t + \frac{1}{2}\alpha t^2 \\
        \omega(t) & = \omega_0 + \alpha t
    \end{cases}
$$

\subsection{Rotational kinetic energy}

First we will consider a simpler case of a discrete rigid body of $n$ masses connected by a stick around a pivot which also happens to be the center of mass.

Then the kinetic energy $K = \sum_{i=1}^n K_i$ where

\begin{align*}
    K_i & = \frac{1}{2}m_i v_i^2          \\
        & = \frac{1}{2}m_i r_i^2 \omega^2
\end{align*}

note that the $i$ subscript is only present in the mass and the radius, but not in the angular velocity, then we can write

\begin{align*}
    K & = \frac{1}{2} \omega^2 \sum_{i=1}^n m_i r_i^2 \\
      & = \frac{1}{2} I \omega^2
\end{align*}

where $I \coloneq \sum_{i=1}^n m_i r_i^2$ is the \textbf{moment of inertia} of the rigid body.

We can make some trivial observations about the moment of inertia:

\begin{itemize}
    \item It depends on $m_i$ and $r_i$: if the RB is heavier or more spread out, the moment of inertia will be bigger.
    \item It depends on the axis of rotation: if we change the pivot point the moment of inertia will change because the $r_i$ will change.
\end{itemize}

In the case of a continuous rigid body, we can see it just as a sum of an infinite number of masses, so we can write

\begin{align*}
    K & = \int_{RB} \dd{K}                          \\
      & = \int_{RB} \frac{1}{2} v^2 \dd{m}          \\
      & = \int_{RB} \frac{1}{2} r^2 \omega^2 \dd{m} \\
      & = \frac{1}{2} I \omega^2
\end{align*}

where

$$
    I \coloneq \int_{RB} r^2 \dd{m} = \int_{RB} r^2 \sigma \dd{S}
$$

where $\sigma$ is the density mass function on a surface.

Note that $r$ and $\sigma$ are functions of the position, even if we dropped the notation this does not mean that $I$ is a constant.

\begin{remark}
    For a matter of notation, we will use the symbol $\lambda$ to denote the linear density mass function, $\sigma$ to denote the surface density mass function and $\rho$ to denote the volume density mass function.
\end{remark}

\begin{example}[Moment of inertia of a ring]
    Consider a ring of radius $R$ and mass $M$ with uniform density. We want to calculate the moment of inertia with respect to the center of the ring.

    Since the density is uniform, we can write $\lambda = \frac{M}{2\pi R}$, then

    \begin{align*}
        I & = \int_{\text{ring}} r^2 \dd{m}         \\
          & = \int_{\text{ring}} R^2 \lambda \dd{r} \\
          & =  R^2 \int_{\text{ring}} \dd{m}        \\
          & = R^2 M
    \end{align*}
\end{example}

\begin{example}[Moment of inertia of a disk]
    Consider a disk of radius $R$ and mass $M$ with uniform density. We want to calculate the moment of inertia with respect to the center of the disk.

    Since the density is uniform, we can write $\sigma = \frac{M}{\pi R^2}$.

    We have two ways to compute $I$. The first one is to brute force the definition of $I$, passing to polar coordinates and solving the integral.

    With this method we get that $\dd{S} = r \dd{r} \dd{\theta}$, then

    \begin{align*}
        I & = \int r^2 \dd{m} = \int r^2 \sigma \dd{S}             \\
          & = \int r^2 \sigma r \dd{r} \dd{\theta}                 \\
          & = \sigma \int_0^{2\pi} \dd{\theta} \int_0^R r^3 \dd{r} \\
          & = 2 \pi \sigma \frac{R^4}{4} = \frac{1}{2} M R^2
    \end{align*}

    Alternatively we can sum the moment of inertia of a ring of radius $r$ and width $\dd{r}$ and use the result we got in the previous example.
    We have

    \begin{align*}
        I & = \int \dd{I_\text{ring}} = \int r^2 \dd{m_\text{ring}}             \\
          & = \int r^2 \sigma \dd{S_\text{ring}}= \int r^2 \sigma 2\pi r \dd{r} \\
          & = 2\pi \sigma \int r^3 \dd{r} = 2\pi \sigma \frac{R^4}{4}           \\
    \end{align*}

    which indeed is the same result as before at the expense of a much simpler integral.
\end{example}

\end{document}
