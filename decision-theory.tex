\documentclass[12pt]{extarticle}

\setlength{\headheight}{15pt} % ??? we do what fancyhdr tells us to do  

\title{Decision Theory and Human Behavior}
\author{Giacomo Ellero}
\date{a.y. 2024/2025}

\usepackage{preamble}

\renewcommand{\vec}[1]{\uvec{#1}}
\newcommand{\Gr}{{\operatorfont Gr} \,}

\begin{document}

\firstpage

\section{Binary relations}

\begin{definition}{Proposition}{proposition}
    A proposition is a statement that can be either true or false.
\end{definition}

\begin{definition}{Logical predicate}{predicate}
    A logical predicate $p(x)$ is an expression that contains an argument $x$ that varies in a given domain $X$.
\end{definition}

Note that if we take a predicate and choose an $x$ and fix it we get a proposition.

To turn a predicate into a proposition we can use the universal $\forall$ and existential $\exists$ quantifier.

We can also identify a predicate with the set $A$ of all the elements $x \in X$ such that the proposition $p(x)$ is true.
\begin{equation}
    \label{eq:extension_set}
    A = \{ x \in X : p(x) \text{ is true} \}
\end{equation}

Moreover we can also have predicates with two or more variables that may have different domains.
We call a predicate with two variables a \textbf{binary predicate}. We can again define a set like the one in \cref{eq:extension_set}:
\begin{equation}
    R = \{ (x,y) \in X \times Y : p(x,y) \text{ is true} \}
\end{equation}
we call this the \textbf{extension set} of $p$.

For our purposes we will always consider $X = Y$.

\begin{definition}{Binary relation}{binary-relation}
    Let $X$ be a domain. A binary relation is a subset $R$ of $X \times X$.
\end{definition}

Therefore a binary relation is the extension of a binary predicate.

\begin{remark}{Set-theoretic notation}{set-theoretic-notation}
    Instead of $(x, y) \in R$ we will instead just write
    \begin{equation}
        xRy
    \end{equation}
    It reads as \say{$x$ is in the relation $R$ with $y$}.
\end{remark}

\begin{proposition}{Properties of binary relations}{prop-binary-relations}
    Given a binary relation $R$ on a set $X$ we say that $R$ is
    \begin{enumerate}[label=\roman*.]
        \item \textbf{Reflexive} if $xRx \enspace \forall x \in X$
        \item \textbf{Transitive} if $xRy \land yRz \implies xRz$
        \item \textbf{Complete} if $\forall x, y \in R$ either $xRy$ or $yRx$ or both
        \item \textbf{Symmetric} if $xRy \implies yRx \enspace \forall x,y \in X$
        \item \textbf{Asymmetric} if $xRy \implies \lnot yRn \enspace \forall x,y \in X$
        \item \textbf{Antisymmetric} if $xRy \land yRx \implies x = y \enspace \forall x,y \in X$
    \end{enumerate}
\end{proposition}

\begin{definition}{The $I$ relation}{i-relation}
    Given a binary relation $R$ and $x, y \in X$ we define $I$ as
    \begin{equation}
        xIy \iff xRy \land yRx
    \end{equation}
\end{definition}

\begin{proposition}{Properties of $I$}{props-i}
    Let $R$ be reflexive and transitive (a \textit{preorder}) on a set $X$.
    Then the induced binary relation $I$ is reflexive, symmetric and transitive.
\end{proposition}

\begin{proof}
    We want to show that $I$ is transitive.
    Let $xIy$ and $yIz$, this is equivalent, by definition, to $xRy \land yRx$ and $yRz \land zRy$.
    By transitivity of $R$ we have that $xRy \land yRz \implies xRz$ and $yRx \land zRy \implies zRx$, meaning $xIz$.
\end{proof}

\begin{definition}{Equivalence relation}{equivalence-relation}
    A binary relation $R$ is an equivalence relation if it satisfies \textit{reflexivity}, \textit{symmetry}, and \textit{transitivity}.
\end{definition}

\begin{corollary}{}{}
    $I$ is an equivalence relation.
\end{corollary}

\begin{definition}{Equivalence class}{equivalence-class}
    Given an equivalence relation $R$ and $x \in X$ we define the equivalence class of $x$ as
    \begin{equation}
        [x] = \{y \in X : xRy\}
    \end{equation}
    that is, the set of all elements of $X$ equivalent to $x$ according to $R$.
\end{definition}

\begin{lemma}{}{}
    If $y \in [x]$, then $[y] = [x]$.
\end{lemma}

\begin{proposition}{Equivalence classes a partition}{eq-classes-partition}
    If $R$ is an equivalence relation on $X$ define
    \begin{equation}
        X/R = \{ [x] : x\in X \}
    \end{equation}
    as the set all equivalence classes in $X$ w.r.t. $R$.

    $X/R$ form a partition on $X$.
\end{proposition}

\begin{definition}{Partial order}{partial-order}
    A binary relation $R$ is a partial order if it satisfies \textit{reflexivity}, \textit{transitivity}, and \textit{antisymmetry}.
\end{definition}

A partial order is a preorder (as seen in \cref{prop:props-i}) with antisymmetry added on top.

\begin{remark}{Partially ordered set}{partially-ordered-set}
    If we are able to define a partial order $\geq$ on $X$ we call $(X, \geq)$ a \emph{partially ordered set}.
\end{remark}


\begin{remark}{Total order}{total-order}
    If $R$ is a partial order and also satisfies \emph{completeness} we call it a total order.
\end{remark}

\begin{example}{Partial order for functions}{}
    Consider the set $\R^\Omega$, the set of functions $f: \Omega \to \R$.
    One can define a partial order on $\R^\Omega$ as
    \begin{equation}
        f \geq g \implies f(\omega) \geq g(\omega) \enspace \forall \omega \in \Omega
    \end{equation}

\end{example}

\begin{example}{Partial order for sets}{}
    Let $\Omega$ be a set and $\mathcal P(\Omega)$ the power set of $\Omega$.
    The easiest partial order for sets is inclusion:
    let $A, B \in \mathcal P(\Omega)$, then
    \begin{equation}
        A \leq B \iff A \subseteq B
    \end{equation}
\end{example}

\begin{definition}{Ordered vector space}{ordered-vector-space}
    Let $V$ be a vector space and $\leq$ a partial order on $V$.
    If $\leq$ satisfies the following properties we call $(V, \leq)$ an ordered vector space:
    \begin{itemize}
        \item $v \leq w \implies v + z \leq w + z \enspace \forall v, w, z \in V$.
        \item $v \leq w \implies \alpha v \leq \alpha w \enspace \forall \alpha \geq 0$ and $\forall v, w \in V$.
    \end{itemize}
\end{definition}

Some examples of ordered vector spaces are $(\R^n, \leq)$, $(\mathscr F(\Omega, \R), \leq)$, while $(\mathcal P(\Omega), \leq)$ is not.

\begin{remark}{Order in the limit for sequences}{}
    Let $x_n$ be a sequence converging to $x$.
    Let $y \in \R$ such that $x_n \geq y \enspace \forall n \in \N^+$.
    Then, in the limit the order is preserved, therefore $x \geq y$.
\end{remark}

\section{Preference}

In this section we will have a \emph{decision maker} (DM), a set of alternatives $X$ and a binary relation $\succsim_X$ called \emph{preference}.
We write
\begin{equation}
    x \succsim_X y
\end{equation}
if DM prefers $x$ to $y$ or it is indifferent.

We can also define $\sim_X$ for indifference ($x \succsim_X y \land y \succsim_X x$) and $\succ_X$ for strict preference ($x \succsim_X y \land \lnot (y \succsim_X x)$).

We say two alternatives are \emph{comparable} if DM can have a preference, therefore either $x \succsim_X y$ or $y \succsim_X x$ or both.

\begin{proposition}{Assumptions on preference}{assumptions-preference}
    \begin{itemize}
        \item
              \emph{Extensibility}: if $X = X'$ then
              \begin{equation}
                  x \succsim_X y \iff x \succsim_{X'} y
              \end{equation}
        \item \emph{Reflexiveness}, \emph{Transitivity} and \emph{Completeness} as defined in \Cref{prop:prop-binary-relations}.
        \item \emph{Strong monotonicity}: $x \gg y \implies x \succ y$, this means that all the goods in $x$ are greater than in $y$.
        \item \emph{Archimedean}: Let $x, y, z \in X$ s.t. $x \succ y \succ z$.
              Then there exist $\alpha, \beta \in (0, 1)$ such that
              \begin{equation}
                  \alpha x + (1-\alpha) z \succ y \succ \beta x + (1-\beta)z
              \end{equation}
        \item \emph{Convexity}: Let $x, y \in X$ s.t $x \sim y$ then, $\forall \alpha$
              \begin{equation}
                  \alpha x + (1-\alpha) y \succsim x
              \end{equation}
              Moreover, this axiom could be extended to one of the following:
              \begin{itemize}
                  \item \emph{Strict convexity}: $\alpha x + (1-\alpha) y \succ x$
                  \item \emph{Affinity}: $\alpha x + (1-\alpha) y \sim x$
              \end{itemize}
    \end{itemize}
\end{proposition}

Extensibility is saying that preference does not depend on how the sets $X$ and $X'$ are described but only on the elements in them.
During this course we will assume that preference is extensional even though this is a non-trivial assumption, see the example on budget sets in the book.

Moreover, we will assume that preference relations are \emph{reflexive} and \emph{transitive}.
Transitivity is especially useful to prevent cycles and therefore allow \emph{optimization}.

\begin{corollary}{}{}
    If $\succsim$ is reflexive and transitive, then $\sim$ is an equivalence relation.
\end{corollary}

Transitivity is not a trivial assumption either because of the \emph{threshold effect} in perception:
perception usually reacts in a discrete way, even to continuous effects.

We require strong monotonicity and not just strict monotonicity because otherwise we would be assuming that we give a value to every good individually, which is sometimes not true (see the example on shoes).

\begin{corollary}{}{}
    Strong monotonicity implies strict monotonicity.
\end{corollary}

Convexity means that in general we will likely want to differentiate more.
Strict convexity is needed in order to make sure there exists a best bundle.
Affinity is not always valid but it can be useful in some scenarios.

\section{Utility}

\begin{definition}{Utility function}{utility-function}
    A function $u: X \to \R$ is a (Paretian) utility function for the preference $\succsim$ if, $\forall x, y \in X$
    \begin{equation}
        x \succsim y \iff u(x) \geq u(y)
    \end{equation}
\end{definition}

We can write indifference curves as the level curves of the utility function:
\begin{equation}
    [x] = \{ y \in X : u(x) = u(y) \}
\end{equation}

\begin{proposition}{Ordinality}{ordinality-ic}
    Let $u: X \to \R$ be an utility function for $\succsim$.
    Another function $\tilde{u}: X \to \R$ is also an utility function for $\succsim$ iff
    $\exists f: \Im u \to \R$ \emph{strictly increasing} s.t.
    \begin{equation}
        \tilde{u} = f \circ u
    \end{equation}
\end{proposition}

\begin{remark}{Strictly increasing}{strictly-increasing}
    Note that strictly increasing means that
    \begin{equation}
        x \geq y \iff f(x) \geq f(y)
    \end{equation}
    The important part here is the $\iff$.

    (No typos.)
\end{remark}

\begin{proof}
    \skiplineafterproof
    \begin{description}
        \item[$\implies$] Let $u$ be an utility function for $\succsim$.
              Then, according to \cref{rk:strictly-increasing}, we have
              \begin{equation}
                  x \succsim y \iff u(x) \geq u(y) \iff f(u(x)) \geq f(u(y))
              \end{equation}
        \item[$\impliedby$] Trivial from the definition of $\tilde{u}$.
    \end{description}
\end{proof}

A very common problem in economics is to maximize the utility function:
\begin{equation}
    \max_x u(x) \quad {\operatorfont sub} \enspace  x \in C
\end{equation}

\subsection{Existence}

\begin{theorem}{Existence of utility functions}{exisence-utility}
    A preference relation has a utility representation only if it is
    \emph{complete} and \emph{transitive}.
\end{theorem}

\begin{proof}
    Let $\succsim$ be a preference and $u$ one of its utility function.

    First let's show that $\succsim$ is transitive.
    Let $x, y, z \in X$ s.t. $x \succsim y$ and $y \succsim z$. By hypothesis $u(x) \geq u(y)$ and $u(y) \geq u(z)$ hence $u(x) \geq u(z)$, which implies, again by hypothesis, that $x \succsim z$.

    Completeness can be proved similarly by going to scalars and back.
\end{proof}

\begin{theorem}{Quotient space of preference}{quotient-space-preference}
    Let $\succsim$ be a preference relation on $X$. Then an utility function function exists if $X/\sim$ has at most the cardinality of $\R$.

    Moreover, if $X / \sim$ is at most countable then $\succsim$ is complete and transitive iff it has a utility function.
\end{theorem}

\begin{proof}
    We will prove just the second part in the case $X$ finite: we have a complete and transitive preference over a finite set $X$ and we want to show that an utility function exists.

    Let $L(x, \succsim) = \{ z \in X : x \succsim z \}$, then define $u: X \to \R$ as
    \begin{equation}
        u(x) = \abs{L\left(x, \succsim\right)}
    \end{equation}
    the cardinality of $L$.
    Of course if $x \succsim y$ then $\abs{L\left(x, \succsim\right)} \geq \abs{L\left(y, \succsim\right)}$ and the converse is true as well.
\end{proof}

\subsubsection{Cantor properties}

\begin{definition}{$\succsim$-order dense}{order-dense}
    Let $\succsim$ be binary relation on $X$, then $Z \subseteq X$ is $\succsim$-order dense if, $\forall x, y \in X$ with $x \succ y$, there exists $z \in Z$ s.t.
    \begin{equation}
        x \succsim z \succsim y
    \end{equation}
\end{definition}

\begin{theorem}{Cantor-Debreau theorem}{cantor-debreau}
    Let $\succsim$ be a complete and transitive preference on $X$.
    Then, the following conditions are equivalent:
    \begin{enumerate}[label=\roman*.]
        \item $\succsim$ has an $\succsim$-order dense $Z \subseteq X$ which is at most countable.
        \item $\succsim$ admits an utility function.
    \end{enumerate}
\end{theorem}

\subsubsection{Quasiconcavity}

\begin{definition}{Quasiconvexity and quasiconcavity}{quasiconcavity}
    Let $V$ be a vector space and $f: V \to \R$.
    $f$ is quasiconcave if
    \begin{equation}
        f(\alpha \vec x + (1- \alpha) \vec y) \geq \min \{ f(\vec x), f(\vec y) \}
    \end{equation}
    for $\vec x, \vec y \in V$ and $\alpha \in (0,1)$.

    Similarly quasiconvexity is defined as
    \begin{equation}
        f(\alpha \vec x + (1- \alpha) \vec y) \leq \max \{ f(\vec x), f(\vec y) \}
    \end{equation}
\end{definition}

\begin{remark}{}{}
    Concavity implies quasiconcavity.
\end{remark}
\begin{remark}{}{}
    Increasing implies quasiconcavity.
\end{remark}

\begin{proposition}{}{}
    For a function $f: C \to \R$ the following conditions are equivalent:
    \begin{enumerate}[label=\roman*.]
        \item $f$ is quasi concave
        \item The sets $U(t) = \{ x \in C : f(x) \geq t \}$ are (strictly) convex for all $t \in \R$
    \end{enumerate}
\end{proposition}

\begin{theorem}{}{}
    Let $\succsim$ be a preference on $\R^n_+$.
    Then, the following conditions are equivalent:
    \begin{enumerate}[label=\roman*.]
        \item $\succsim$ is transitive, complete, strongly monotone and Archimedean.
        \item There exists a strongly monotone and continuous function $u: \R^n_+ \to \R$ which is an utility function for $\succsim$.
    \end{enumerate}

    Moreover, $\succsim$ is (strictly) convex iff $u$ is (strictly) quasiconcave.
\end{theorem}

\begin{proposition}{}{}
    Let $f: C \to \R$ and $\varphi: D \in \R \to \R$ be two functions defined on convex sets, with $\Im f \in D$.
    If $f$ is quasiconcave and $\varphi$ is increasing, then
    \begin{equation}
        \varphi \circ f \text{ is quasiconcave}
    \end{equation}
\end{proposition}

\subsubsection{Lexicographic preferences}

Let $X = \R^2$. Define $x \succsim y$ if either
\begin{equation}
    x_1 > y_1 \quad \text{or} \quad x_1 = y_1 \text{ and } x_2 \geq y_2.
\end{equation}

This is the preference system used, for example, in sorting words in a dictionary.

\begin{proposition}{}{}
    Lexicographic preferences have no utility function.
\end{proposition}

\begin{proof}
    Assume that an utility function $u$ exists.
    Let $r, r' \in \R$ s.t. $r < r'$.
    For every $x \in \R$ we have
    \begin{equation}
        u(x, r) > u(x, r')
    \end{equation}

    For any $x, y \in \R$ s.t. $x > y$ there exists $q(x), q(y) \in \Q$ s.t.
    \begin{equation}
        u(y, r') < q(y) < u(y, r) < u(x, r') < q(x) < u(x, r)
    \end{equation}
    therefore $q(x) \ne q(y)$.
    But this would mean that we have constructed an injective function $q : \R \to \Q$ which is impossible.
\end{proof}

\section{Rational choice}

\begin{definition}{Decision framework}{decision-framework}
    Let $\vec X$ be a choice space of alternatives.
    A decision framework is the pair $(\vec X,\mathcal X)$ where $\mathcal X$ is a set of $X \subset \mathcal X$.
\end{definition}

In consumer theory we could have a decision framework where $\vec X$ is the set of all possible bundles of goods and $\mathcal X$ is the collection of budget sets $X$.

Another example could be $\vec X$ is the set of all the restaurant meals available in town, each $X$ is the menu of each restaurant and $\mathcal X$ is the collection of restaurants in town.

A decision maker has a preference $\succsim_X$ over each set $X$ which we will consider \emph{reflexive} and \emph{transitive}.
Let $P: X \mapsto \succsim_X$ be the \emph{preference map} that given a set $X$ gives the DM's preference on that set.

\begin{definition}{Decision environment}{decision-environment}
    A decision framework and a preference map make a decision environment:
    \begin{equation}
        (\vec X, \mathcal X, P)
    \end{equation}
\end{definition}

A \emph{decision problem} is a pair $(X, \succsim_X)$, the DM wants to find the optimal alternative according to $\succsim_X$

\begin{definition}{Optimal alternative}{optimal}
    An alternative $\hat x \in X$ is optimal if $\nexists x \in X$
    \begin{equation}
        x \succsim_X \hat  x
    \end{equation}
\end{definition}

\begin{theorem}{Existance of optimal choice}{existance-optimal}
    If $\succsim_X$ is a preorder, an optimal alternative if the sets
    \begin{equation}
        U(x) = \{ z \in X : z \succsim_X x \}
    \end{equation}
    are compact $\forall x \in X$.
\end{theorem}

\begin{proof}
    First we need to introduce a new mathematical tool, the correspondence.
    \begin{equation}
        F:X  \rightrightarrows Y
    \end{equation}
    where $F(x) \in \mathcal P(Y)$.

    This is which is basically just a \say{multivalued function} returning a set of sets: a function is just a single-valued correspondence.

\end{proof}

\begin{corollary}{}{}
    If $\succsim_X$ is a preorder, optimal alternatives exists if $X$ is finite.
\end{corollary}

\begin{proof}
    Finite sets are compact, hence we can apply the theorem directly.
\end{proof}

\begin{definition}{Rational choice correspondence}{rational-choice-correspondence}
    The rational choice correspondence $\sigma : \mathcal D \rightrightarrows \vec X$ is defined as
    \begin{equation}
        \sigma (X) = \{ \hat x \in X: \nexists x \in X, x \succ_X \hat x \}
    \end{equation}
    where $\mathcal D$ is the collection of choice sets that admits optimal alternatives.
\end{definition}

\begin{lemma}{}{indifference-optimal}
    If $\succsim_X$ is a preorder, optimal alternatives are pairwise either incomparable ($x \parallel y$) or indifferent.
\end{lemma}

If $\succsim_X$ is complete and transitive (\emph{weak order}) then
\begin{equation}
    \sigma(X) = \{\hat x \in X : \forall x \in X, \hat x \succsim_X x\}
\end{equation}
If $\succsim_X$ is a weak order optimal alternatives are indifferent.

\begin{proposition}{}{}
    If $X$ is convex and $\succsim_X$ is a weak order
    \begin{enumerate}[label=\roman*.]
        \item if $\succsim_X$ is convex, then $\sigma(X)$ is convex-valued;
        \item if $\succsim_X$ is strictly convex, then $\sigma(X)$ is a singleton.
    \end{enumerate}
\end{proposition}

\begin{proof}
    \skiplineafterproof
    \begin{enumerate}[label=\roman*.]
        \item Let $\succsim_X$ be convex.
              Let $x, y \in \sigma (X)$ and $\alpha \in (0,1)$. Then by \cref{lemma:indifference-optimal} $x \sim y$.
              By convexity of $\succsim$ we have
              \begin{equation}
                  x \sim y \implies \alpha x + (1- \alpha y) \succsim x
              \end{equation}
              hence
              $x \sim y \implies \alpha x + (1- \alpha y) \in \sigma(X)$.

        \item Let $\succsim$ be strictly convex.
              By contradiction, suppose that $x, y \in \sigma(X)$ and $x \neq y$.
              Again by \cref{lemma:indifference-optimal} $x \sim y$ and let $z = 0.5 x + 0.5 y$.
              By strict convexity $z \succsim x$ but is a contradiction because $x$ is optimal by hypothesis.
    \end{enumerate}
\end{proof}

\begin{definition}{Menu preference}{menu-preference}
    The menu (or indirect preference) $\succeq$ is defined between choice sets in $\mathcal D$ as
    \begin{equation}
        X \succeq Y \iff \forall y \in Y, \exists x \in X \text{ s.t. } x \succsim y
    \end{equation}
\end{definition}

\begin{proposition}{}{}
    If $\succsim$ is a weak order, then $\succeq$ is a weak order too.
    Moreover, for any $X, Y \in \mathcal D$:
    \begin{enumerate}[label=\roman*.]
        \item $X \sim \sigma (X)$
        \item $X \subseteq Y \implies X \succeq Y$
        \item $X \succeq Y \implies X \sim X \cup Y$
    \end{enumerate}
    where $X \sim Y$ means $X \succeq Y$ and $Y \succeq X$.
\end{proposition}

Since $\hat x \sim \sigma(X)$ we also have that $\hat x \sim X$.
This means that for the menu preference any optimal alternative is indifferent to the whole choice set.

\begin{corollary}{}{}
    \begin{equation}
        X \succeq Y \iff \sigma(X) \succeq \sigma(Y)
    \end{equation}
\end{corollary}

Assuming $\succsim$ has an utility function $u$ we have that
\begin{equation}
    \sigma(X) = \argmax_{x \in X} u(x)
\end{equation}
and
\begin{equation}
    X \succeq Y \iff \max_{x \in X} (x) \geq \max_{y \in Y} u(y)
\end{equation}

\begin{definition}{Value function}{value-function}
    The value function $v: \mathcal D \to \R$ is defined as
    \begin{equation}
        v(X) = \max_{x \in X} u(x)
    \end{equation}
\end{definition}

This means that the value function represents the menu preference:
\begin{equation}
    X \succsim Y \iff v(x) \geq v(Y)
\end{equation}

\subsection{Contextualized analysis}

A contextualized alternative is a pair $(x, X)$ with $X \in \mathcal X$ and $x \in X$.
We call $\mathcal C$ the set of all contextualized alternatives.

A preference relation on $\mathcal C$ looks like
\begin{equation}
    (x, X) \succsim (y, Y)
\end{equation}
which means the DM prefers $x$ in the context $X$ over $y$ in the context $Y$.

A contextualized preference over $\mathcal C$ implies that $\forall X \in \mathcal X$ and $x, x' \in X$ it holds
\begin{equation}
    (x, X) \succsim (x', X) \iff x \succsim_X x'
\end{equation}

If $\succsim$ is a weak order we can define the optimal in the case of contextualized alternatives as in \Cref{def:rational-choice-correspondence}.

Moreover, the menu preference is induced by a contextualized universal preference:
\begin{equation}
    X \succeq Y \iff \forall y \in Y, \exists x \in X \text{ s.t. } (x, X) \succsim (y, Y)
\end{equation}
forall $X, Y \in \mathcal D$.

\subsection{Parametric analysis}

Choice sets are usually parametrized by elements of some set $\Theta$ (e.g. for budgets sets we have parameters of the form $(p, w) \in \R^2_+$).
Parametrization is carried out using a \emph{menu correspondence}:
\begin{equation}
    \varphi: \Theta \rightrightarrows \vec X
\end{equation}
thus
\begin{equation}
    \mathcal X = \{ \varphi(\theta) : \theta \in \Theta \} \quad \text{and} \quad \mathcal D = \{ \varphi(\theta) : \theta \in D \}
\end{equation}

The contextualized universal preference $\succsim$ is a binary relation over pairs
\begin{equation}
    (x, \theta) \in \Gr \varphi
\end{equation}

Some preferences $\succsim$ can also admit a parametric utility function of the form $u : \Gr \varphi \to \R$ such that, for all pairs $(x, \theta) \in \Gr \phi$
\begin{equation}
    (x, \theta) \succsim (x', \theta') \iff u(x, \theta) \geq u(x', \theta')
\end{equation}
and, for a fixed $\theta \in \Theta$ we have
\begin{equation}
    x \succsim_\theta x' \iff u(x,\theta) \geq u(x', \theta') \quad \forall x,x' \in \varphi(\theta)
\end{equation}

We can then generalize the optimization problem to take in account a parameter $\theta$:
\begin{equation}
    \max_x u(x) \quad {\operatorfont sub} \, x \in \varphi (\theta)
\end{equation}

Moreover we extend $\sigma$ and $v$ as follows:
$\sigma : D \rightrightarrows \vec X$ and $v: D \to \R$
\begin{gather}
    \sigma(\theta) = \sigma(\varphi(\theta)) = \{ \hat x \in \varphi(\theta) : \forall x \in \varphi(\theta), u(\hat x, \theta) \geq u(x, \theta)\} \\
    v(\theta) = v(\varphi(\theta)) = \max_{x \in \varphi (\theta)} u(x, \theta)
\end{gather}

\begin{theorem}{Maximum theorem}{maximum-theorem}
    Let $\vec X$ and $\Theta$ be metrizable (?).
    If the parametric utility function $u: \Gr \varphi \to \R$ and the menu correspondence $\varphi: \Theta \rightrightarrows \vec X$ are both continuous then $D = \Theta$ and
    \begin{enumerate}[label=\roman*.]
        \item the rational choice correspondence $\sigma \Theta \rightrightarrows \vec X$ is compact-valued and upper continuous when it is a function;
        \item the value function $v: \Theta \to \R$ is continuous
    \end{enumerate}
\end{theorem}

\section{Certainty}

First we define the action space $A$, the consequence space $C$, a collection $\mathcal A \in \mathcal P(A)$ and a consequence function $\rho: A \to C$.
Summing up we have $(A, \mathcal a, C, \rho)$.

DMs can have a preference over actions $\succsim$
\begin{equation}
    \rho(a) = \rho(b) \implies a \sim b
\end{equation}
but can also have a preference over consequences $\dot \succsim$.

\begin{theorem}{Outcome consequentialism}{outcome-consequentialism}
    There is a bijective mapping
    \begin{equation}
        \dot \succsim \enspace \mapsto \enspace \succsim
    \end{equation}
    if both are preorders.
\end{theorem}

\section{Decision under uncertainty}

\emph{This is macchero's part.}

TODO: missing class of 03/10

\subsection{Axioms}

We have $\bm C$ the set of material consequences, $\bm L$ the set of function $\ell: \bm C \to \R^+$ which are discrete probability mass functions on $\bm C$.

On $\bm L$ we can define preferences.
As usually we will define some axioms on preference:
\begin{enumerate}
    \item \emph{Weak order}: completeness and transitivity
    \item \emph{Independence}: for any three lotteries $l, l', l''$ in $\bm L$ and $0<p<1$ we have
          \begin{equation}
              l \succ l' \implies pl + (1-p) l'' \succ pl' + (1- p)l''
          \end{equation}

    \item \emph{Archimedean}: for any three lotteries $l, l', l''$ in $\bm L$ there exists $q, p \in (0, 1)$ such that
          \begin{equation}
              pl + (1-p)l'' \succ l' \succ ql + (1-q)l''
          \end{equation}
\end{enumerate}

Now assume that beyond the DM's preference, there exists a meaningful order to the space $\bm C$.
Then we can define also \emph{monotonicity} in the same way as we did for normal preferences.

\subsection{Utility functions}

\begin{theorem}{von Neumann-Morgentern Representation Theorem}{representation-theorem}
    Let $\succsim$ be a preference on $\bm L$ on the space $\bm C$.
    The following are equivalent:
    \begin{enumerate}
        \item $\succsim$ satisfies the axioms described above
        \item There exists a function $u:\bm C \to \R$ such that the function $\ddot u : \bm L \to \R$ defined as
              \begin{equation}
                  \ddot u(l) = \sum^n_{i = 1} u(c_i) p_i
              \end{equation}
              or equivalently
              \begin{equation}
                  U(l) = \sum_{c \in C} u(c) l(c)
              \end{equation}
              which represents $\succsim$.
        \item There exists $f:\bm L \to \R$ affine such that
              \begin{equation}
                  l \succsim l' \iff f(l) \geq f(l')
              \end{equation}
    \end{enumerate}
\end{theorem}

\begin{lemma}{}{}
    A function $f:\bm L \to \R$ is affine iff $\exists \varphi : C \to \R$ unique s.t. $f(l) = \sum_{c \in C} \varphi(c) l(c)$, where $\varphi = f \circ \delta$.

    This means that $f$ is affine only if it is an expectation of some $l$.
\end{lemma}
\begin{proof}[Proof of lemma]
    The macchero did it but I didn't write it :D
\end{proof}

\begin{proof}[Proof of \cref{thm:representation-theorem}]
    We will just proof point $2 \iff 3$.

    Recall that a function is affine if $f(\lambda k_1 + (1- \lambda) k_2) = \lambda f (k_1) + (1 - \lambda ) f(k_2)$ for $\lambda \in (0, 1)$.
    Note that affinity holds even for multiple $\lambda$ such that $\sum \lambda = 1$.

    Given the lemma above we can easily prove $3 \implies 2$ by choosing $\varphi = u$ and $f = U$.
    $2 \implies 3$ \say{grazie al cazzo} by taking $f = U$.
\end{proof}

We now want to show that $u$ is an utility of consumption.
To do so we will assume that $c \succsim c' \iff \delta_c \succsim \delta_{c'}$ but this means that $U(\delta_c) \geq U(\delta_{c'})$.
From here we can apply \cref{thm:representation-theorem} and get that $c \succsim c' \iff u(c) \geq u(c')$.

\subsubsection{Robustness}

\emph{Not in the exam?}

This is not a perfect representation of reality, usually we are not sure of the probability, we can \emph{estimate} the probability, but we cannot be sure.

\begin{equation}
    U_D(l) = \min_{m \in L} \left( U(m) + D(m \mid l) \right)
\end{equation}
where $l$ is our estimate, $m$ are other models and $D$ is a distance function.

\subsubsection{Cardinality}

The vN-M utility function $u$ on prizes is cardinal.
This means that if there exists two distinct $u$ such that
\begin{equation}
    l \succsim m \iff \sum u(c) l(c) \geq \sum u(c) l(c)
\end{equation}
then $\exists \alpha > 0, \beta \in \R$ s.t. $u = \alpha v + \beta$.

We have that the order of the differences over cardinal transformations is preserved:
\begin{equation}
    u(c_1) - u(c_2) \geq u(c'_1) - u(c'_2)
\end{equation}
will still hold under cardinal transformations.
This is equivalent to
\begin{equation}
    \left\{ (c_1, 0.5),  (c_2, 0.5) \right\} \succsim \left\{ (c'_1, 0.5),  (c'_2, 0.5) \right\}
\end{equation}

\subsection{Probability??}

Here the macchero went on a tangent explaining some basic stuff about discrete probability measures, their support, and expectation.
Idk, it seems like this is not in the lecture notes.

We can define lotteries as the restriction of a simple probability measure $p$ on $\bm C$.

We define $P_l = \sum_{c \in A} l(c)$ to be the probability of getting a prize $c \in A$ given a lottery $l$, with $A \subseteq \bm C$.

\begin{proposition}{}{}
    $P$ is a simple probability measure with finite carrier $A$ ($P(A)=1$, $A$ finite) iff $\exists l \in L$ s.t.
    \begin{equation}
        P(A) = \sum_{c \in A} l(c)
    \end{equation}
    Moreover, $l$ is unique and $l(c) = P(\{c\})$.
\end{proposition}

\end{document}
