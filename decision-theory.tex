\documentclass[12pt]{extarticle}

\setlength{\headheight}{15pt} % ??? we do what fancyhdr tells us to do  

\title{Decision Theory and Human Behavior}
\author{Giacomo Ellero}
\date{a.y. 2024/2025}

\usepackage{preamble}

\renewcommand{\vec}[1]{\uvec{#1}}
\newcommand{\Gr}{{\operatorfont Gr} \,}

\begin{document}

\firstpage

\section{Decision problems}

We will discuss how a Decision Maker (DM) makes decisions.

\subsection{Preference}

\subsubsection{Binary relations}

A binary relation in a set $X$ is defined as a subset $R \subset X \cross X$ where a certain predicate is true:
\begin{equation}
	R = \{ (x, y) \in X \cross X : p(x) \text{ is true}\}
\end{equation}
If $(x, y) \in R$ we write $xRy$.

\begin{definition}{Properties of binary relations}{props-relations}
	A binary relation $R$ on $X$ can have one or more of the following properties:
	\begin{enumerate}
		\item \textbf{Reflexivity}: $xRx$
		\item \textbf{Transitivity}: $xRy \land yRz \implies xRz$
		\item \textbf{Completeness}: either $xRy$ or $yRx$ or both
		\item \textbf{Symmetry}: $xRy \implies yRx$
		\item \textbf{Asymmetry}: $xRy \implies \lnot yRn$
		\item \textbf{Antisymmetry}: $xRy \land yRx \implies x = y$
	\end{enumerate}
	where $x, y, z \in X$.

	Moreover, if a relation has some of the properties above we give it a specific name:
	\begin{enumerate}
		\item \textbf{Preorder}: reflexivity and transitivity;
		\item \textbf{Weak order}: reflexivity, transitivity, and completeness;
		\item \textbf{Partial order}: reflexivity, transitivity, and antisymmetry;
		\item \textbf{Partial order}: reflexivity, transitivity, antisymmetry, and completeness;
		\item \textbf{Equivalence}: reflexivity, symmetry, and transitivity;
	\end{enumerate}
\end{definition}

Note that completeness implies reflexivity.

The equivalence class of $R$ is
\begin{equation}
	[x] = \{ y \in X : xRy \}
\end{equation}
If $y \in [x]$, then $[x] = [y]$.

The set of equivalence classes form a partition:
\begin{equation}
	X/R = \{ [x] : x \in X \}
\end{equation}

\subsubsection{Axioms of preference}

Now consider a set of alternatives $X$ where the DM has to make a choice upon.
We consider the binary relation $\succsim$
\begin{equation}
	x \succsim y
\end{equation}
which means the DM prefers $x$ to $y$ or it is indifferent.
Similarly, define $\sim$ for indifference ($x \succsim y \land y \succsim x$) and
$\succ_X$ for strict preference ($x \succsim y \land \lnot (y \succsim x)$).

Depending on the situation we might be able to assume certain axioms on the preference relation.
Consider $x, y, z \in X$.
\begin{description}
	\item[Extensionality]
	      If $X = X'$ then $x \succsim_X y \iff x \succsim_{X'} y$.
	      This is particularly useful when $X$ and $X'$ have different representations (e.g. budget sets)
	      but they are actually the same set.
	      We usually assume this property as it means that the DM is able to identify equal sets of alternatives
	      and therefore does not suffer of \emph{framing effects}.

	\item[Reflexivity] As in \cref{def:props-relations}.
	\item[Transitivity] As in \cref{def:props-relations}.
	      This axioms tells us that there are no cycles,
	      therefore it enables optimization.
	      \begin{description}[font=\normalfont\itshape\space]
		      \item[Strict transitivity] As $x \succ y \land y \succ z \implies x \succ z$.
	      \end{description}
	\item[Completeness] As in \cref{def:props-relations}.
	      This axiom imply that the DM is always able to choose between any alternative.

	\item[Monotonicity] For this property we assume there exists a topology $\tau$ on $X$
	      and we are therefore able to decide whether $x \geq y$.
	      Monotonicity says that
	      \begin{equation}
		      x \geq y \implies x \succsim y
	      \end{equation}

	      In case of $\R^n$ we can distinguish the following axioms as well.
	      \begin{description}[font=\normalfont\itshape\space]
		      \item[Strong monotonicity] \say{The more of all goods is definitely better}
		            \begin{equation}
			            x \gg y \implies x \succ y \quad \text{and} \quad x \geq y \implies x\succsim y
		            \end{equation}
		      \item[Strict monotonicity] \say{The more of at least one good is still better}
		            \begin{equation}
			            x > y \implies x \succ y
		            \end{equation}
	      \end{description}

	      These axioms are a way to tell us that the goods are actually \say{goods}.

	\item[Archimedean] If $x \succ y \succ z$ then $\exists \alpha, \beta \in (0, 1)$ such that
	      \begin{equation}
		      \alpha x + (1- \alpha)z \succ y \succ \beta x + (1-\beta) z
	      \end{equation}

	      This axiom tells us that there is no good that it is preferred infinitely more than any other.

	\item[Convexity] If $x \sim y$ for all $\alpha \in (0, 1)$ there holds
	      \begin{equation}
		      \alpha x + (1-\alpha y) \succsim x
	      \end{equation}
	      which means that the DM likes to differentiate as much as possible.

	      \begin{description}[font=\normalfont\itshape\space]
		      \item[Strict convexity] For \emph{distinct} $x, y$
		            \begin{equation}
			            \alpha x + (1-\alpha y) \succ x
		            \end{equation}

		            This axiom is particularly useful as it ensure the uniqueness of an optimal choice.
		      \item[Affinity]
		            \begin{equation}
			            \alpha x + (1-\alpha y) \sim x
		            \end{equation}

	      \end{description}
\end{description}

\begin{lemma}{Strict monotonicity implies strong monotonicity}{strict-strong-monotonicity}
	Let $\succsim$ be reflexive.
	If $\succsim$ is strictly monotone then it is also strongly monotone.

	The converse is true only if $d = 1$.
\end{lemma}

\begin{proof}
	Suppose $\succsim$ is strictly monotone.
	If $x\gg y$ then also $x \gt y$ and by strictly monotonicity we get $x \succ y$.
	If $x\geq y$ then either $x = y$, and we get that $x \succsim y$ by reflexivity,
	or $x > y$, and by strict monotonicity we have $x\succ y$ and therefore $x \succsim y$.

	The converse is easily proven by noticing that in $\R$ $\gg$ and $\gt$ mean the same thing.
\end{proof}


\subsection{Utility functions}

\begin{definition}{Utility function}{utility-function}
	A function $u: X \to \R$ is a (Paretian) utility function for the preference $\succsim$ if, $\forall x, y \in X$
	\begin{equation}
		x \succsim y \iff u(x) \geq u(y)
	\end{equation}
\end{definition}

We can write indifference curves as the level curves of the utility function:
\begin{equation}
	[x] = \{ y \in X : u(x) = u(y) \}
\end{equation}

Moreover, utility functions are unique up to a strictly increasing transformation:
if $u$ is an utility function and $f$ is strictly increasing
(which means that $x \geq y \iff f(x) \geq f(y)$, note the $\iff$)
then $f \circ u$ is also an utility function for the same preference.
We refer to this property as \textbf{ordinality}.

A very common problem in economics is to maximize the utility function:
\begin{equation}
	\max_{x \in C} u(x)
\end{equation}
where $C \subseteq X$ is the constraint.

\subsubsection{Existence}

\begin{proposition}{}{}
	A preference relation has a utility representation only if it is
	\emph{complete} and \emph{transitive}.
\end{proposition}

\begin{proof}
	Let $\succsim$ be a preference and $u$ one of its utility function.

	First let's show that $\succsim$ is transitive.
	Let $x, y, z \in X$ s.t. $x \succsim y$ and $y \succsim z$.
	By hypothesis $u(x) \geq u(y)$ and $u(y) \geq u(z)$ hence $u(x) \geq u(z)$,
	which implies, again by hypothesis, that $x \succsim z$.

	Completeness can be proved similarly by going to scalars and back.
\end{proof}

\begin{proposition}{}{}
	Let $\succsim$ be a preference relation on $X$.
	Then an utility function function exists if $X/\sim$ has at most the cardinality of $\R$.
\end{proposition}

\begin{proposition}{}{}
	Let $x, y \in \R^n$. Explain the difference between $x > y$, $x \geq y$ and $x \gg y$. Make examples.
\end{proposition}

\begin{proof}
	We will prove the proposition just in the case $X$ finite.
	Consider a complete and transitive preference over a finite set $X$:
	we want to show that an utility function exists.

	Let $L(x, \succsim) = \{ z \in X : x \succsim z \}$, then define $u: X \to \R$ as
	\begin{equation}
		u(x) = \abs{L\left(x, \succsim\right)}
	\end{equation}
	the cardinality of $L$.
	Of course if $x \succsim y$ then $\abs{L\left(x, \succsim\right)} \geq \abs{L\left(y, \succsim\right)}$
	and the converse is true as well.
\end{proof}

\begin{definition}{$\succsim$-order dense}{order-dense}
	Let $\succsim$ be binary relation on $X$, then $Z \subseteq X$ is $\succsim$-order dense if, $\forall x, y \in X$ with $x \succ y$, there exists $z \in Z$ s.t.
	\begin{equation}
		x \succsim z \succsim y
	\end{equation}
\end{definition}

\begin{theorem}{Cantor-Debreau theorem}{cantor-debreau}
	Let $\succsim$ be a complete and transitive preference on $X$.
	Then, the following conditions are equivalent:
	\begin{enumerate}[label=\roman*.]
		\item $\succsim$ has an $\succsim$-order dense $Z \subseteq X$ which is at most countable.
		\item $\succsim$ admits an utility function.
	\end{enumerate}
\end{theorem}

\begin{theorem}{}{}
	Let $\succsim$ be a preference on $\R^n_+$.
	Then, the following conditions are equivalent:
	\begin{enumerate}[label=\roman*.]
		\item $\succsim$ is transitive, complete, strongly monotone and Archimedean.
		\item There exists a strongly monotone and continuous function $u: \R^n_+ \to \R$ which is an utility function for $\succsim$.
	\end{enumerate}

	Moreover, $\succsim$ is (strictly) convex iff $u$ is (strictly) quasiconcave.
\end{theorem}

\begin{proof}
	TODO: ot says it should be done but i don't think we did it in class.
\end{proof}

\subsubsection{Lexicographic preferences}

Let $X = \R^2$. Define $x \succsim y$ if either
\begin{equation}
	x_1 > y_1 \quad \text{or} \quad x_1 = y_1 \text{ and } x_2 \geq y_2.
\end{equation}

This is the preference system used, for example, in sorting words in a dictionary.

\begin{proposition}{}{}
	Lexicographic preferences have no utility function.
\end{proposition}

\begin{proof}
	Assume that an utility function $u$ exists.
	Let $r, r' \in \R$ s.t. $r < r'$.
	For every $x \in \R$ we have
	\begin{equation}
		u(x, r) > u(x, r')
	\end{equation}

	For any $x, y \in \R$ s.t. $x > y$ there exists $q(x), q(y) \in \Q$ s.t.
	\begin{equation}
		u(y, r') < q(y) < u(y, r) < u(x, r') < q(x) < u(x, r)
	\end{equation}
	therefore $q(x) \ne q(y)$.
	But this would mean that we have constructed an injective function $q : \R \to \Q$ which is impossible.
\end{proof}

\subsection{Rational choice}

\begin{definition}{Decision framework}{decision-framework}
	Let $\bm X$ be a choice space of alternatives.
	A decision framework is the pair $(\bm X,\mathcal X)$
	where $\mathcal X$ is the non-empty subset $\mathcal X \subset \bm X$ containing alternatives $X$.
\end{definition}

In consumer theory we could have a decision framework where $\vec X$ is the set of all possible bundles of goods and $\mathcal X$ is the collection of budget sets $X$.

Another example could be $\vec X$ is the set of all the restaurant meals available in town, each $X$ is the menu of each restaurant and $\mathcal X$ is the collection of restaurants in town.

A decision maker has a preference $\succsim_X$ over each set $X$
which we will consider \emph{reflexive} and \emph{transitive}.
Let $P: X \mapsto \succsim_X$ be the \emph{preference map} that given a set $X$ gives the DM's preference on that set.
We call the triplet $(\bm X, \mathcal X, P)$ a \textbf{decision environment}.

A \emph{decision problem} is a pair $(X, \succsim_X)$,
the DM wants to find the optimal alternative according to $\succsim_X$.

\begin{definition}{Optimal alternative}{optimal}
	In a decision problem an alternative $\hat x \in X$ is optimal if $\nexists x \in X$
	\begin{equation}
		x \succsim_X \hat  x
	\end{equation}
\end{definition}

\begin{theorem}{Existance of optimal choice}{existance-optimal}
	If $\succsim_X$ is a preorder, an optimal alternative if the sets
	\begin{equation}
		U(x) = \{ z \in X : z \succsim_X x \}
	\end{equation}
	are compact $\forall x \in X$.
\end{theorem}

\begin{proof}
	First we need to introduce a new mathematical tool, the correspondence.
	\begin{equation}
		F:X  \rightrightarrows Y
	\end{equation}
	where $F(x) \in \mathcal P(Y)$.

	This is which is basically just a \say{multivalued function} returning a set of sets: a function is just a single-valued correspondence.

\end{proof}

\begin{corollary}{}{}
	If $\succsim_X$ is a preorder, optimal alternatives exists if $X$ is finite.
\end{corollary}

\begin{proof}
	Finite sets are compact, hence we can apply the theorem directly.
\end{proof}

\begin{definition}{Rational choice correspondence}{rational-choice-correspondence}
	The rational choice correspondence $\sigma : \mathcal D \rightrightarrows \vec X$ is defined as
	\begin{equation}
		\sigma (X) = \{ \hat x \in X: \nexists x \in X, x \succ_X \hat x \}
	\end{equation}
	where $\mathcal D$ is the collection of choice sets that admits optimal alternatives.
\end{definition}

\begin{lemma}{}{indifference-optimal}
	If $\succsim_X$ is a preorder, optimal alternatives are pairwise either incomparable ($x \parallel y$) or indifferent.
\end{lemma}

If $\succsim_X$ is complete and transitive (\emph{weak order}) then
\begin{equation}
	\sigma(X) = \{\hat x \in X : \forall x \in X, \hat x \succsim_X x\}
\end{equation}
If $\succsim_X$ is a weak order optimal alternatives are indifferent.

\begin{proposition}{}{}
	If $X$ is convex and $\succsim_X$ is a weak order
	\begin{enumerate}[label=\roman*.]
		\item if $\succsim_X$ is convex, then $\sigma(X)$ is convex-valued;
		\item if $\succsim_X$ is strictly convex, then $\sigma(X)$ is a singleton.
	\end{enumerate}
\end{proposition}

\begin{proof}
	\skiplineafterproof
	\begin{enumerate}[label=\roman*.]
		\item Let $\succsim_X$ be convex.
		      Let $x, y \in \sigma (X)$ and $\alpha \in (0,1)$. Then by \cref{lemma:indifference-optimal} $x \sim y$.
		      By convexity of $\succsim$ we have
		      \begin{equation}
			      x \sim y \implies \alpha x + (1- \alpha y) \succsim x
		      \end{equation}
		      hence
		      $x \sim y \implies \alpha x + (1- \alpha y) \in \sigma(X)$.

		\item Let $\succsim$ be strictly convex.
		      By contradiction, suppose that $x, y \in \sigma(X)$ and $x \neq y$.
		      Again by \cref{lemma:indifference-optimal} $x \sim y$ and let $z = 0.5 x + 0.5 y$.
		      By strict convexity $z \succsim x$ but is a contradiction because $x$ is optimal by hypothesis.
	\end{enumerate}
\end{proof}

\begin{definition}{Menu preference}{menu-preference}
	The menu (or indirect preference) $\succeq$ is defined between choice sets in $\mathcal D$ as
	\begin{equation}
		X \succeq Y \iff \forall y \in Y, \exists x \in X \text{ s.t. } x \succsim y
	\end{equation}
\end{definition}

\begin{proposition}{}{}
	If $\succsim$ is a weak order, then $\succeq$ is a weak order too.
	Moreover, for any $X, Y \in \mathcal D$:
	\begin{enumerate}[label=\roman*.]
		\item $X \sim \sigma (X)$
		\item $X \subseteq Y \implies X \succeq Y$
		\item $X \succeq Y \implies X \sim X \cup Y$
	\end{enumerate}
	where $X \sim Y$ means $X \succeq Y$ and $Y \succeq X$.
\end{proposition}

Since $\hat x \sim \sigma(X)$ we also have that $\hat x \sim X$.
This means that for the menu preference any optimal alternative is indifferent to the whole choice set.

\begin{corollary}{}{}
	\begin{equation}
		X \succeq Y \iff \sigma(X) \succeq \sigma(Y)
	\end{equation}
\end{corollary}

Assuming $\succsim$ has an utility function $u$ we have that
\begin{equation}
	\sigma(X) = \argmax_{x \in X} u(x)
\end{equation}
and
\begin{equation}
	X \succeq Y \iff \max_{x \in X} (x) \geq \max_{y \in Y} u(y)
\end{equation}

\begin{definition}{Value function}{value-function}
	The value function $v: \mathcal D \to \R$ is defined as
	\begin{equation}
		v(X) = \max_{x \in X} u(x)
	\end{equation}
\end{definition}

This means that the value function represents the menu preference:
\begin{equation}
	X \succsim Y \iff v(x) \geq v(Y)
\end{equation}

\subsection{Contextualized analysis}

A contextualized alternative is a pair $(x, X)$ with $X \in \mathcal X$ and $x \in X$.
We call $\mathcal C$ the set of all contextualized alternatives.

A preference relation on $\mathcal C$ looks like
\begin{equation}
	(x, X) \succsim (y, Y)
\end{equation}
which means the DM prefers $x$ in the context $X$ over $y$ in the context $Y$.

A contextualized preference over $\mathcal C$ implies that $\forall X \in \mathcal X$ and $x, x' \in X$ it holds
\begin{equation}
	(x, X) \succsim (x', X) \iff x \succsim_X x'
\end{equation}

If $\succsim$ is a weak order we can define the optimal in the case of contextualized alternatives as in \Cref{def:rational-choice-correspondence}.

Moreover, the menu preference is induced by a contextualized universal preference:
\begin{equation}
	X \succeq Y \iff \forall y \in Y, \exists x \in X \text{ s.t. } (x, X) \succsim (y, Y)
\end{equation}
forall $X, Y \in \mathcal D$.

\subsection{Parametric analysis}

Choice sets are usually parametrized by elements of some set $\Theta$ (e.g. for budgets sets we have parameters of the form $(p, w) \in \R^2_+$).
Parametrization is carried out using a \emph{menu correspondence}:
\begin{equation}
	\varphi: \Theta \rightrightarrows \vec X
\end{equation}
thus
\begin{equation}
	\mathcal X = \{ \varphi(\theta) : \theta \in \Theta \} \quad \text{and} \quad \mathcal D = \{ \varphi(\theta) : \theta \in D \}
\end{equation}

The contextualized universal preference $\succsim$ is a binary relation over pairs
\begin{equation}
	(x, \theta) \in \Gr \varphi
\end{equation}

Some preferences $\succsim$ can also admit a parametric utility function of the form $u : \Gr \varphi \to \R$ such that, for all pairs $(x, \theta) \in \Gr \phi$
\begin{equation}
	(x, \theta) \succsim (x', \theta') \iff u(x, \theta) \geq u(x', \theta')
\end{equation}
and, for a fixed $\theta \in \Theta$ we have
\begin{equation}
	x \succsim_\theta x' \iff u(x,\theta) \geq u(x', \theta') \quad \forall x,x' \in \varphi(\theta)
\end{equation}

We can then generalize the optimization problem to take in account a parameter $\theta$:
\begin{equation}
	\max_x u(x) \quad {\operatorfont sub} \, x \in \varphi (\theta)
\end{equation}

Moreover we extend $\sigma$ and $v$ as follows:
$\sigma : D \rightrightarrows \vec X$ and $v: D \to \R$
\begin{gather}
	\sigma(\theta) = \sigma(\varphi(\theta)) = \{ \hat x \in \varphi(\theta) : \forall x \in \varphi(\theta), u(\hat x, \theta) \geq u(x, \theta)\} \\
	v(\theta) = v(\varphi(\theta)) = \max_{x \in \varphi (\theta)} u(x, \theta)
\end{gather}

\begin{theorem}{Maximum theorem}{maximum-theorem}
	Let $\vec X$ and $\Theta$ be metrizable (?).
	If the parametric utility function $u: \Gr \varphi \to \R$ and the menu correspondence $\varphi: \Theta \rightrightarrows \vec X$ are both continuous then $D = \Theta$ and
	\begin{enumerate}[label=\roman*.]
		\item the rational choice correspondence $\sigma \Theta \rightrightarrows \vec X$ is compact-valued and upper continuous when it is a function;
		\item the value function $v: \Theta \to \R$ is continuous
	\end{enumerate}
\end{theorem}

\section{Certainty}

First we define the action space $A$, the consequence space $C$, a collection $\mathcal A \in \mathcal P(A)$ and a consequence function $\rho: A \to C$.
Summing up we have $(A, \mathcal a, C, \rho)$.

DMs can have a preference over actions $\succsim$
\begin{equation}
	\rho(a) = \rho(b) \implies a \sim b
\end{equation}
but can also have a preference over consequences $\dot \succsim$.

\begin{theorem}{Outcome consequentialism}{outcome-consequentialism}
	There is a bijective mapping
	\begin{equation}
		\dot \succsim \enspace \mapsto \enspace \succsim
	\end{equation}
	if both are preorders.
\end{theorem}

\section{Decision under uncertainty}

\emph{This is macchero's part.}

TODO: missing class of 03/10

\subsection{Axioms}

We have $\bm C$ the set of material consequences, $\bm L$ the set of function $\ell: \bm C \to \R^+$ which are discrete probability mass functions on $\bm C$.

On $\bm L$ we can define preferences.
As usually we will define some axioms on preference:
\begin{enumerate}
	\item \emph{Weak order}: completeness and transitivity
	\item \emph{Independence}: for any three lotteries $l, l', l''$ in $\bm L$ and $0<p<1$ we have
	      \begin{equation}
		      l \succ l' \implies pl + (1-p) l'' \succ pl' + (1- p)l''
	      \end{equation}

	\item \emph{Archimedean}: for any three lotteries $l, l', l''$ in $\bm L$ there exists $q, p \in (0, 1)$ such that
	      \begin{equation}
		      pl + (1-p)l'' \succ l' \succ ql + (1-q)l''
	      \end{equation}
\end{enumerate}

Now assume that beyond the DM's preference, there exists a meaningful order to the space $\bm C$.
Then we can define also \emph{monotonicity} in the same way as we did for normal preferences.

\subsection{Utility functions}

\begin{theorem}{von Neumann-Morgentern Representation Theorem}{representation-theorem}
	Let $\succsim$ be a preference on $\bm L$ on the space $\bm C$.
	The following are equivalent:
	\begin{enumerate}
		\item $\succsim$ satisfies the axioms described above
		\item There exists a function $u:\bm C \to \R$ such that the function $\ddot u : \bm L \to \R$ defined as
		      \begin{equation}
			      \ddot u(l) = \sum^n_{i = 1} u(c_i) p_i
		      \end{equation}
		      or equivalently
		      \begin{equation}
			      U(l) = \sum_{c \in C} u(c) l(c)
		      \end{equation}
		      which represents $\succsim$.
		\item There exists $f:\bm L \to \R$ affine such that
		      \begin{equation}
			      l \succsim l' \iff f(l) \geq f(l')
		      \end{equation}
	\end{enumerate}
\end{theorem}

\begin{lemma}{}{}
	A function $f:\bm L \to \R$ is affine iff $\exists \varphi : C \to \R$ unique s.t. $f(l) = \sum_{c \in C} \varphi(c) l(c)$, where $\varphi = f \circ \delta$.

	This means that $f$ is affine only if it is an expectation of some $l$.
\end{lemma}
\begin{proof}[Proof of lemma]
	The macchero did it but I didn't write it :D
\end{proof}

\begin{proof}[Proof of \cref{thm:representation-theorem}]
	We will just proof point $2 \iff 3$.

	Recall that a function is affine if $f(\lambda k_1 + (1- \lambda) k_2) = \lambda f (k_1) + (1 - \lambda ) f(k_2)$ for $\lambda \in (0, 1)$.
	Note that affinity holds even for multiple $\lambda$ such that $\sum \lambda = 1$.

	Given the lemma above we can easily prove $3 \implies 2$ by choosing $\varphi = u$ and $f = U$.
	$2 \implies 3$ \say{grazie al cazzo} by taking $f = U$.
\end{proof}

We now want to show that $u$ is an utility of consumption.
To do so we will assume that $c \succsim c' \iff \delta_c \succsim \delta_{c'}$ but this means that $U(\delta_c) \geq U(\delta_{c'})$.
From here we can apply \cref{thm:representation-theorem} and get that $c \succsim c' \iff u(c) \geq u(c')$.

\subsubsection{Robustness}

\emph{Not in the exam?}

This is not a perfect representation of reality, usually we are not sure of the probability, we can \emph{estimate} the probability, but we cannot be sure.

\begin{equation}
	U_D(l) = \min_{m \in L} \left( U(m) + D(m \mid l) \right)
\end{equation}
where $l$ is our estimate, $m$ are other models and $D$ is a distance function.

\subsubsection{Cardinality}

The vN-M utility function $u$ on prizes is cardinal.
This means that if there exists two distinct $u$ such that
\begin{equation}
	l \succsim m \iff \sum u(c) l(c) \geq \sum u(c) l(c)
\end{equation}
then $\exists \alpha > 0, \beta \in \R$ s.t. $u = \alpha v + \beta$.

We have that the order of the differences over cardinal transformations is preserved:
\begin{equation}
	u(c_1) - u(c_2) \geq u(c'_1) - u(c'_2)
\end{equation}
will still hold under cardinal transformations.
This is equivalent to
\begin{equation}
	\left\{ (c_1, 0.5),  (c_2, 0.5) \right\} \succsim \left\{ (c'_1, 0.5),  (c'_2, 0.5) \right\}
\end{equation}

\subsection{Probability??}

Here the macchero went on a tangent explaining some basic stuff about discrete probability measures, their support, and expectation.
Idk, it seems like this is not in the lecture notes.

We can define lotteries as the restriction of a simple probability measure $p$ on $\bm C$.

We define $P_l = \sum_{c \in A} l(c)$ to be the probability of getting a prize $c \in A$ given a lottery $l$, with $A \subseteq \bm C$.

\begin{proposition}{}{}
	$P$ is a simple probability measure with finite carrier $A$ ($P(A)=1$, $A$ finite) iff $\exists l \in L$ s.t.
	\begin{equation}
		P(A) = \sum_{c \in A} l(c)
	\end{equation}
	Moreover, $l$ is unique and $l(c) = P(\{c\})$.
\end{proposition}

\end{document}
