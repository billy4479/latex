\documentclass[14pt]{extarticle}
\title{Analysis 2 Notes}
\author{Giacomo Ellero}
\date{Semester 2, 2023/2024}

\usepackage{amsfonts}
\usepackage{amsthm}
\usepackage{amssymb}
\usepackage{amsmath}
\usepackage{mathtools}
\usepackage{commath}
\usepackage{dirtytalk}
\usepackage{parskip}
\usepackage{mathrsfs}
\usepackage[many]{tcolorbox}
\usepackage{xparse}
\usepackage[a4paper,margin=1.5cm]{geometry}
\usepackage{bookmark}
\usepackage{physics}
\usepackage{tikz}

\newcommand{\C}{\mathbb{C}}
\newcommand{\R}{\mathbb{R}}
\newcommand{\F}{\mathcal{F}}
\renewcommand{\Re}{\operatorname{Re}}
\renewcommand{\Im}{\operatorname{Im}}

\newenvironment{absolutelynopagebreak}
  {\par\nobreak\vfil\penalty0\vfilneg
   \vtop\bgroup}
  {\par\xdef\tpd{\the\prevdepth}\egroup
   \prevdepth=\tpd}

\newtcolorbox{examplebox}[1]{colback=green!5!white,colframe=green!40!black,title={#1},fonttitle=\bfseries,parbox=false}
\newtcolorbox{notebox}[1]{colback=blue!5!white,colframe=blue!40!black,title={Note: #1},fonttitle=\bfseries,parbox=false}
\newtcolorbox{bluebox}[1]{colback=blue!5!white,colframe=blue!40!black,title={#1},fonttitle=\bfseries,parbox=false}
\newtcolorbox{warningbox}[1]{colback=orange!5!white,colframe=orange!90!black,title={Warning: #1},fonttitle=\bfseries,parbox=false}
   
\begin{document}

\maketitle
\tableofcontents
\clearpage

\section{Class of 12/02/2024}

In this class we will discuss mainly multivariable calculus:
\begin{itemize}
    \item Parametric curves: $\R \to \R^d$.
    \item Graphs: $\R^2 \to \R$.
    \item Vector fields: $\R^d \to \R^d$.
\end{itemize}

And in general we will discuss functions $f: \R^d \to \R^p$.

$\R^d$ is a vector space, like the ones we have seen in linear algebra, so many notions carry over.

\subsection{Dot product and distance}

\underline{Definition}: Let $x, y \in \R^d$. The dot product of $x$ and $y$ is defined as

$$
    \underline{x} \cdot \underline{y} = \sum_{i=1}^d x_i y_i
$$

We have seen dot products (inner products) in linear algebra but we will see some of their proprieties again:
\begin{enumerate}
    \item $\underline{x} \cdot \underline{y} = \underline{y} \cdot \underline{x}$.
    \item $\underline{x} \cdot (\underline{y} + \underline{z}) = \underline{x} \cdot \underline{y} + \underline{x} \cdot \underline{z}$.
    \item $\underline{x} \cdot \underline{x} \geq 0$ and $\underline{x} \cdot \underline{x} = 0 \iff \underline{x} = \underline{0}$.
\end{enumerate}

% TODO: add previous classes

\section{Class of 16/02/2024}

\subsection{Parametrizing ellipses}

Last class we saw how to parametrize lines and circles, now we will have a look at ellipses.

An ellipse is defined by the parametrization

\[
    \begin{cases}
        x = a \cos t \\
        y = b \sin t
    \end{cases}
\]

where $a$ and $b$ are the semi-axes of the ellipse. Note that if $a = b$ we get a circle of radius $a$.

For the equation we have

$$
    \frac{x^2}{a^2} + \frac{y^2}{b^2} = 1
$$

The ellipse is the image of the unit circle by tge linear transformation $\begin{pmatrix}
        x \\ y
    \end{pmatrix} \to \begin{pmatrix}
        ax \\ by
    \end{pmatrix}$.

\subsubsection{Applying transformations to curves}

Consider curve $\gamma_1$ which we want to rotate by an angle $\theta$ to obtain $\gamma_2$.

Remember the rotation matrix

$$
    R(\theta) = \begin{pmatrix}
        \cos \theta & -\sin \theta \\
        \sin \theta & \cos \theta
    \end{pmatrix}
$$

For the parametrization is quite easy since we just need to multiply the vector by the rotation matrix.

$$
    \begin{pmatrix}
        x \\ y
    \end{pmatrix} = \begin{pmatrix}
        \cos \theta & -\sin \theta \\
        \sin \theta & \cos \theta
    \end{pmatrix} \begin{pmatrix}
        f(t) \\ g(t)
    \end{pmatrix}
$$

Getting the equation is a bit trickier.
We have to invert the transformation matrix:
consider a point $(x, y)$ on $\gamma_2$, we have that $(x, y) = R(\theta) (x', y')$, hence $(x', y') \in \gamma_1 = R(-\theta) (x, y)$.
We have that $(x,y) \in \gamma_2 \iff R(-\theta)(x, y) \in \gamma_1$.
Now we have $(x', y')$ in terms of $(x, y)$ and we can plug the result of the reverse transformation into the equation of $\gamma_1$.

\subsection{Cylinders}

Let $(x, y, z) \in \R^3$. We have that $\sqrt{x^2 + y^2 + z^2}$ is the distance from the origin and $\sqrt{x^2 + y^2}$ is the distance to the $Oz$ axis.

The equation of a cylinder is

$$
    \left\{ (x, y, z) \in \R^3  : x^2 + y^2 = r^2\right\}
$$

we are basically saying that the distance from the $Oz$ axis is constant and is equal to $r$.

\subsection{Cones}

The way we describe cones is by saying that if we slice the cone at altitude $z$ we will have a circle of radius $z$ and center at $(0,0,z)$.

So we want $\sqrt{x^2 + y^2} = |z|$, hence

$$
    \left\{ (x, y, z) \in \R^3 : x^2 + y^2 = z^2 \right\}
$$

\subsubsection{Variations of the cone}

We can consider the following variations

$$
    x^2 + y^2 = z^2 + 1 \\
    x^2 + y^2 = z^2 - 1
$$

In the first one we have that at $z=0$ we have a circle of radius 1, in the second one we have that at $z=0$ we have a circle of radius $-1$ which is not possible, hence the second equation only gets to $z = \pm 1$.

These type of surfaces are called hyperboloids of revolution.

% TODO: Add the drawings

\subsection{Parametric curves}

We can now give a formal definition of a parametric curve.

\underline{Definition}: A parametric curve $\gamma$ is a function of $t \in I \subseteq \R$ and valued in $\R^d$.

$$
    \gamma(t) = \begin{pmatrix}
        \gamma_1(t) \\
        \vdots      \\
        \gamma_d(t)
    \end{pmatrix} \in \R^d
$$

where $\gamma_1, \ldots, \gamma_d$ are the coordinate functions defined as $\gamma_i: I \to \R$.

Parametric curves are basically the description of the motion of a point in space as a function of time.

\underline{Definition}: The \textbf{domain of definition} of $\gamma$ is the intersection of the domains of the coordinate functions.

\subsubsection{How to represent a curve}

There are many ways to represent a curve, for example we could graph the individual coordinate functions but this is not very representative of the curve itself.

We can try to use a graph to represent a curve $\gamma$:

\underline{Definition}: The \textbf{graph} of $\gamma$ is the subset of $\R^{d+1}$ made of points
$$
    (t, \gamma(t)) = (t, \gamma_1(t), \ldots, \gamma_d(t))
$$
for $t \in I$.

The graphs contains all the information about the curve, but it is not very practical to use because of its complexity.

\underline{Definition}: The \textbf{image} of $\gamma$ is the subset of $\R^d$ made of points $\gamma(t)$ for $t \in I$.

The image is basically the graphs seen from the $x$ axis.

This representation is quite practical but we are losing some information about the parameter $t$, this is mostly ok though.

Note that different curves will always have different graphs but might have the same image.

\underline{Remark}: Every graph is the image of some function in $\R^{d+1}$, but not all images are graphs of some function in $\R^{d-1}$.

\end{document}