\documentclass[10pt]{extarticle}

\title{Physics 2 Formulas}
\author{Giacomo Ellero}
\date{a.y. 2024/2025}

\usepackage[arrowdel]{physics}
\usepackage{siunitx}
\usepackage{preamble_base}
\usepackage{preamble_math}
\usepackage{multicol}

% \renewcommand{\vec}[1]{\uvec{#1}}
\setlength{\columnsep}{1cm}
\setlength{\columnseprule}{1pt}
\def\columnseprulecolor{\color{blue}}

\numberwithin{equation}{section}

\begin{document}

\section{Change of coordinates}
\begin{multicols}{2}

    \subsection{Spherical}
    \begin{equation}
        \begin{cases}
            x = r \sin \theta \cos \varphi & r \in [0, \infty)     \\
            y = r \sin \theta \sin \varphi & \theta \in [0, \pi]   \\
            z = r \cos \theta              & \varphi \in [0, 2\pi]
        \end{cases}
    \end{equation}

    Jacobian:
    \begin{equation}
        \dd{\tau} = r^2 \sin \theta \dd{r} \dd{\theta} \dd{\varphi}
    \end{equation}

    \subsection{Cylindrical}
    \begin{equation}
        \begin{cases}
            x = s \cos \varphi & s \in [0, \infty)       \\
            y = s \sin \varphi & \varphi \in [0, \pi]    \\
            z = z              & z \in (-\infty, \infty)
        \end{cases}
    \end{equation}

    Jacobian:
    \begin{equation}
        \dd{\tau} = s \dd{s} \dd{\varphi} \dd{z}
    \end{equation}
\end{multicols}

\section{Electrostatics}
\begin{multicols}{2}
    \subsection{Electric Field}

    \textbf{Coulomb law}
    \begin{equation}
        \vec F = \oneover{4\pi \varepsilon_0} \frac{Q_1 Q_2}{\Delta r^2} \hat{\Delta r}
    \end{equation}

    \textbf{Electric field}
    \begin{equation}
        \vec E =
        \oneover{4 \pi \varepsilon_0}
        \int_{R^3} \frac{\rho (\vec{r'})}{\Delta r} \hat{\Delta r} \dd{\tau'}
    \end{equation}
    where
    \begin{itemize}
        \item $\vec r$ is the position of the point where we want to compute the field
        \item $\vec{r'}$ is the position of the space we are integrating over
        \item $\Delta r = \vec r - \vec{r'}$
        \item $\dd \tau'$ is the portion of space at $\vec{r'}$
    \end{itemize}

    \textbf{Gauss law}
    \begin{equation}
        \oint_S \vec E \cdot \vec{\dd S} = \frac{Q_\text{enc}}{\varepsilon_0}
    \end{equation}

    \textbf{Maxwell equation (Gauss in local form)}
    \begin{equation}
        \div \vec E = \frac{\rho}{\varepsilon_0}
    \end{equation}

    \subsection{Potential}

    \textbf{Definition}
    \begin{equation}
        \vec E = - \grad V \implies \int_C \vec E \cdot \vec{\dd \ell} = V(A) - V(B)
    \end{equation}
    assume that $V(\infty) = 0$.

    \textbf{Point charge}
    \begin{equation}
        V(\vec r) = \oneover{4 \pi \varepsilon_0} \frac{q}{\Delta r}
    \end{equation}

    \textbf{Continuous charge distribution}
    \begin{equation}
        V(\vec{r}) = \oneover{4 \pi \varepsilon_0} \int_{R^3} \frac{\rho(\vec{r'})}{\Delta r} \dd{\tau'}
    \end{equation}

    \textbf{Potential}
    \begin{equation}
        \oneover{4 \pi \varepsilon_0} \frac{\vec p \cdot \hat r}{r^2}
    \end{equation}

    \subsection{Energy}

    \textbf{Potential energy}
    \begin{equation}
        U(\vec r) = q V(\vec r)
    \end{equation}

    \textbf{Energy stored}
    \begin{equation}
        W = \half \int_{\mathcal V} \rho V \dd \tau = \half \varepsilon_0 \int_{\R^3} E^2 \dd \tau
    \end{equation}

    \subsection{Dipoles}

    \textbf{Ideal dipole}
    \begin{equation}
        r \gg d
    \end{equation}

    \textbf{Dipole moment}
    \begin{equation}
        \vec p = q \vec d
    \end{equation}

    \textbf{Torque}
    \begin{equation}
        \vec N = \vec p \cross \vec E
    \end{equation}

    \textbf{Force}
    \begin{equation}
        \vec F = (\vec p \cdot \grad) \vec E
    \end{equation}

    \textbf{Energy}
    \begin{equation}
        U = - \vec p \cdot \vec E
    \end{equation}

    \subsection{Conductors}

    \textbf{Properties}
    \begin{enumerate}
        \item $\vec E_\text{inside} = 0$
        \item The induced charged are located on the surface of the conductor.
        \item $\vec E^\text{just outside} = \frac{\sigma}{\varepsilon_0} \hat n$.
        \item $V_\text{inside} = \text{const} \implies$ conductors are \emph{equipotential}.
    \end{enumerate}

    \subsubsection{Capacitors}

    \textbf{Capacitance}
    \begin{equation}
        C = \frac{Q }{\Delta V} = \varepsilon_0 \frac{A}{d}
    \end{equation}

    \textbf{Parallel} (same potential)
    \begin{equation}
        C = C_1 + C_2
    \end{equation}

    \textbf{Series} (same charge)
    \begin{equation}
        C = \frac{C_1 C_2}{C_1 + C_2}
    \end{equation}

    \textbf{Energy}
    \begin{equation}
        W = \half Q \Delta V
    \end{equation}

    \textbf{Force}
    \begin{equation}
        F = \frac{Q}{2 \varepsilon_0 A}
    \end{equation}

    \textbf{Pressure}
    \begin{equation}
        P = \frac{Q}{2 \varepsilon_0 A^2} = u
    \end{equation}

    \subsection{Insulators}

    \textbf{Bound densities}
    \begin{equation}
        \begin{cases}
            \rho_b = -\div \vec P \\
            \sigma_b = \vec P \cdot \hat n
        \end{cases}
    \end{equation}

    \textbf{Electric displacement}
    \begin{gather}
        \vec D = \varepsilon_0 \varepsilon_r \vec E \\
        \div D = \rho_f \\
        \oint_S \vec D \cdot \vec{\dd S} = Q_\text{enc}^f
    \end{gather}

    \textbf{Polarization}
    \begin{equation}
        \vec P = \varepsilon_0 (\varepsilon_r-1) \vec E
    \end{equation}

\end{multicols}

\section{Electrodynamics}
\begin{multicols}{2}
    \textbf{Current}
    \begin{equation}
        I = \dv{q}{t} = \int_\mathcal{S} \vec J \cdot \vec{\dd S}
    \end{equation}

    \textbf{Current density}
    \begin{equation}
        \vec J = q n \vec v_d
    \end{equation}

    \textbf{Ohm's law}
    \begin{equation}
        \Delta V = RI = \frac{1}{\sigma} \frac{\ell}{S} I
    \end{equation}

    \textbf{Joule's law}
    \begin{equation}
        P = R I ^2 = \Delta V I = \frac{\Delta V^2}{R}
    \end{equation}

    \textbf{Discharge of a capacitor}
    \begin{equation}
        Q(t) = Q_0 \exp(-\frac{t}{RC})
    \end{equation}

    \subsection{Kirchoff's laws}
    \textbf{Current in a node}
    \begin{equation}
        \sum_{k = 1}{n} I_k = 0
    \end{equation}

    \textbf{Potential difference in loop}
    \begin{equation}
        \sum_{k = 1}{n} \Delta V_k = 0
    \end{equation}

    \subsection{Resistors}

    \textbf{Series}
    \begin{equation}
        R_\text{eq} = R_1 + R_2
    \end{equation}

    \textbf{Parallel}
    \begin{equation}
        \oneover{R_\text{eq}} = \oneover {R_1} + \oneover {R_2}
    \end{equation}
\end{multicols}

\section{Magnetostatic}
\begin{multicols}{2}
    \textbf{Lorentz force}
    \begin{equation}
        \vec F = q \vec v \cross \vec B
    \end{equation}

    \textbf{Force on a 1D wire}
    \begin{equation}
        \vec F = I \int \vec{\dd l} \cross \vec B
    \end{equation}

    \subsection{Magnetic field}

    \textbf{Biot-Savart Law}
    \begin{equation}
        \vec B = \frac{\mu_0 I}{4 \pi} \oint_\text{circuit} \frac{\vec{\dd l}' \cross \hat{\Delta r}}{\Delta r^2}
    \end{equation}

    \textbf{Magnetic field of a wire}
    \begin{equation}
        \vec B = \frac{\mu_0 I}{2 \pi s} \hat \varphi
    \end{equation}

    \textbf{Magnetic field of a solenoid}
    \begin{equation}
        \vec B = \mu_0 n I \hat z
    \end{equation}

    \subsubsection{Ampere's law}
    \textbf{Local form}
    \begin{equation}
        \curl \vec B = \mu_0 \vec J
    \end{equation}

    \textbf{Integral form}
    \begin{equation}
        \oint_\mathcal{C} \vec B \cdot \vec{\dd l} = \mu_0 I_\text{enc}
    \end{equation}

    \subsection{Magnetic field in matter}

    \textbf{Magnetic dipole}
    \begin{equation}
        \vec m = I \vec S
    \end{equation}

    \textbf{Magnetization}
    \begin{equation}
        \vec M = \dv{\vec m}{\tau} = \chi_m \vec H
    \end{equation}

    \textbf{Bound currents}
    \begin{equation}
        \begin{cases}
            \vec J_b = \curl \vec M \\
            \vec K_b = \vec M \cross \hat n
        \end{cases}
    \end{equation}

    \textbf{Free magnetic field}
    \begin{equation}
        \vec H = \frac{\vec B}{\mu_0} - \vec M
    \end{equation}

    \textbf{Ampere law in matter}
    \begin{equation}
        \oint_\mathcal{C} \vec H \cdot \vec{\dd l } = I^f_{\text{enc}}
    \end{equation}

    \textbf{Total magnetic field in matter}
    \begin{equation}
        \vec B = \mu_0(\vec H + \vec M) = \mu_0 (1 + \chi_m) \vec H = \mu_0 \mu_r \vec H
    \end{equation}

    \subsubsection{Ferromagnets}
    $\vec B_\text{ind}$ in the \emph{same} direction of $\vec B_f$, maintained.
    Non-linear but can approximated with $\chi_m \gg 1$.

    \subsubsection{Paramagnets}
    $\vec B_\text{ind}$ in the \emph{same} direction of $\vec B_f$, not maintained.
    \begin{align}
        \vec F & = \grad(\vec m \cdot \vec B) \\
        \vec N & = \vec m \cross \vec B
    \end{align}
    Attracted by magnets.

    \subsubsection{Diamagnets}
    $\vec B_\text{ind}$ in the \emph{opposite} direction of $\vec B_f$, not maintained.
    Repelled from the magnets.
\end{multicols}

\section{Maxwell's Equations}
\begin{multicols}{2}
    \subsection{Faraday's law}

    \textbf{Integral form}
    \begin{equation}
        \mathcal E = - \dv{\Phi_\mathcal{S}(\vec B)}{t} = v B h
    \end{equation}

    \textbf{Local form}
    \begin{equation}
        \curl \vec E = - \pdv{\vec B}{t}
    \end{equation}

    \subsection{Inductance}
    \textbf{Mutual inductance}
    \begin{equation}
        M = \frac{\mu_0}{4 \pi} \oint \oint \frac{\vec{\dd l_1} \cdot \vec{\dd l_2}}{\Delta r}
    \end{equation}

    \textbf{Self inductance}
    \begin{equation}
        \Phi(\vec B) = LI
    \end{equation}

    \textbf{Back EMF}
    \begin{equation}
        \mathcal E = - L \dv{I}{t}
    \end{equation}

    \textbf{Energy stored in magnetic field}
    \begin{equation}
        W = \half L I ^2 = \frac{1}{2 \mu_0} \int_{\R^3} B^2 \dd \tau
    \end{equation}

    \subsection{Maxwell's equations}

    \textbf{Maxwell's equations}
    \begin{equation}
        \begin{cases}
            \div \vec E = \frac{\rho}{\varepsilon_0}                          \\
            \div \vec B = 0                                                   \\
            \curl \vec E = -\pdv{\vec B}{t}                                   \\
            \curl \vec B = \mu_0 \vec J + \mu_0 \varepsilon_0 \pdv{\vec E}{t} \\
        \end{cases}
    \end{equation}

    \textbf{Lorentz force}
    \begin{equation}
        F = q(\vec E + \vec v \cross \vec B)
    \end{equation}
\end{multicols}

\end{document}

