\documentclass[8pt]{extarticle}

% \usepackage{preamble_base}
% \usepackage{preamble_math}
\usepackage{preamble_code}

\author{Giacomo Ellero}

\usepackage{parskip} % for no indentation
\usepackage[a4paper,margin=0.5cm]{geometry} % to adjust margins
\usepackage{bookmark} % for pdf bookmarks

\usepackage{float} % for H in figures
\usepackage{graphicx} % for images
\usepackage{subfig} % for subfigures

\usepackage{dirtytalk} % for quotes
% \usepackage[many]{tcolorbox} % for boxes
% \usepackage{tikz} % for drawings
\usepackage{enumitem} % for custom lists
\usepackage{hyperref} % for hyperlinks

\usepackage{fontspec}
\setsansfont{Ubuntu}
\setmonofont{FiraCode Nerd Font}
% Use sans-serif font
\renewcommand{\familydefault}{\sfdefault}

\hypersetup{
    colorlinks=true,
    linkcolor=red,
    urlcolor=blue,
}
\urlstyle{same}

\newenvironment{absolutelynopagebreak}
{\par\nobreak\vfil\penalty0\vfilneg
    \vtop\bgroup}
{\par\xdef\tpd{\the\prevdepth}\egroup
    \prevdepth=\tpd}

% \newtcolorbox{examplebox}[1]{colback=green!5!white,colframe=green!40!black,title={#1},fonttitle=\bfseries,parbox=false}
% \newtcolorbox{notebox}[1]{colback=blue!5!white,colframe=blue!40!black,title={Note: #1},fonttitle=\bfseries,parbox=false}
% \newtcolorbox{bluebox}[1]{colback=blue!5!white,colframe=blue!40!black,title={#1},fonttitle=\bfseries,parbox=false}
% \newtcolorbox{warningbox}[1]{colback=orange!5!white,colframe=orange!90!black,title={Warning: #1},fonttitle=\bfseries,parbox=false}

\setcounter{secnumdepth}{3}

\newcommand{\firstpage}{
    \maketitle
    \phantomsection
    \hypertarget{toc}{}
    \tableofcontents
    \clearpage
}

\usepackage{multicol}
% Math stuff
\usepackage{amsfonts}
\usepackage{amsthm}
\usepackage{amssymb}
\usepackage{amsmath}
\usepackage{mathtools}
\usepackage{commath}
\usepackage{mathrsfs}
\usepackage{bm} % for bold math symbols
\usepackage{physics} % for derivatives
\usepackage{cancel} % for canceling terms

\newcommand{\gt}{>}
\newcommand{\lt}{<}
\newcommand{\C}{\mathbb{C}}
\newcommand{\R}{\mathbb{R}}
\newcommand{\N}{\mathbb{N}}
\newcommand{\Q}{\mathbb{Q}}
\newcommand{\Z}{\mathbb{Z}}
\newcommand{\indep}{\perp \!\!\! \perp}

\newcommand{\skiplineafterproof}{$ $\par\nobreak\ignorespaces}

\renewcommand{\Re}{\operatorname{Re}}
\renewcommand{\Im}{\operatorname{Im}}

\renewcommand{\O}{\mathcal{O}}
% https://tex.stackexchange.com/questions/191059/how-to-get-a-small-letter-version-of-mathcalo
\renewcommand{\o}{
    \mathchoice
    {{\scriptstyle\mathcal{O}}}% \displaystyle
    {{\scriptstyle\mathcal{O}}}% \textstyle
    {{\scriptscriptstyle\mathcal{O}}}% \scriptstyle
    {\scalebox{.6}{$\scriptscriptstyle\mathcal{O}$}}%\scriptscriptstyle
}

% Sorcery taken from https://tex.stackexchange.com/a/163284
\makeatletter
\def\munderbar#1{\underline{\sbox\tw@{$#1$}\dp\tw@\z@\box\tw@}}
\makeatother

\newcommand{\uvec}[1]{\munderbar{\bm{#1}}}

\newtheorem{theorem}{Theorem}[section]
\newtheorem{corollary}{Corollary}[theorem]
\newtheorem{lemma}[theorem]{Lemma}
\newtheorem{proposition}[theorem]{Proposition}

\theoremstyle{definition}
\newtheorem{definition}{Definition}[section]
\newtheorem{example}{Example}[section]

\theoremstyle{remark}
\newtheorem*{remark}{Remark}

\numberwithin{equation}{section}

% autoref stuff
\def\definitionautorefname{Definition}
\def\propositionautorefname{Proposition}

\setlength{\columnseprule}{0.5pt}
\def\columnseprulecolor{\color{blue}}
\setlength{\columnsep}{0.24cm}

\title{Computer Science 2 CheatSheet, Partial 2}
\date{Semester 2, 2023/2024}

\newcommand{\NP}{{\mathcal{NP}}}
\newcommand{\NPC}{$\NP$-Complete}
\newcommand{\reducesto}{\leq_p}

\renewcommand{\vec}[1]{\uvec{#1}}

\usepackage{pdflscape}

\begin{document}

\numberwithin{table}{section}
\numberwithin{figure}{section}

\begin{landscape}


    \begin{multicols}{3}
        \section{Flows}

        \begin{definition}[$\delta^+(u)$ and $\delta^-(u)$]
            Let $u \in V$.
            \begin{align}
                \delta^+(u) & = \{(w, v) \in E : w = u\} \\
                \delta^-(u) & = \{(w, v) \in E : v = u\}
            \end{align}
            $\delta^+(u)$ are the \emph{outgoing} edges and $\delta^-(u)$ are the \emph{incoming} edges.

            For cuts:
            \begin{align}
                \delta^+(U) & = \{(w, v) \in E : w \in U, v \notin U\} \\
                \delta^-(U) & = \{(w, v) \in E : w \notin U, v \in U\}
            \end{align}
        \end{definition}

        \begin{definition}[flow]
            \label{def:flow:flow}
            $f:E \to \R^+$ is an $s$-$t$ flow if:
            \begin{enumerate}[label=\roman*.]
                \item $f(a) \geq 0 \enspace \forall e \in E$
                \item $\sum_{e \in \delta^-(u)} f(e) = \sum_{e \in \delta^+(u)} f(e) \enspace \forall u \in V \setminus \{ s, t \}$: the flow that comes in goes out except for $s$ and $t$.
            \end{enumerate}
        \end{definition}

        \begin{definition}[value of the flow]
            The flow going out of $s$ or coming into $t$.
        \end{definition}

        \begin{definition}[$s$-$t$ cut]
            A partition of $V$ of the form $(U, V \setminus U)$ such that $s \in U$ and $t \notin U$.
        \end{definition}

        \begin{definition}[capacity of a cut]${\operatorfont cap}(U) = \sum_{e \in \delta^+(U)} c(e)$
        \end{definition}

        \begin{theorem}[Ford-Fulkerson's algorithm]
            \skiplineafterproof
            \begin{enumerate}
                \item Set $f(e) = 0 \enspace \forall e \in E$
                \item \label{itm:flow:alg2_iter} Construct an auxiliary graph $D_f = (V, E_f)$ where
                      \begin{itemize}
                          \item $e \in E_f$ if $c(e) - f(e) > 0$, that is, $e$ has some residual capacity;
                          \item $e^{-1} \in E_f$ if $f(e) > 0$.
                      \end{itemize}
                \item Find a directed $s$-$t$ path in $D_f$. If such path $P$ doesn't exists jump to \autoref{itm:flow:alg2_return}, otherwise
                      \begin{enumerate}[label*=\arabic*.]
                          \item Let $\{e_1, \ldots, e_k\}$ be the edges in $P$
                          \item Let $\varepsilon$ be the edge such that it gives the minimum between
                                \begin{itemize}
                                    \item the edge that minimizes $c(e_i) - f(e_i)$, that is, the edge with minimum residual capacity
                                    \item the flipped edge $e^{-1}_i$ that minimizes $f(e_i)$, that is, the edge that minimizes the amount we can push back
                                \end{itemize}
                          \item For all $i = 1, \ldots, k$, if $e_i \in P$
                                \begin{itemize}
                                    \item If $e_i \in P$ and $e_i \in E$ set $f(e_i) = f(e_i) + \varepsilon$
                                    \item If $e_i \in P$ and $e_i^{-1} \in E$ $f(e_i^{-1}) = f(e_i^{-1}) - \varepsilon$
                                \end{itemize}
                          \item Go back to \autoref{itm:flow:alg2_iter}
                      \end{enumerate}
                \item \label{itm:flow:alg2_return} Return $f$
            \end{enumerate}
        \end{theorem}

        \begin{remark}
            Code to update the graph in-place instead of reconstructing it from scratch:
            For each $e_i \in P$
            \begin{itemize}
                \item If $e_i \in E$ and $f(e_i) = c(e_i)$ then remove $e_i$ from $E_f$.
                      Moreover, if $e_i^{-1} \notin E_f$ add $e_i^{-1}$ to $E_f$
                \item Else, if $e^{-1}_i \in E$ and $f(e_i^{-1}) = 0$ then remove $e^{-1}_i$ from $E_f$.
                      Moreover if $e_i \notin E_f$ add $e_i$ to $E_f$.
            \end{itemize}
        \end{remark}

        \section{Linear programming}
        \begin{equation}
            \vec x = \begin{bmatrix}
                x_1    \\
                \vdots \\
                x_n
            \end{bmatrix}                        \quad
            \vec c = \begin{bmatrix}
                c_1    \\
                \vdots \\
                c_n
            \end{bmatrix}                        \quad
            A = \begin{bmatrix}
                a_{1,1} & \dots  & a_{1, n} \\
                \vdots  & \ddots & \vdots   \\
                a_{k,1} & \dots  & a_{k, n} \\
            \end{bmatrix} \quad
            \vec b = \begin{bmatrix}
                b_1    \\
                \vdots \\
                b_k
            \end{bmatrix}
        \end{equation}

        \begin{definition}[primal]
            \begin{equation}
                \max \vec c^T \vec x \quad \text{s.t.} \quad A \vec x \leq \vec b, \enspace \vec x \geq 0
            \end{equation}
        \end{definition}

        \begin{definition}[dual]
            \begin{equation}
                \min \vec y^T \vec b \quad \text{s.t.} \quad \vec y^T A \geq \vec c^T, \enspace \vec y \geq 0
            \end{equation}
        \end{definition}

        \begin{remark}
            If instead of $\leq$ we have $=$ just remove the constraint of $y_i \geq 0$.
        \end{remark}

        \begin{theorem}[strong duality]
            The dual has an optimal solution iff the primal does.
            \begin{equation}
                \vec c^T \vec x^* = \left(\vec y^*\right)^T \vec b
            \end{equation}
        \end{theorem}

        \begin{theorem}[complementrary slackness]
            If $\vec x$ and $\vec y$ are optimal solutions, then
            \begin{align}
                \vec y^T (A \vec x - \vec b)   & = 0 \\
                \vec x (\vec y^T A - \vec c^T) & = 0
            \end{align}
        \end{theorem}

        \begin{remark}[sensitivity]
            $\vec y$ has somewhat the same meaning of the derivative.
        \end{remark}

        \section{NP-Completeness}

        \begin{definition}[\NPC{} problem]
            A problem $X$ is \NPC{} if $X \in \NP$ and $\forall Y \in \text{\NPC}, Y \reducesto X$.
        \end{definition}

        \begin{remark}
            A problem $X$ is in $P$ if $X \reducesto Y \in P$.
        \end{remark}

        \subsection{VertexCover}
        \begin{definition}
            Given $G=(V, E)$ and $k \in \N^+$ such that $0 < k < \abs{V}$, decide if $C \subseteq V$ s.t. $\forall (u, v) \in E$ either $u \in C$ or $v \in C$ with $\abs{C} \leq k$ exists.
        \end{definition}

        \textbf{Reductions}:
        \begin{itemize}
            \item 3-SAT $\reducesto$ VertexCover: do the gadgets thingy
        \end{itemize}

        \subsection{IndependentSet}
        \begin{definition}
            Given $G=(V,E)$ and $k \in \N^+$ decide whether $\exists S \subseteq V$ such that $\abs{S} \geq k$ and $\forall (u, v) \in S$ we have $(u, v) \notin E$.
        \end{definition}

        \textbf{Reductions}:
        \begin{itemize}
            \item VertexCover $\reducesto$ IndependentSet and viceversa: $C$ is a VertexCover iff $S = V \setminus C$ is an IndependentSet.
        \end{itemize}

        \subsection{Clique}
        \begin{definition}
            Given $G=(V,E)$ and $k \in \N^+$ decide whether $\exists S \subseteq V$ such that $\abs{S} \geq k$ and $\forall (u, v) \in S$ we have $u \in S$ and $v \in S$.
        \end{definition}

        \textbf{Reductions}:
        \begin{itemize}
            \item IndependentSet $\reducesto$ Clique and viceversa: Construct a graph $\overline{G} = (V, \overline{E})$ containing all the vertices not in $E$. Then a set of nodes $S$ is an IndependentSet in $G$ iff $S$ is a Clique in $\overline{G}$.
        \end{itemize}


        \subsection{IntegerProgramming}
        \begin{definition}
            Given a set of linear inequalities $A \vec x \leq b$ is it possible to find $\vec X \in \N^n$ such that all the inequalities are satisfied?
        \end{definition}

        \textbf{Reductions}:
        \begin{itemize}
            \item VertexCover $\reducesto$ IntegerProgramming: Introduce $y_v$ $\forall v \in V$. Write the following integer program: $\min \sum{v \in V} y_v$; $y_v + y_u \geq 1$ $\forall (u, v) \in E$; $y_v \leq 1$ $\forall v \in V$; $y_v \geq 0$ $\forall v \in V$; with $y_v \in \N$.
        \end{itemize}


        \subsection{SubsetSum}
        \begin{definition}
            Given $A = \{a_1, \ldots, a_n\}$, $a_i \in \N$ and $k \in \N$, does there exist some $I \subseteq \{1, \ldots, n\}$ of indices of $A$ such that $\sum_{i \in I} a_i = k$?
        \end{definition}

        \textbf{Reductions}:
        \begin{itemize}
            \item IndependentSet $\reducesto$ SubsetSum: Given an instance of independent set define $a_v \forall v \in V$ and $b_{uv} \forall (u,v) \in E$. Construct a matrix that represents the integers: first column 1 if vertex 0 otherwise; one column for each edge $(u, v)$, 1 at rows $a_v, a_u, b_{uv}$, 0 otherwise; row representing $k' = 10^{\abs{E}}k + \sum^{\abs{E}-1}_{i = 0} 10^1$. The sum of the row is $k'$ iff the $a_v$s form an independent set.
        \end{itemize}


        \subsection{Knapsack}
        \begin{definition}
            Given $n$ objects with weights $w_1, \ldots, w_n$, a budget $B$, profits $p_1, \ldots, p_n$, and a parameter $k$, is it possible to find $I \subseteq {1, \ldots, n}$ such that $\sum_{i \in I} w_i \leq B$ and $\sum_{i \in I} p_i \geq k$?
        \end{definition}

        \textbf{Reductions}:
        \begin{itemize}
            \item SubsetSum $\reducesto$ Knapsack: Construct Knapsack with $w_i = p_i = a_i$ and $B = k$.
        \end{itemize}


        \subsection{SetCover}
        \begin{definition}
            Given a set $X$, $S_1, \dots, S_n$ s.t. $S_i \in \mathcal P(X)$, and $k \in \N$. We ask if a collection $\mathcal S$ of $S_i$s such that $\bigcup_{S \in \mathcal S} S = X$. Then a solution to this Knapsack is a solution to the SubsetSum problem.
        \end{definition}

        \textbf{Reductions}:
        \begin{itemize}
            \item VertexCover $\reducesto$ SetCover: Associate each edge to the elements of $X$ and each vertex to a subset $S_v$ containing all the elements associated to the edges in $\delta^+(S_v)$.
        \end{itemize}

        \subsection{Partition}
        \begin{definition}
            Given $A = \{a_1, \ldots, a_n\}$ with $a_i \in \N$ we ask whether it is possible to find a partition $\mathcal A$ such that $\sum_{a \in \mathcal A} a = \sum_{a \notin \mathcal A} a$.
        \end{definition}

        \textbf{Reductions}:
        \begin{itemize}
            \item SubsetSum $\reducesto$ Partition: Let $z_1 = \sum_{a \in A} a$ and $z_2 = 2k$, then the target sum of the partition is $\sum_{a \in A} a + k$.
                  If $S'$ is a solution so SubsetSum then $\sum_{a \in S'} a = k$ and $\sum_{a \in S' \cup \{z_1\}} = \sum_{a \in A} a + k$.
            \item Partition $\reducesto$ SubsetSum: Let $z_1, z_2$ as before. Then if $S''$ is a solution to Partition $\sum_{a \in S''} a = \sum_{a \in A} a + k$.
                  Since $S''$ is a partition it must contain either $z_1$ or $z_2$ but not both because $z_1 + z_2 > \sum_{a \in S''} a$.
                  Consider the partition hat contains $z_1$, then the $\sum_{S'' \setminus \{z_1\}} = k$.
        \end{itemize}

    \end{multicols}
\end{landscape}
\end{document}