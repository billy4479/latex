\documentclass[10pt]{extarticle}
\title{Physics 1 Notes}
\author{Giacomo Ellero}

\usepackage{amsfonts}
\usepackage{amsthm}
\usepackage{amssymb}
\usepackage{amsmath}
\usepackage{mathtools}
\usepackage{commath}
\usepackage{dirtytalk}
\usepackage{parskip}
\usepackage{mathrsfs}
\usepackage[many]{tcolorbox}
\usepackage{xparse}
\usepackage[a4paper,margin=1.5cm]{geometry}
\usepackage{bookmark}

\newcommand{\C}{\mathbb{C}}
\newcommand{\R}{\mathbb{R}}
\newcommand{\F}{\mathcal{F}}
\renewcommand{\Re}{\operatorname{Re}}
\renewcommand{\Im}{\operatorname{Im}}

\newenvironment{absolutelynopagebreak}
  {\par\nobreak\vfil\penalty0\vfilneg
   \vtop\bgroup}
  {\par\xdef\tpd{\the\prevdepth}\egroup
   \prevdepth=\tpd}

\newtcolorbox{examplebox}[1]{colback=green!5!white,colframe=green!40!black,title={#1},fonttitle=\bfseries,parbox=false}
\newtcolorbox{notebox}[1]{colback=green!5!white,colframe=blue!40!black,title={Note: #1},fonttitle=\bfseries,parbox=false}

\begin{document}

\maketitle
\tableofcontents
\clearpage

\section{Class of 06/02/2024 - Introduction}

\subsection{How does science work?}

\subsubsection{Dimension, measure and uncertainty}

We will use SI units. Defining such units is a hard task.

When we measure something we need to take into account the uncertainty of the measurement:
the tool we use to measure has a precision, hence we cannot measure something smaller than the precision of the tool.

Moreover we have to check that the measurements are compatible with the errors in each of them.

\subsubsection{Dimensionsional analysis}

When we do a calculation we need to check that the units are consistent.

This can also be done before the actual calculation to check that the theory is truthful.

\subsubsection{Theory vs reality}

In our theory we take in account way less phenomena than what happens in reality.
We need to understand what we can neglect and what we can't.

\begin{enumerate}
  \item Identify the key ingredients
  \item Build a simplified model of reality
  \item Solve it (if you can)
  \item Add additional ingredients
\end{enumerate}

\subsection{Introduction to kinematics}

In this course we will discuss \textbf{kinematics} and \textbf{dynamics} which are two branches of mechanics:
\begin{itemize}
  \item \textbf{Kinematics}: How did the object move?
  \item \textbf{Dynamics}: Why did the object move?
\end{itemize}

\subsubsection{Position vector}

We will start describing the motion of a point mass or a particle
and to do so we wll need to introduce vectors from linear algebra.

We have two ways to represent a vector.
Here we use the position vector as an example.

\textbf{Position vector} $\vec{r}$:
\begin{itemize}
  \item $\vec{r} = (x, y, z)$
  \item $\vec{r} = x \hat{i} + y \hat{j} + z \hat{k}$
\end{itemize}

In the second representation we have the \textbf{unit vectors} $\hat{i}, \hat{j}, \hat{k}$ which are the basis of the space.

Moreover we can represent the position vector as a function of time:
$$
  \vec{r}(t) = \begin{pmatrix} x(t) \\ y(t) \\ z(t) \end{pmatrix}
$$

$\vec {r}(t)$ is called trajectory.

\subsubsection{Degrees of freedom}

We call \textbf{degree of freedom} the number of independent coordinates needed to describe the position of a point.
Generally we need 3 degrees of freedom.

\begin{examplebox}{Example}
  \begin{itemize}
    \item Consider two points connected by a rigid rod of length $l$ (of negligible cross section).

          The position of the first point $A$ requires all 3 degrees of freedom,
          but for the second point $B$ we only need 2 degrees of freedom since the distance from $A$ is fixed.

    \item A rigid body has 6 degrees of freedom: 3 for the position of the center of mass and 3 for the orientation.
  \end{itemize}

\end{examplebox}

\subsubsection{Velocity}

We define the \textbf{displacement vector} as
$$
  \Delta \vec{r} = \vec{r}(t') - \vec{r}(t) \text{ with } t' > t
$$

Note that the displacement vector could be zero even if the particle moved.

We define the \textbf{average velocity} as $\vec{v}_{\text{avg}} = \frac{\Delta \vec{r}}{\Delta t}$.

We define the \textbf{instantaneous velocity} as
\begin{align*}
  \vec{v}(t) & = \lim_{\Delta t \to 0} \frac{\Delta \vec{r}}{\Delta t}                                                        \\
             & = \frac{d\vec{r}}{dt} \text{ this is the derivative of a vector}                                               \\
             & = \begin{pmatrix} \frac{dx(t)}{dt} \hat{i} & \frac{dy(t)}{dt} \hat{j} & \frac{dz(t)}{dt} \hat{k} \end{pmatrix}
\end{align*}

\subsubsection{Acceleration}

Similarly to the velocity, we define the \textbf{average acceleration} as $\vec{a}_{\text{avg}} = \frac{\Delta \vec{v}}{\Delta t}$
and we pass to the limit to get

\begin{align*}
  \vec{a}(t) & = \lim_{\Delta t \to 0} \frac{\Delta \vec{v}}{\Delta t} \\
             & = \frac{d\vec{v}}{dt} = \frac{d^2\vec{r}}{dt^2}
\end{align*}

We see that the acceleration is the second derivative of the position vector.

We will be able to reconstruct the whole trajectory starting from the acceleration.

\end{document}