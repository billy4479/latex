\documentclass[10pt]{extarticle}
\title{Physics 1 Notes}
\author{Giacomo Ellero}

\usepackage{amsfonts}
\usepackage{amsthm}
\usepackage{amssymb}
\usepackage{amsmath}
\usepackage{mathtools}
\usepackage{commath}
\usepackage{dirtytalk}
\usepackage{parskip}
\usepackage{mathrsfs}
\usepackage[many]{tcolorbox}
\usepackage{xparse}
\usepackage[a4paper,margin=1.5cm]{geometry}
\usepackage{bookmark}
\usepackage{physics}
\usepackage{tikz}

\newcommand{\C}{\mathbb{C}}
\newcommand{\R}{\mathbb{R}}
\newcommand{\F}{\mathcal{F}}
\renewcommand{\Re}{\operatorname{Re}}
\renewcommand{\Im}{\operatorname{Im}}

\newenvironment{absolutelynopagebreak}
  {\par\nobreak\vfil\penalty0\vfilneg
   \vtop\bgroup}
  {\par\xdef\tpd{\the\prevdepth}\egroup
   \prevdepth=\tpd}

\newtcolorbox{examplebox}[1]{colback=green!5!white,colframe=green!40!black,title={#1},fonttitle=\bfseries,parbox=false}
\newtcolorbox{notebox}[1]{colback=blue!5!white,colframe=blue!40!black,title={Note: #1},fonttitle=\bfseries,parbox=false}
\newtcolorbox{bluebox}[1]{colback=blue!5!white,colframe=blue!40!black,title={#1},fonttitle=\bfseries,parbox=false}
\newtcolorbox{warningbox}[1]{colback=orange!5!white,colframe=orange!90!black,title={Warning: #1},fonttitle=\bfseries,parbox=false}
   
\begin{document}

\maketitle
\tableofcontents
\clearpage

\section{Class of 06/02/2024 - Introduction}

\subsection{How does science work?}

\subsubsection{Dimension, measure and uncertainty}

We will use SI units. Defining such units is a hard task.

When we measure something we need to take into account the uncertainty of the measurement:
the tool we use to measure has a precision, hence we cannot measure something smaller than the precision of the tool.

Moreover we have to check that the measurements are compatible with the errors in each of them.

\subsubsection{Dimensionsional analysis}

When we do a calculation we need to check that the units are consistent.

This can also be done before the actual calculation to check that the theory is truthful.

\subsubsection{Theory vs reality}

In our theory we take in account way less phenomena than what happens in reality.
We need to understand what we can neglect and what we can't.

\begin{enumerate}
  \item Identify the key ingredients
  \item Build a simplified model of reality
  \item Solve it (if you can)
  \item Add additional ingredients
\end{enumerate}

\subsection{Introduction to kinematics}

In this course we will discuss \textbf{kinematics} and \textbf{dynamics} which are two branches of mechanics:
\begin{itemize}
  \item \textbf{Kinematics}: How did the object move?
  \item \textbf{Dynamics}: Why did the object move?
\end{itemize}

\subsubsection{Position vector}

We will start describing the motion of a point mass or a particle
and to do so we wll need to introduce vectors from linear algebra.

We have two ways to represent a vector.
Here we use the position vector as an example.

\textbf{Position vector} $\vec{r}$:
\begin{itemize}
  \item $\vec{r} = (x, y, z)$
  \item $\vec{r} = x \hat{i} + y \hat{j} + z \hat{k}$
\end{itemize}

In the second representation we have the \textbf{unit vectors} $\hat{i}, \hat{j}, \hat{k}$ which are the basis of the space.

Moreover we can represent the position vector as a function of time:
$$
  \vec{r}(t) = \begin{pmatrix} x(t) \\ y(t) \\ z(t) \end{pmatrix}
$$

$\vec {r}(t)$ is called trajectory.

\subsubsection{Degrees of freedom}

We call \textbf{degree of freedom} the number of independent coordinates needed to describe the position of a point.
Generally we need 3 degrees of freedom.

\begin{examplebox}{Example}
  \begin{itemize}
    \item Consider two points connected by a rigid rod of length $l$ (of negligible cross section).

          The position of the first point $A$ requires all 3 degrees of freedom,
          but for the second point $B$ we only need 2 degrees of freedom since the distance from $A$ is fixed.

    \item A rigid body has 6 degrees of freedom: 3 for the position of the center of mass and 3 for the orientation.
  \end{itemize}

\end{examplebox}

\subsubsection{Velocity}

We define the \textbf{displacement vector} as
$$
  \Delta \vec{r} = \vec{r}(t') - \vec{r}(t) \text{ with } t' > t
$$

Note that the displacement vector could be zero even if the particle moved.

We define the \textbf{average velocity} as $\vec{v}_{\text{avg}} = \frac{\Delta \vec{r}}{\Delta t}$.

We define the \textbf{instantaneous velocity} as
\begin{align*}
  \vec{v}(t) & = \lim_{\Delta t \to 0} \frac{\Delta \vec{r}}{\Delta t}                                            \\
             & = \dv{\vec{r}}{t} \text{ this is the derivative of a vector}                                       \\
             & = \begin{pmatrix} \dv{x(t)}{t} \hat{i} & \dv{y(t)}{t} \hat{j} & \dv{z(t)}{t} \hat{k} \end{pmatrix}
\end{align*}

\subsubsection{Acceleration}

Similarly to the velocity, we define the \textbf{average acceleration} as $\vec{a}_{\text{avg}} = \frac{\Delta \vec{v}}{\Delta t}$
and we pass to the limit to get

\begin{align*}
  \vec{a}(t) & = \lim_{\Delta t \to 0} \frac{\Delta \vec{v}}{\Delta t} \\
             & = \dv{\vec{v}}{t} = \dv[2]{\vec{r}}{t}
\end{align*}

We see that the acceleration is the second derivative of the position vector.

We will be able to reconstruct the whole trajectory starting from the acceleration.

\section{Class of 08/02/2024 - Kinematics}

\subsection{Decomposing vectors}

Consider a 2D position vector, let $\theta$ be the angle between the vector and the $x$ axis,
let $r = \norm{\vec{r}}$ be the magnitude of the vector,
and let $\hat{r} = \vec{r} / r$ be the unit vector in the direction of $\vec{r}$.

\begin{align*}
  \vec{r} & = x \hat{i} + y \hat{j}                         \\
          & = r \cos \theta \hat{i} + r \sin \theta \hat{j} \\
          & = r \hat{r}
\end{align*}

This decomposition works for any vector in 2D, like the velocity and the acceleration.

\subsection{Changing the direction of motion is an acceleration}

According to these definitions we have that the acceletation, which is the derivative of the velocity, can be written as

\begin{align*}
  \vec{a} & = \dv{\vec{v}}{t} = \dv{t} (v \hat{v}) \\ &=
  \underbrace{ \dv{v}{t} \hat{v} }_{\text{change in speed}} +
  \underbrace{ v \dv{\hat{v}}{t} }_{\text{change in direction}}
\end{align*}

We see how changing the direction of the velocity will cause an acceleration.

\subsection{Recovering the trajectory from the acceleration}

We can use differential equations to get that

\begin{align*}
  \vec{v}(t) & = \int \vec{a}(t) \dd{t} + c_1 \\
  \vec{r}(t) & = \int \vec{v}(t) \dd{t} + c_2
\end{align*}

In the result we get two constants of integration $c_1$ and $c_2$
which can be determined by knowing the position or velocity at a certain time.

For example if $v(t_0)$ and $r(t_0)$ are known we have that, at $t_1$,

\begin{align*}
  \vec{v}(t_1) & = \vec{v}(t_0) + \int_{t_0}^{t_1} \vec{a}(t) \dd{t} \\
  \vec{r}(t_1) & = \vec{r}(t_0) + \int_{t_0}^{t_1} \vec{v}(t) \dd{t}
\end{align*}

\subsubsection{Uniformly accelerated motion}

If we consider the case where $\vec{a}$ is constant we get
the high-school equation for uniformly accelerated motion:

\begin{align*}
  \vec{v}(t) & = \vec{a} t - v(t_0)                                \\
  \vec{r}(t) & = \frac{1}{2} \vec{a} t^2 - v(t_0) t + \vec{r}(t_0)
\end{align*}

Moreover, we can substitute one into the other to get an equation for the trajectory:
$$
  r = \frac{v^2(t) - v^2(t_0)}{2a}
$$

\subsection{Relative motion}

Multiple observers may have different reference frames,
we need a way to convert from a reference frame to the other.

\begin{center}
  \begin{tikzpicture}
    % Normal axis
    \draw[black, ->] (-3,0) -- (5,0) node[anchor=north]{$\hat{i}$};
    \draw[black, ->] (0,-3) -- (0,5) node[anchor=east]{$\hat{j}$};

    % Rotated axis
    \draw[dashed, black, ->] (-3,-1.8) -- (5,3) node[anchor=south]{$\hat{i'}$};
    \draw[dashed, black, ->] (1.8,-3) -- (-3,5) node[anchor=east]{$\hat{j'}$};

    % Vector
    \draw[blue, thick, ->] (0,0) -- (3,3) node[anchor=south]{$\vec{r}$};

    % Angle between axis
    \draw[red, thick] (0.5,0) arc (0:31:0.5) node[right]{$\theta$};
  \end{tikzpicture}
\end{center}

We can write $\hat{i}$ and $\hat{j}$ in terms of $\hat{i'}$ and $\hat{j'}$:

\begin{align*}
  \hat{i'} & = \left( \hat{i'} \cdot \hat{i} \right) \hat{i}
  + \left( \hat{i'} \cdot \hat{j} \right) \hat{j}            \\
           & = \cos \theta \hat{i} + \sin \theta \hat{j}     \\
  \hat{j'} & = \left( \hat{j'} \cdot \hat{i} \right) \hat{i}
  + \left( \hat{j'} \cdot \hat{j} \right) \hat{j}            \\
           & = -\sin \theta \hat{i} + \cos \theta \hat{j}
\end{align*}

Then, $\vec{r} = x \hat{i} + y \hat{j}$ can be written as

\begin{align*}
  \vec{r} & = x \left( \cos \theta \hat{i'} + \sin \theta \hat{j'} \right)  + y \left( -\sin \theta \hat{i'} + \cos \theta \hat{j'} \right) \\
          & = \underbrace{ (x \cos \theta - y \sin \theta) }_x \hat{i'}
  + \underbrace{ (x \sin \theta + y \cos \theta) }_y \hat{j'}
\end{align*}

We just performed a linear transformtion on the position vector,
we can write the matrix representing this transformation:

$$
  R = \begin{pmatrix}
    \cos \theta & -\sin \theta \\
    \sin \theta & \cos \theta
  \end{pmatrix}
$$

We have
$$
  \begin{pmatrix}
    x \\
    y
  \end{pmatrix}
  =
  \begin{pmatrix}
    \cos \theta & -\sin \theta \\
    \sin \theta & \cos \theta
  \end{pmatrix}
  \begin{pmatrix}
    x' \\
    y'
  \end{pmatrix}
$$

The inverse transformation is given by the inverse of the matrix:

$$
  R^{-1} = \begin{pmatrix}
    \cos \theta  & \sin \theta \\
    -\sin \theta & \cos \theta
  \end{pmatrix}
$$

\subsubsection{Frames of reference in relative motion}

Using vectors we can abstract away the complexity derived from the fact that the reference frames are moving.

Consider the classic example of a train moving at constant speed, with two observers, one on the train and one on the ground,
and a ball which is moving inside the train.

We have that the trajectory of the ball for the observer on the ground is given by
$$
  \vec{r}_{gb} = \vec{r}_{gt} + \vec{r}_{tb}
$$

We don't need to worry about the direction of the train, we can just sum the vectors.

Similar considerations can be made for the velocity and the acceleration.

\end{document}