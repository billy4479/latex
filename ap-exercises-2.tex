\documentclass[12pt]{extarticle}

\title{Advanced Programming and Optimization Algorithms \\ Homework 02}
\author{Giacomo Ellero}
\date{26/03/2025}

\usepackage{preamble_base}
\usepackage{preamble_math}

\numberwithin{equation}{section}

\begin{document}

\maketitle

\section*{Exercise 1}
\stepcounter{section}

Recall the definition of convex function: $f: X \to Y$ is convex if and only if
$\forall x_1, x_2 \in X$ and $\forall \tau \in [0, 1]$ we have
\begin{equation}
	f(\tau x_1 + \overline \tau x_2) \leq \tau f(x_1) + \overline \tau f(x_2)
\end{equation}
where we use the notation $\overline \tau = 1 - \tau$.

We will show that all the functions in the sequence of the partial sums $g_1, \dots, g_k = g$ are
convex.
Assume that $\alpha_1, \dots \alpha_k > 0$, since if $\alpha_i = 0$ it is equivalent to removing it
from the sum, therefore we could construct another sequence $\tilde \alpha_n$ of length $k - 1$
where we have removed $\alpha_i$ and we wouldn't alter the resulting $g$.
We proceed by induction on $i \in \{1, \dots, k\}$.

\begin{description}[font=\normalfont\itshape]
	\item[Base step] For $i = 1$ we have $g_1(x) = \alpha_1 f_1(x)$, so that $\forall x_1, x_2 \in X$
	      and $\forall \tau \in [0, 1]$ we have
	      \begin{align}
		      g_1(\tau x_1 + \overline \tau x_2) & = \alpha_1 f_1(\tau x_1 + \overline \tau x_2)                  \\
		                                         & \leq \alpha_1 \tau f_1(x_1) + \alpha_1 \overline \tau f_1(x_2) \\
		                                         & = \tau g_1(x_1) + \overline \tau g_1(x_2)
	      \end{align}
	      where we have used the definition of $g_1$, the convexity of $f_1$ and the fact that
	      $\alpha_1 > 0$ and the definition of
	      $g_1$ again.
	      We conclude that $g_1$ is convex.
	\item[Inductive step] Assume that $g_1, \dots, g_i$ are convex. We want to show that $g_{i+1}$ is
	      also convex.
	      We have that, $\forall x_1, x_2 \in X$ and $\forall \tau \in [0, 1]$,
	      \begin{align}
		      g_{i + 1}(\tau x_1 + \overline \tau x_2) & = g_i(\tau x_1 + \overline \tau x_2) +
		      \alpha_{i + 1} f_{i + 1}(\tau x_1 + \overline \tau x_2)                                      \\
		                                               & \leq \tau g_i(x_1) + \overline \tau g_i(x_2) +
		      \alpha_{i+1} \tau f_{i+1}(x_1) + \overline \tau f(x_2)                                       \\
		                                               & = \tau g_{i+1}(x_1) + \overline \tau g_{i+1}(x_2)
	      \end{align}
	      where we have used the definition of $g_{i+1}$, the inductive hypothesis and the convexity
	      of $f_{i+1}$ (and the fact that $\alpha_{i+1} > 0$, and finally again the definition of
	      $g_{i+1}$. We conclude that $g_{i+1}$ is also convex.
\end{description}

Therefore $g$ is convex.

\clearpage

\section*{Exercise 2}
\stepcounter{section}

We substitute each variable $x_1, \dots x_5$ with $x_i^+ - x_i^-$ both $\geq 0$ which represent the
positive and negative part of each variable $x_i$.
Then we add the slack variables $s_1, \dots, s_4 \geq 0$, one for each constraint.

We obtain the equation form of the polyhedron:
\begin{gather*}
	\begin{array}{ccccc cccc cc}
		+ x_1^+ - x_1^- & + x_2^+ - x_2^-    &                 &                   &                 & + s_1 &       &       &       & = & 3  \\
		                & + x_2^+ - x_2^-    & + x_3^+ - x_3^- &                   &                 &       & + s_2 &       &       & = & 12 \\
		+ x_1^+ - x_1^- & + 3(x_2^+ - x_2^-) &                 & - (x_4^+ - x_4^-) &                 &       &       & - s_3 &       & = & -7 \\
		                &                    &                 &                   & + x_5^+ - x_5^- &       &       &       & - s_4 & = & 6
	\end{array} \\
	\begin{array}{ccc}
		x_i^+, x_i^- & \geq 0 & \enspace \forall i \in \{1, \dots, 5\} \\
		s_i          & \geq 0 & \enspace \forall i \in \{1, \dots, 4\}
	\end{array}
\end{gather*}

From the equation form we notice that if we set all the $x_i^+ = x_i^- = 0$ for all $i$ we obtain a
basic feasible solution where $s_1 = 3$, $s_2 = 12$, $s_3 = 7$, and $s_4 = -6$.

\end{document}
