\documentclass[10pt]{extarticle}
\title{Probability Notes}
\author{Giacomo Ellero}

\usepackage{amsfonts}
\usepackage{amsthm}
\usepackage{amssymb}
\usepackage{amsmath}
\usepackage{mathtools}
\usepackage{commath}
\usepackage{dirtytalk}
\usepackage{parskip}
\usepackage{mathrsfs}
\usepackage[many]{tcolorbox}
\usepackage{xparse}
\usepackage[a4paper,margin=1.5cm]{geometry}
\usepackage{bookmark}
\usepackage{bytefield}

\newcommand{\C}{\mathbb{C}}
\newcommand{\R}{\mathbb{R}}
\newcommand{\N}{\mathbb{N}}
\newcommand{\F}{\mathcal{F}}
\renewcommand{\Re}{\operatorname{Re}}
\renewcommand{\Im}{\operatorname{Im}}

\newenvironment{absolutelynopagebreak}
  {\par\nobreak\vfil\penalty0\vfilneg
   \vtop\bgroup}
  {\par\xdef\tpd{\the\prevdepth}\egroup
   \prevdepth=\tpd}

\newtcolorbox{examplebox}[1]{colback=green!5!white,colframe=green!40!black,title={#1},fonttitle=\bfseries,parbox=false}
\newtcolorbox{notebox}[1]{colback=green!5!white,colframe=blue!40!black,title={Note: #1},fonttitle=\bfseries,parbox=false}
\newtcolorbox{bluebox}[1]{colback=green!5!white,colframe=blue!40!black,title={#1},fonttitle=\bfseries,parbox=false}

\begin{document}

\maketitle
\tableofcontents
\clearpage

\section{Class of 06/02/2024}

\subsection{Asymptotic notation}

\begin{bluebox}{Definition}
    If $\exists C \in \R^+$ and $N \in \N$ such that for two sequences $a_n, b_n > 0$
    we have that $a_n \leq Cb_n$ for all $n \geq N$, then we write $a_n = O(b_n)$ and we read \say{$a_n$ is big O of $b_n$}.
\end{bluebox}

These sequences describe the time it takes for an algorithm to solve a certain problem of size $n$.

\begin{examplebox}{Example}
    Let $a_n = n^2 + 2n + 1$. We will prove that $a_n = O(n^2)$.

    \begin{proof}
        We have that
        \begin{align*}
            a_n & = n^2 + 2n + 1        \\
                & \leq n^2 + 2n^2 + n^2 \\
                & = 4n^2
        \end{align*}
    \end{proof}

    As we can see just need to show that $C$ exists, we don't need to find the best one.
\end{examplebox}

Usually we don't use limits to prove that a sequence is big O of another, it is usually more convenient to proceed by inequalities.

\begin{notebox}{Logarithms}
    When we have sequences with logarithms we don't need to specify the basis,
    as the logarithm is a constant factor of another logarithm by the change of basis formula.
\end{notebox}

\subsubsection{Operations and other notation}

Let $a_n = O(c_n)$ and $b_n = O(d_n)$, then we have that

\begin{itemize}
    \item $a_n + b_n = O(\max\{c_n, d_n\})$
    \item $a_n \cdot b_n = O(c_n \cdot d_n)$
\end{itemize}

We also define the \say {opposite} of big O notation, the $\Omega$ notation:
$a_n$ is $\Omega(b_n)$ if $a_n \geq Cb_n$ for all $n \geq N$.

Moreover, if $a_n = O(b_n)$ and $a_n = \Omega(b_n)$ (for some different $C$ and $N$) then we write $a_n = \Theta(b_n)$.

\subsection{Randomness}

When we run an experiment we can define
\begin{itemize}
    \item an outcome $\omega_i$
    \item the value of the outcome $x_i$ (similar to a bet)
    \item the probability of the outcome $p_i$
\end{itemize}

We can also define the expected result $E(X)$ of the experiment as
$$
    E(X) = \sum ^n _{i = 1} x_i p_i
$$

\subsection{IEEE-754}

This is not in the syllabus but it's useful to know.

IEEE-754 is a standard for representing floating point numbers in computers.
We will discuss in particular the 32-bits representation but larger or smaller formats also exist.

We subdivide the 32 bits in 3 parts: 1 bit for the \textbf{sign}, 8 bits for the \textbf{exponent}, and 23 bits for the \textbf{fraction} or mantissa.

\begin{center}
    \begin{bytefield}[bitwidth=1.1em, bitheight=\widthof{~Sign~}]{32}
        \bitheader{0,8,31} \\
        \bitbox{1}{1}
        \bitboxes{1}{00100101}
        \bitboxes{1}{00110110101101001111011} \\
        \bitbox{1}{\rotatebox{90}{Sign}}
        \bitbox{8}{Exponent}
        \bitbox{23}{Fraction}
    \end{bytefield}
\end{center}

The number is interpreted according to the following formula:

$$
    n = (-1)^s \cdot 2^{e - 127} \cdot (1 + f)
$$

Basically the number is represented in scientific notation, in base 2.

Note that when converting the fraction part to decimal the powers of 2 are negative and decreasing
(i.e.: bit at 9 is $2^{-1}$, bit at 10 is $2^{-2}$, etc).

\end{document}