\documentclass[12pt]{extarticle}

\author{Giacomo Ellero}

% *------------------*
% | *** Packages *** |
% *------------------*

\usepackage{parskip} % for no indentation
\usepackage[a4paper,margin=1.5cm]{geometry} % to adjust margins
\usepackage{bookmark} % for pdf bookmarks

\usepackage{float} % for H in figures
\usepackage{graphicx} % for images
\usepackage{subfig} % for subfigures

\usepackage{dirtytalk} % for quotes
\usepackage[many]{tcolorbox} % for boxes
% \usepackage{tikz} % for drawings
\usepackage{enumitem} % for custom lists
\usepackage{hyperref} % for hyperlinks

% *---------------*
% | *** Fonts *** |
% *---------------*

\usepackage{fontspec}

\setsansfont{Ubuntu}
% \setmonofont{FiraCode Nerd Font} % the normal font doesn't support bold for some reason

% Use sans-serif by default
\renewcommand{\familydefault}{\sfdefault}

% *---------------*
% | *** Links *** |
% *---------------*

\hypersetup{
    colorlinks=true,
    linkcolor=teal,
    urlcolor=blue,
}
\urlstyle{same}

% *----------------*
% | *** Legacy *** |
% *----------------*
% TODO: inline them in the documents that need them

\newenvironment{absolutelynopagebreak}
{\par\nobreak\vfil\penalty0\vfilneg
    \vtop\bgroup}
{\par\xdef\tpd{\the\prevdepth}\egroup
    \prevdepth=\tpd}

% *-------------------*
% | *** Numbering *** |
% *-------------------*

\numberwithin{table}{section}
\numberwithin{figure}{section}

% We want numbers on subsections too
\setcounter{secnumdepth}{3}

% *-------------------*
% | *** Utilities *** |
% *-------------------*

% Make the first page, with title, toc and link to toc
\newcommand{\firstpage}{
    \maketitle
    \phantomsection
    \hypertarget{toc}{}
    \tableofcontents
    \clearpage
}

% *------------------*
% | *** Packages *** |
% *------------------*

\usepackage{amssymb}   % for varnothing and other weirds symbols
\usepackage{amsmath}   % basically everything
\usepackage{mathtools} % for underbrace, arrows, and a lot of other things
\usepackage{mathrsfs}  % for "mathscr"
\usepackage{bm}        % for bold math symbols
\usepackage{physics}   % for derivatives and lots of operators
\usepackage{cancel}    % for canceling terms

% *-------------------*
% | *** Shortcuts *** |
% *-------------------*

% Operators
\newcommand{\gt}{>}
\newcommand{\lt}{<}
\newcommand{\indep}{\perp \!\!\! \perp}

% Sets
\newcommand{\C}{\mathbb{C}}
\newcommand{\R}{\mathbb{R}}
\newcommand{\N}{\mathbb{N}}
\newcommand{\Q}{\mathbb{Q}}
\newcommand{\Z}{\mathbb{Z}}

% Big-O and small-o notation
\renewcommand{\O}{\mathcal{O}}
% https://tex.stackexchange.com/questions/191059/how-to-get-a-small-letter-version-of-mathcalo
\renewcommand{\o}{
    \mathchoice
    {{\scriptstyle\mathcal{O}}}% \displaystyle
    {{\scriptstyle\mathcal{O}}}% \textstyle
    {{\scriptscriptstyle\mathcal{O}}}% \scriptstyle
    {\scalebox{.6}{$\scriptscriptstyle\mathcal{O}$}}%\scriptscriptstyle
}

% *-------------------*
% | *** Utilities *** |
% *-------------------*

% Skip line after theorem or proof title, useful when using itemize or enumerate
\newcommand{\skiplineafterproof}{$ $\par\nobreak\ignorespaces}

% Use underbar in math mode to define vectors
% Sorcery taken from https://tex.stackexchange.com/a/163284
\makeatletter
\def\munderbar#1{\underline{\sbox\tw@{$#1$}\dp\tw@\z@\box\tw@}}
\makeatother

\newcommand{\uvec}[1]{\munderbar{\bm{#1}}}

% *-------------------*
% | *** Theorems *** |
% *-------------------*

% yes, i know, the formatting sucks

\usepackage{amsthm}    % for theorem styles
\usepackage{thmtools}  % for "declaretheorem"
\usepackage{tcolorbox} % for boxes around theorems

\theoremstyle{definition}

% theorem
\declaretheorem[
    numberwithin=section,
    numbered=no,
    refname={Thm.},
    title={Theorem},
]{theorem*}


% *-------------------*
% | *** Equation *** |
% *-------------------*

% \numberwithin{equation}{subsection}
\def\equationautorefname{Eq.} % Autoref

\usepackage{preamble_code}

\begin{document}

\section*{Dynamic programming exercise}

\textbf{Prompt}: Given a sequence $a_n$ of length $N$ find the length $L$ of the longest strictly increasing subsequence.

\subsection*{Algorithm description}

Let $b_n$ be another sequence of length $N$, where each element $b_i$ represents the length of the longest strictly increasing subsequence that has $a_i$ as first item.

We will assigning values to $b_n$ starting from $b_N$ going back to $b_1$ using the following criterion:
\begin{equation}
    \label{main_eq}
    b_i = \max \{ b_{j} : i < j \leq N \land a_j > a_i \} + 1
\end{equation}

Then $L = \max b_n$.

\subsection*{Code}

Note that this is real python code that can be run, therefore both \texttt{a[i]} and \texttt{b[i]} are $0$ indexed,
unlike the description provided above, where they start from $1$.

\begin{minted}{python}
def longest_increasing_subsequence(a):
    # Here a[N] is out of bound because the first element of a
    # is a[0] and the last one is a[N-1]
    N = len(a)
    b = [0 for _ in range(len(a))]

    # Loop from N-1 (inclusive) to -1 (non-inclusive) decreasing i at each step.
    # This is equivalent to looping from N-1 to 0 inclusive.
    for i in range(N-1, -1, -1):
        if i == N-1:
            # If we are at the last item in the list the longest
            # increasing subsequence just contains a_i itself, therefore
            # it has length 1
            b[i] = 1
            continue

        m = 0

        # Loop from i+1 (inclusive) to N (non-inclusive),
        # this time increasing j at each step.
        for j in range(i+1, N, +1):
            # This if imposes the strictly increasing condition
            if a[j] > a[i]:
                # Store the maximum length of the sequence that can be 
                # started from some j > i
                m = max(m, b[j])

        # Add 1 since we need to count a[i]
        b[i] = m + 1

    L = max(b)
    return L
\end{minted}

\subsection*{Proof of correctness}

\emph{In this section lists start from $1$ and end at $N$ as in the description.}

\textbf{Claim}: At each iteration of the algorithm $b_n$ contains the length of the longest strictly increasing of $a_n$ which has as first element the element of $a_n$ with corresponding index (i.e. $b_5$ has the length of the sequence that starts from $a_5$).
Then $L = \max b_n$.

\begin{proof}
    We will proceed by induction on $t$.
    Let $i = N - t$.
    \begin{description}
        \item[Base case] We have $t = 0$ and $i = N$. The longest strictly increasing subsequence that starts from $a_N$ is the subsequence that contains just $a_N$, therefore it has length $1$.
        \item[Indutive step] Assume the claim is true at iteration $t$ and at all the previous iterations; we want to show that it will be true at $t+1$. At $t+1$ we have that $i = N - t - 1$.

              Call $c^*_n$ the longest strictly increasing subsequence that has $a_i$ as the first element, that is $c^*_1 = a_i$.
              Now consider $c^*_2$: this will be equal to $a_j$ for some $i < j \leq N$ with $a_j > a_i$.

              It is possible that there will be more than one choice of $j$ respecting these constraints but in order to maximize the length of $c^*_n$ we have to choose the one from which there starts the longest subsequence: this information is contained in $b_j$ which we know it contains the right value because $j$ has been set during some iteration $t' < t$.
    \end{description}

    Notice that \autoref{main_eq} chooses $b_i$ according to the proof.

    Moreover, since the set of all the increasing subsequences of $a_n$ can be partitioned by starting element of each subsequence, and since $b_n$ contains the length of the sequence with maximum length over each partition, $L = \max b_n$ is the solution to the problem.
\end{proof}

\end{document}
