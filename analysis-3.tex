\documentclass[12pt]{extarticle}

\setlength{\headheight}{16pt} % ??? we do what fancyhdr tells us to do  

\title{Analysis 3}
\author{Giacomo Ellero}
\date{a.y. 2024/2025}

\usepackage{preamble}

\renewcommand{\vec}[1]{\uvec{#1}}

\begin{document}

\firstpage

\section{Ordinary differential equations}

\subsection{Introduction: What are differential equations}

Differential equations are equations where the unknown is a function (usually $u(t)$).
The equation usually involves regular functions, $u$ and its derivatives up to an arbitrary order $k$.

If $u$ is a function of more than one variable we call this type of equation \textbf{Partial Differential Equations (PDE)}.
In this course we will focus on \textbf{Ordinary Differential Equations (ODE)}:
in these equations $u$ is a functions of only one variable which is evaluated at the same point $t$, with a finite number of derivatives involved.

We will focus on ODEs in \textbf{normal form}:
\begin{equation}
    \label{eq:normal-form}
    \dv[k]{u(t)}{t} = f\left( t, u(t), u'(t), \dots, \dv[k-1]{u(t)}{t} \right)
\end{equation}
where $f$ is a known function with some properties we will describe later.
For this kind of equation there exists a general theory on how to solve them.

Note that we don't really have to worry about the domain: the solution itself will tell us what its natural domain is.
Moreover, not all ODEs have a nice explicit formula for their solution.

In order to get a unique solution for a given ODE we need so specify some \textbf{initial conditions}, that is, we need to be given the values of $u(t_0)$ up to $u^{(k-1)}(t_0)$.

We will use $u(t)$ when we want to represent something that varies in time and $u(x)$ when something varies in space.
Moreover, sometimes, for brevity, I will write $u$ instead of $u(t)$, $u'$ instead of $u'(t)$ and so on.

\subsubsection{Separation of variable}

This is the easiest type of differential equation. We have seen them in physics before, now we will prove how to solve them.

\begin{definition}{Separable differential equation}{separable}
    Let $f: \R^2 \to \R$.
    We say a that a differential equation is separable if it is of the form
    \begin{equation}
        u' = f(t, u) = g(t) \cdot h(u)
    \end{equation}
    for some functions $g(t)$ and $h(u)$.
\end{definition}

\begin{proof}[Solution]
    First, separate the variables:
    \begin{equation}
        \frac{u'}{h(u)} = g(t)
    \end{equation}

    Let $H$ be the primitive of $\frac{1}{h}$ and $G$ the primitive of $g$.
    Now let's differentiate $H(u(t))$ using the chain rule:
    \begin{equation}
        \dv{t} H(u(t)) = H'(u(t)) \cdot u'(t) = \frac{u'}{h(u)}
    \end{equation}
    which we know to be equal to $g(t) = G'(t)$.
    Therefore, by integrating both sides w.r.t. $t$ we get the solution.
\end{proof}

\section{Existence and uniqueness of the solution}

\subsection{Reminder of old courses}

For this section we will need to recall a few tools from topology.

\begin{definition}{Euclidean space and norm}{euclidean-space-norm}
    An euclidean space is a vector space where an operator called \textit{inner product} is defined.
    Then norm is defined as $\norm{\vec x} = \langle \vec x, \vec x \rangle$.

    If $\vec x \in \R^n$, given the definition of inner product in $\R^n$, we get that
    \begin{equation}
        \norm{\vec x} = \sqrt{x_1^2 + \dots + x_n^2}
    \end{equation}
\end{definition}

For a complete list of properties of the inner product or the norm see the Linear Algebra notes.
For our purposes we will recall the following ones:

\begin{proposition}{Notable properties of the norm}{norm-props}
    Let $\lambda \in \mathbb F$ and $\vec x, \vec y \in \R^n$. Then
    \begin{itemize}
        \item $\norm{\lambda \vec x} = \abs{\lambda}\norm{\vec x}$ (\textit{homogeneity}).
        \item $\norm{\vec x + \vec y} \leq \norm{\vec x} + \norm{\vec y}$ (\textit{triangle inequality}).
    \end{itemize}
\end{proposition}

The following definitions will also result useful. These were covered in Analysis 2.

\begin{definition}{Open set}{open-set}
    Let the ball $B_r(p)$ be defined as
    \begin{equation}
        B_r(p) = \{ q \in \R^n : \norm{q - p} < r \}
    \end{equation}
    with $p \in \R^n$ and $r \in \R$.

    A set $A \in \R^n$ is open if for all points $p \in A$ there exists a ball $B_r(p)$ with $r > 0$ such that $B_r(p) \subseteq A$.
\end{definition}

\begin{proposition}{Properties of open sets}{props-open-set}
    \begin{itemize}
        \item The union of open sets is still an open set.
        \item The intersection of finitely many open sets is an open set.
        \item A set is open iff it can be written as the union of open balls.
        \item Given a function $f:A \to \R^n$ with $A \subseteq \R^m$ open, $f$ is continuous iff for any $U \subseteq \R^n$ the preimage $f^{-1}(U)$ is an open set (in $\R^m$).
    \end{itemize}
\end{proposition}

The following definitions and properties where covered in detail in Analysis 1.

\begin{definition}{Lipschitz functions}{lipschitz-functions}
    A function $f: S \to \R^n$ with $S\in \R^m$ is Lipschitz if $\forall a, b \in S$ there exists $C \in \R$ such that
    \begin{equation}
        \norm{f(a) - f(b)} \leq C \norm{a - b}
    \end{equation}

    Moreover, a function is \textbf{locally Lipschitz} if, assuming this time $S$ open,
    for any $p \in S$ there exists a ball $B_r(p) \in S$ with $r > 0$ where the function is Lipschitz.
\end{definition}

\begin{proposition}{Properties of Lipschitz functions}{props-lipschitz}
    \begin{itemize}
        \item Lipschitz functions are also locally Lipschitz.
        \item Locally Lipschitz functions are also continuous.
        \item A function is Lipschitz iff all its partial derivatives are bounded.
        \item If a functions is differentiable with continuous partial derivatives then it is locally Lipschitz
    \end{itemize}
\end{proposition}

\subsection{Main result of ODE theory}

We will present this theorem in various forms, from the most basic one to the most sophisticated one.
This theorem comes with different names such as Peano-Picard theorem, Picard-Lindelöf theorem, or Cauchy-Lipschitz.

\begin{theorem}{Picard–Lindelöf theorem}{picard-lindelof}
    Let $f: A \to \R$ with $A \subseteq \R^2$ open and $f$ jointly continuous on both inputs and locally Lipschitz in the second input.
    Then, for any $t_0, \lambda_0 \in \R$ there exists a unique \emph{maximal solution}
    \begin{equation}
        u: I \to \R
    \end{equation}
    where $I$ is an open interval called the \emph{interval of maximal existence} such that $t_0 \in I$ and
    \begin{equation}
        \begin{cases}
            u'(t) = f(t, u(t)) \\
            u(t_0) = \lambda_0
        \end{cases}
    \end{equation}
\end{theorem}

\begin{remark}{Meaning of unique}{}
    In this context \say{unique} means that if $v: J \to \R$ is another solution defined on another interval $J$ containing $t_0$ then $J \subseteq I$ and $v = u|_J$.
\end{remark}

\begin{remark}{Existance if not locally Lipschitz}{}
    Peano proved that the existence of a solution is guaranteed even if the function is continuous but not locally Lipschitz.
    Uniqueness is not guaranteed though.
\end{remark}

\begin{corollary}{}{}
    Let $u$ and $v$ be solutions for the same ODE with different initial conditions, $t_0$ and $t_1$ respectively.
    Then $u$ and $v$ never cross.
\end{corollary}

\begin{proof}
    We have that $u(t_0) \neq v(t_0)$ hence we cannot have that $u(t_i) = v(t_i)$ at some other point $t_i$ because otherwise we could apply uniqueness with initial point $t_i$ and we would have that are the same function on a restricted domain,
    contradicting that $u(t_0) \neq v(t_0)$.
\end{proof}

\begin{theorem}{Peano theorem}{peano}
    Let $f: A \to \R$ with $A \subseteq \R^2$ open and $f$ continuous.
    Then, for any $t_0, \lambda_0 \in \R$ there exists a solution
    \begin{equation}
        u: I \to \R
    \end{equation}
    where $I$ is an open interval such that $t_0 \in I$ and
    \begin{equation}
        \begin{cases}
            u'(t) = f(t, u(t)) \\
            u(t_0) = \lambda_0
        \end{cases}
    \end{equation}

    Note that uniqueness is not implied.
\end{theorem}

\section{System of ODEs}

\begin{definition}{First order system of ODEs in normal form}{system-odes}
    A first order system of ODEs in normal form is a system of $n$ equations
    \begin{equation}
        \begin{cases}
            u_1' = f(t, u_1, \dots, u_n) \\
            \vdots                       \\
            u_n' = f(t, u_1, \dots, u_n) \\
        \end{cases}
    \end{equation}
    where $f_1, \dots, f_n: A \to \R$ and $A \in \R^{n+1}$ open.
\end{definition}

A solution to a system of ODEs consists of $u_1, \dots, u_n: I \to \R$, with $I \subseteq \R$ is an interval.

An initial value problem for a system of ODEs is in the form of
\begin{equation}
    {\operatorfont IVP:}
    \begin{cases}
        u_1' = \ldots             \\
        u_1(t_0) = \lambda_{0, 1} \\
        \vdots                    \\
        u_n' = \ldots             \\
        u_n(t_0) = \lambda_{0, n} \\
    \end{cases}
\end{equation}
with $(t_0, \lambda_{0,1}, \dots, \lambda_{0,n}) \in A$.

\subsection{Solutions of systems of ODEs}

These systems can be thought as a ODE with vector-valued functions:
\begin{theorem}{Picard–Lindelöf theorem - second formulation}{picard-lindelof-2nd}
    Let $f: A \to \R^n$ with $A \subseteq \R^{n+1}$ open and $f$ of class $C^1$.
    Then, for any $t_0 \in \R$ and $\lambda_0 \in \R^n$ s.t. $(t_0, \lambda_0) \in A$ there exists a unique \emph{maximal solution}
    $u: I \to \R^n$
    where $I$ is an open interval called the \emph{interval of maximal existence} such that $t_0 \in I$ and
    \begin{equation}
        \begin{cases}
            u'(t) = f(t, u(t)) \\
            u(t_0) = \lambda_0
        \end{cases}
    \end{equation}
\end{theorem}

\begin{theorem}{Escape from compacts}{escape-from-compacts}
    Let $K \in A$ compact, the maximal solution $u: I \to \R^n$ to IVP is s.t. $I$ is open and $\exists [a, b] \in I$ s.t. $(t, u(t)) \notin K$ for $t \notin [a, b]$.
\end{theorem}

\begin{corollary}{}{}
    Let $A = \R^{n+1}$. If $u: I \to \R^n$ is the maximal solution and $u$ is bounded, then $I = \R$.
\end{corollary}

\begin{theorem}{}{}
    Let $A, f$ as in \Cref{thm:picard-lindelof-2nd}.
    Assume $\abs{f(t, \lambda) - f(t, \lambda')} \leq c(t)\abs{\lambda - \lambda'}$, where $c: \R \to [0, \infty)$.
    Then any IVP has solution defined on $I = \R$.
\end{theorem}

\subsection{Equivalence between higher order ODEs and first order systems}

Let us consider a scalar ODE like
\begin{equation}
    u'' = 2 u' - 3u
\end{equation}
we can define $v_1 = u$ and $v = u'$ and rewrite the ODE as a system of first order ODEs as follows:
\begin{equation}
    \begin{cases}
        v_1' = v_2 \\
        v_2' = 2 v_2 - 3v_1
    \end{cases}
\end{equation}

These two formulations are equivalent.
This works for any higher order ODE and it is the standard procedure.

\section{Bootstrap}


\begin{proposition}{Bootstrap}{bootstrap}
    Consider and ODE $u' = f(t, u)$ with $f : A \to \R$ and $A \in \R^2$ open.
    Assume $f$ is of class $C^m$ for some $m \geq 1$.
    Then any solution $u$ is of class $C^{m+1}$.
\end{proposition}

\begin{proof}
    Proceed by induction on $l = 0, 1, \dots, m$.
    We want to prove that $u$ is of class $l +1$.
    \begin{description}
        \item[Base case] $l = 0$. We know that $u$ is differentiable by ODE $\implies$ $u$ continuous $\implies$ $f(t, u(t))$ is continuous $\implies$ $u'$ is continuous.
        \item[Inductive step] By induction $u$ is $C^{l+1}$ which means it is also $C^1$.
              We need to prove that $u$ is $C^{l+2}$.
              Let $g(t) = (t, u(t))$ which is $C^1$.
              Then $f \circ g$ is $C^1$ and
              \begin{equation}
                  u''(t) = (f \circ g)'(t) = \underbrace{\pdv{f}{x_1}()}_{C^l} \underbrace{(t, u(t))}_{C^{l+1}} + \underbrace{\pdv{f}{x_2}()}_{C^l}\underbrace{(t, u(t))}_{C^{l+1}} \underbrace{u'(t)}_{C^l}
              \end{equation}
    \end{description}
\end{proof}

\section{Autonomous ODEs}

We say an ODE is autonomous if it doesn't depend on time:
\begin{equation}
    u' = f(u)
\end{equation}
with $f: \R \to \R$ of class $C'$.
This means that the solution does not depend on $t$, but just on $u$ itself.

If $u$ is a solution to an autonomous ODE and $v(t) = u(t + \delta)$ with $\delta \in \R$ we can expect $v$ to be a solution as well.
This property is called \emph{translation invariance}.

We can easily see that autonomous ODEs have some constant solutions:
\begin{equation}
    u(t) = c \iff f(c) = 0
\end{equation}
these solutions are called \emph{equilibrium points} and are the zeroes of $f$.

If $f$ is differentiable we can alo study $f'$: if $f'(c) < 0$ we say that the equilibrium is \emph{stable}, which means that if we have $u(t_0)$ close to $c$ at any $t \geq t_0$ where $u$ exists $u(t)$ will become closer and closer to $c$; conversely, if $f'(c) > 0$ we say the equilibrium is \emph{unstable} and $u(t)$ will move away from $c$ as time passes.

Switching future and past has the effect of also switching the equilibrium points: $v(t) = u(-t)$ solves $v' = -f(v)$.

\section{Linear higher order ODEs with constant coefficients}

Given $k \geq 1 \in \N$, linear higher order ODEs with constant coefficients take the form
\begin{equation}
    \label{eq:higher-order-linear-const-coeff-ode}
    a_k \dv[k]{u(t)}{t} + a_{k-1}\dv[k-1]{u(t)}{t} + \dots a_1 u'(t) + a_0 u(t) = 0
\end{equation}
we will assume they are homogeneous for simplicity.
We assume that $a_k \neq 0$, therefore we can divide everything by $a_k$ and therefore assume that $a_k = 1$.

Equivalently we can rewrite as a system as follows:
\begin{equation}
    \begin{cases}
        u_0' = u_1         \\
        \vdots             \\
        u_{k_2}' = u_{k-1} \\
        u_{k-1}' = - \sum_{j = 0}^{k-1} a_j u_j
    \end{cases}
\end{equation}
Since the last equation is a lipschitz function of $u_0, \dots, u_{k_1}$, given the initial conditions, the maximal solution is always defined and is unique.

\begin{proposition}{}{}
    The set $V$ of solutions $u: \R \to \R$ of \cref{eq:higher-order-linear-const-coeff-ode} is a vector space of dimension $k$.
\end{proposition}

\begin{proof}
    Operations are the usual ones (sum of functions, multiplication of a function by a scalar).
    $V$ is a vector space because \cref{eq:higher-order-linear-const-coeff-ode} is linear and homogeneous.

    We consider a map $L: \R^k \to V$ that given, $\vec \lambda (\lambda_0, \lambda_1, \dots, \lambda_{k_1}) \in \R^k$, $L(\vec \lambda)$ is the unique solution for the initial conditions $\vec \lambda$.
    We claim $L$ is linear, injective and bijective.

    Linearity is easy, given $\vec \lambda, \vec \mu \in \R^k$ and consider $w = u + v = L(\vec \lambda) + L(\vec \mu)$.
    $w$ is also a solution since both $u$ and $v$ are and the initial conditions of $w$ are $\lambda_i + \mu_i$.
    Therefore $L(\lambda + \mu) = w$ by uniqueness of the solution.

    To prove injectivity we will take $u = L(\vec \lambda) = v = L (\vec \mu)$ which means that $u(0) = v(0) \implies \lambda_0 = \mu_0$ and so on for all the derivatives.

    Surjectivity is proven by taking $u \in V$ and setting $\lambda_0 = u(0)$, $\lambda_1 = u'(0)$ and so on, giving us $u = L(\vec \lambda)$.
\end{proof}

\begin{remark}{}{}
    Everything we did works also in the case of complex valued solutions.

    The set of solutions $\dim_\C V_\C = k$ and $\dim_\R V_\C = 2k$.

    Note that if the initial conditions are real-valued, the solution will be real-valued as well.
\end{remark}

Solutions are of the form $e^{\gamma t}$ where $\gamma$ si a root of the polynomial $p$
\begin{equation}
    \gamma^k + a_{k-1} \gamma^{k-1} + \dots a_1 \gamma + a_0 = 0
\end{equation}
Moreover, if $p$ has $k$ distinct roots $\gamma_1, \dots, \gamma_k$, then $\{ e^{\gamma_1 t}, \dots, e^{\gamma_k t} \}$ is a basis of $V_\C$.

\end{document}
