\documentclass[12pt]{extarticle}

\usepackage{preamble}

\title{Mathematical Analysis 2 Notes, Partial 2}
\date{Semester 2, 2023/2024}

\setlength{\headheight}{15pt} % ??? we do what fancyhdr tells us to do

\renewcommand{\vec}[1]{\underbar{\ensuremath{#1}}}


\begin{document}

\maketitle
\tableofcontents
\clearpage

\section{Functions from \texorpdfstring{$\R^d$}{Rd} to \texorpdfstring{$\R^p$}{Rp}}

\subsection{Continuity}

\begin{definition}[continuity]
    A function $f: D \subseteq \R^d \to \R^p$ is continuous at $\vec{x}_0$ if

    $$
        \forall \varepsilon > 0,
        \quad
        \exists \delta > 0,
        \quad
        \forall \vec{x} \in D,
        \quad
        \underbrace{\norm{\vec{x} - \vec{x}_0}}_{\text{norm in } \R^d} < \delta
        \implies
        \underbrace{\norm{f(\vec{x}) - f(\vec{x}_0)}}_{\text{norm in } \R^p} < \varepsilon
    $$
\end{definition}

\begin{proposition}[sequential characterization]
    A function $f: D \subseteq \R^d \to \R^p$ is continuous at $\vec{x}_0$ iff for every sequence $\vec{x}_n \in D$ which converges to $\vec{x}_0$, the sequence $f(\vec{x}_n)$ converges to $f(\vec{x}_0)$.
\end{proposition}

\begin{proposition}[composition of continuous functions]
    Let $f: D \subseteq \R^d \to \R^p$ and $g: U \subseteq \R^p \to \R^q$ be continuous at $\vec{x}_0$ and $f(\vec{x}_0)$ respectively such that if $\vec{x} \in D$ then $f(\vec{x}) \in U$ and $g \circ f$ is well defined.
    Then $g \circ f$ is continuous at $\vec{x}_0$.
\end{proposition}

\begin{proof}
    Informally we can say

    \begin{align*}
        g(f(\lim \vec{x}_n)) = g(\lim f(\vec{x}_n)) = \lim g(f(\vec{x}_n))
    \end{align*}

    For a more formal proof you can see the proof we did for parametric curves in the first partial.
\end{proof}

\begin{proposition}[componentwise continuity]
    Let $f: D \subseteq \R^d \to \R^p$ and write

    $$
        f = \begin{pmatrix}
            f_1    \\
            f_2    \\
            \vdots \\
            f_p
        \end{pmatrix}
    $$

    where each $f_j : D \subseteq \R^d \to \R$.

    Then $f$ is continuous at $\vec{x}_0$ iff each $f_j$ is continuous at $\vec{x}_0$.
\end{proposition}

\begin{remark}
    This is more or less the definition of continuity for parametric curves.
\end{remark}

\begin{proof}
    \skiplineafterproof
    \begin{description}
        \item[$\implies$] ($f$ continuous at $\vec{x}_0 \implies \forall j, \enspace f_j$ continuous at $\vec{x}_0$)

            Fix $j$ and let $\left(\vec{y}_n\right)$ be a sequence that converges to $\vec{x}_0$.
            Since $f$ is continuous $f(\vec{y}_n) = \left(f_1(\vec{y}_n), f_2(\vec{y}_n), \ldots, f_p(\vec{y}_n)\right)$ converges to $f(\vec{x}_0) = \left(f_1(\vec{x}_0), f_2(\vec{x}_0), \ldots, f_p(\vec{x}_0)\right)$.

            In particular, the sequence $\left(f_j(\vec{y}_n)\right)$ converges to $f_j(\vec{x}_0)$.

        \item[$\impliedby$] ($\forall j, \enspace f_j$ continuous at $\vec{x}_0 \implies f$ continuous at $\vec{x}_0$)

            Let $\left(\vec{y}_n\right)$ be a sequence that converges to $\vec{x}_0$. Then, since $f_j$ is continuous, for every $j$, the sequence $f_j(\vec{y}_n)$ converges to $f_j(\vec{x}_0)$.

            So the sequence $f(\vec{y}_n) = \left(f_1(\vec{y}_n), \ldots, f_p(\vec{y}_n)\right)$ converges to $f(\vec{x}_0) = \left(f_1(\vec{x}_0), \ldots, f_p(\vec{x}_0)\right)$.
    \end{description}
\end{proof}

\subsection{Differentiability}

\begin{definition}[partial derivatives]
    Let $f: D \subseteq \R^d \to \R^p$ and $\vec{x} = \begin{pmatrix} x_1 \\ x_2 \\ \vdots \\ x_d \end{pmatrix} \in D$.

    We define, if exists,

    $$
        \pdv{f_j}{x_i}()(\vec{x}) = \lim_{h \to 0} \frac{f_j(x_1, \ldots, x_{i-1}, x_i + h, x_{i+1}, \ldots, x_d) - f_j(x_1, \ldots, x_d)}{h} \in \R
    $$

    for all $i = 1, 2, \ldots, d$ and $j = 1, 2, \ldots, p$.

    Moreover we define $\pdv{f}{x_i}()(\vec{x})$, if exists, as

    $$
        \pdv{f}{x_i}()(\vec{x}) = \begin{pmatrix}
            \pdv{f_1}{x_i}()(\vec{x}) \\
            \pdv{f_2}{x_i}()(\vec{x}) \\
            \vdots                    \\
            \pdv{f_p}{x_i}()(\vec{x})
        \end{pmatrix} \in \R^p
    $$
\end{definition}

\begin{definition}[Jacobian matrix]
    Let $f: D \subseteq \R^d \to \R^p$ and $\vec{x} \in D$ such that all the partial derivatives exist at $\vec{x}$.
    Then the Jacobian matrix of $f$ at $\vec{x}$, written as $J_f(\vec{x})$, is the $p \times d$ matrix such that

    $$
        J_f(\vec{x})_{ij} = \pdv{f_j}{x_i}()(\vec{x})
    $$

    That is

    $$
        J_f(\vec{x}) = \begin{pmatrix}
            \pdv{f_1}{x_1}()(\vec{x}) & \pdv{f_1}{x_2}()(\vec{x}) & \cdots & \pdv{f_1}{x_d}()(\vec{x}) \\
            \pdv{f_2}{x_1}()(\vec{x}) & \pdv{f_2}{x_2}()(\vec{x}) & \cdots & \pdv{f_2}{x_d}()(\vec{x}) \\
            \vdots                    & \vdots                    & \ddots & \vdots                    \\
            \pdv{f_p}{x_1}()(\vec{x}) & \pdv{f_p}{x_2}()(\vec{x}) & \cdots & \pdv{f_p}{x_d}()(\vec{x})
        \end{pmatrix}
    $$

    Note that the $i$-th column of the matrix is $\pdv{f}{x_i}()(\vec{x}) \in \R^p$.
\end{definition}

\begin{example}[change of variables to polar coordinates]
    Let $f: \R^2 \to \R^2$ be

    $$
        f(x, y) = \begin{pmatrix}
            x \cos y \\
            x \sin y
        \end{pmatrix}
    $$

    that is $f_1(x, y) = x \cos y$ and $f_2(x, y) = x \sin y$.

    Then the Jacobian matrix of $f$ is

    $$
        J_f(x, y) = \begin{pmatrix}
            \pdv{f_1}{x}() & \pdv{f_1}{y}() \\
            \pdv{f_2}{x}() & \pdv{f_2}{y}()
        \end{pmatrix} = \begin{pmatrix}
            \cos y & -x \sin y \\
            \sin y & x \cos y
        \end{pmatrix}
    $$
\end{example}

\begin{example}[Jacobian of $f: \R^2 \to \R$]
    If $f: \R^2 \to \R$ then the Jacobian matrix of $f$ is a $1 \times 2$ matrix, that is a row vector.
\end{example}

\begin{remark}
    The Jacobian is the transpose of the gradient, that is, $J_f(x, y) = \left(\nabla f(x, y)\right)'$.
\end{remark}

\begin{lemma}[linear applications]
    If $L: \R^d \to \R^p$ is a linear application then $L$ then there exists an unique matrix $M$ of size $p \times d$ such that $L(\vec{x}) = M \vec{x}$ for all $\vec{x} \in \R^d$.
    If we write $M = \begin{pmatrix} \vec{m}_1 & \vec{m}_2 & \cdots & \vec{m}_d \end{pmatrix}$ where $\vec{m}_i \in \R^p$ is the $i$-th column of $M$ for all $i = 1, 2, \ldots, d$, then

    $$
        L(\vec{x} = M\vec{x} = M \begin{pmatrix}
            x_1    \\
            x_2    \\
            \vdots \\
            x_d
        \end{pmatrix} = h_1 \vec{m}_1 + h_2 \vec{m}_2 + \cdots + h_d \vec{m}_d
    $$
\end{lemma}
\begin{remark}
    We can calculate $\vec{m}_i = L(\vec{e}_i)$ where $\vec{e}_i$ is the $i$-th vector of the canonical basis of $\R^d$ for all $i = 1, 2, \ldots, d$.
\end{remark}

We can now intuitively define what differentiability means before giving a formal definition.
Just by extending the definition of differentiability of functions from $\R^2$ to $\R$ we can define the Taylor expansion as follows:

For $d \geq 2, p = 1$ we get
$$
    f(\vec{x} + \vec{h}) = f(\vec{x}) + h_1 \pdv{f}{x_1}()(\vec{x}) + h_2 \pdv{f}{x_2}()(\vec{x}) + \cdots + h_d \pdv{f}{x_d}()(\vec{x}) + \o(\vec{h})
$$

Then to go to $p \geq 1$ we just repeat the process for each component of $f$.
That is, for each $j = 1, 2, \ldots, p$ we have

$$
    f_j(\vec{x} + \vec{h}) = f_j(\vec{x}) + \sum_{i=1}^d h_i \pdv{f_j}{x_i}()(\vec{x}) + \o(\vec{h})
$$

This is the fully-expanded first order Taylor polynomial of $f$ at $\vec{x}$.
Now we can combine everything in matrix form and we get that the derivative part (which is just a linear transformation) becomes

$$
    \sum_{i = 1}^d h_i \pdv{f}{x_i}()(\vec{x}) = M \vec{h}
$$
for some matrix $M$, but this $M$ is just the Jacobian matrix of $f$ at $\vec{x}$, so we can write

$$
    f(\vec{x} + \vec{h}) = f(\vec{x}) + J_f(\vec{x}) \vec{h} + \o(\vec{h})
$$

\begin{definition}[differentiability]
    Let $f: D \subseteq \R^d \to \R^p$, with $D$ open, and $\vec{x}_0 \in D$.
    We say that $f$ is differentiable at $\vec{x}_0$ if all the partial derivatives exist at $\vec{x}_0$
    and $\exists r > 0$ such that for all $\vec{h} \in B_c(\vec{0}, r)$ we have $\vec{x}_0 + \vec{h} \in D$ and
    $$
        f(\vec{x}_0 + \vec{h}) = f(\vec{x}_0) + J_f(\vec{x}_0) \vec{h} + \o(\vec{h})
    $$
\end{definition}

\begin{remark}
    This is an equality in $\R^p$, $\o(\vec{h})$ means that each component is $\o(h)$, and $J_f(\vec{x}_0) \vec{h}$ is the matrix-vector product.
\end{remark}

\begin{definition}[differential]
    We call differential of $f$ at $\vec{x}_0$ (if exists) the linear transformation

    $$
        \vec{h} \mapsto J_f(\vec{x}_0) \vec{h}
    $$
    we denote it as $Df_{\vec{x}_0}(\vec{h})$.
\end{definition}

\begin{proposition}[differentiability, second definition]
    Let $f: D \subseteq \R^d \to \R^p$, with $D$ open, and $\vec{x}_0 \in D$.
    $f$ is differentiable iff there exists a linear transformation $L: \R^d \to \R^p$ such that $\exists r > 0$ such that for all $\vec{h} \in B_c(\vec{0}, r)$ we have $\vec{x}_0 + \vec{h} \in D$ and

    $$
        f(\vec{x}_0 + \vec{h}) = f(\vec{x}_0) + L(\vec{h}) + \o(\vec{h})
    $$
    in this case necessarily $L = Df_{\vec{x}_0}$.
\end{proposition}

\begin{proof}
    \skiplineafterproof
    \begin{description}
        \item[$\implies$] (differentiability $\implies$ linear transformation)
            This is the easy part, just take $L = Df_{\vec{x}_0}$.
        \item[$\impliedby$] (linear transformation $\implies$ differentiability)
            Let $L(\vec{h}) = M \vec{h}$ for some matrix $M$.
            We saw before that the column $m_i = L(\vec{e}_i)$, then let $\vec{h} = h \vec{e}_i$ and by linearity we have

            $$
                f(\vec{x}_0 + h \vec{e}_i) = f(\vec{x}_0) + h L(\vec{e}_i) + \o(h)
            $$
            dividing by $h$ we get

            $$
                \frac{f(\vec{x}_0 + h \vec{e}_i) - f(\vec{x}_0)}{h} = L(\vec{e}_i) + \frac{\o(h)}{h}
            $$
            and taking the limit as $h \to 0$ we get that $\pdv{f}{x_i}()(\vec{x}_0)$ exists and is equal to $L(\vec{e}_i)$.
            Then the matrix representing $L$ has columns that are the partial derivatives of $f$ at $\vec{x}_0$, that is $M = J_f(\vec{x}_0)$.
    \end{description}
\end{proof}

\begin{proposition}[differentiability implies continuity]
    If $f: D \subseteq \R^d \to \R^p$ is differentiable at $\vec{x}_0$ then $f$ is continuous at $\vec{x}_0$.
\end{proposition}

\begin{theorem}[sufficient condition for differentiability]
    Let $f: D \subseteq \R^d \to \R^p$, $D$ open. If $\pdv{f_j}{x_i}()$ exist and are continuous in $D$ for all $i = 1, 2, \ldots, d$ and $j = 1, 2, \ldots, p$ then $f$ is differentiable in $D$.
    In this case we say that $f$ is of class $C^1$.
\end{theorem}
\begin{remark}
    This is a sufficient condition, but not a necessary one.
\end{remark}

\begin{proof}
    We will not prove these results because the proof is quite long and we already did it in the case of functions from $\R^2$ to $\R$.
\end{proof}

\begin{example}
    Consider a function $f: \R \to \R^p$ (a parametric curve) defined as

    $$
        f : t \mapsto \begin{pmatrix}
            f_1(t) \\
            f_2(t) \\
            \vdots \\
            f_p(t)
        \end{pmatrix}
    $$
    Then
    $$
        J_f(t) = \begin{pmatrix}
            f'_1(t) \\
            f'_2(t) \\
            \vdots  \\
            f'_p(t)
        \end{pmatrix} = f'(t)
    $$
    and the Taylor expansion of $f$ is
    $$
        f(t + h) = f(t) + h f'(t) + \o(h)
    $$
    which looks a lot like what we saw in Analysis 1.
\end{example}
\begin{remark}
    Let $\vec{a}, \vec{b} \in \R^d$, then $\vec{a} \cdot \vec{b} = \vec{a}^T \vec{b}$, where the first one is a dot product and the second one is a matrix multiplication.
\end{remark}

\begin{example}[tangent plane to a parametric surface]
    Let $f: \R^2 \to \R^3$ be a parametric surface, then its image is
    $$
        f = \{ f(u, v) : \R^2 \}
    $$

    The tangent plane is the plane passing through $f(u_0, v_0)$ and normal to
    $$
        \pdv{f}{u}() \times \pdv{f}{v}()
    $$
    if the cross product is not zero.

    The equation of the plane is

    $$
        (\vec{x} - f(u_0, v_0)) \cdot \left( \pdv{f}{u}() \times \pdv{f}{v}() \right) = 0
    $$

\end{example}

\end{document}