\documentclass[12pt]{extarticle}

\setlength{\headheight}{16pt} % ??? we do what fancyhdr tells us to do  

\title{Advanced Programming and Optimization}
\author{Giacomo Ellero}
\date{a.y. 2024/2025}

\usepackage{preamble}
\usepackage{preamble_code}

\renewcommand{\vec}[1]{\uvec{#1}}

\begin{document}

\oldfirstpage

\section{Introduction}

This course will all be about
\begin{align}
	\max f(x), & \text{subject to}                          \\
	g_i(x)     & \leq b_i \quad \text{for } i = 1, \dots, m \\
	x          & \in \R^n
\end{align}
where $f, g_1, \dots, g_m: \R^n \to \R$ are linear functions and $b_1, \dots, b_m \in \R$.
This is the standard \emph{linear programming} problem and can be solved in polynomial time.

However we are also going to be looking at problems where $x \in \Z^n$ (integers).
This is a NP hard problem.

The last part of the course we will have the case where $f, g_1, \dots, g_m$ are not linear
but they are \emph{convex} instead. This also is a hard problem.

\section{Linear programming}

Recall that even though we have stated the problem with less-than-or-equal, we can just multiply
the constraints by $-1$ to flip the inequality if needed.
Similarly we can multiply $f$ by $-1$ to change $\min$ to $\max$ or viceversa.

In linear programming our constraint make a polygon in an $n$-dimensional space.
If we let $\vec a$ be the vector of the coefficients of $f$ the problem reduces to finding the
point which is further away from the origin perpendicular to $\vec a$.

It is not always possible to find an unique solution to a linear programming problem: sometimes
the problem has multiple solutions, it is undecidable, or unbounded.

\subsubsection{Examples}

\begin{example}{Flow}{}
	Recall that the flow problem we saw in CS2 can be written as a linear programming problem:
	each edge has a constraint of $-c_i \leq e_i \leq c_i$ (where $c_i$ is the capacity of each edge)
	and each node has an additional constraint of flow preservation.
	We can maximize the output from the source or the input to the destination.
\end{example}

\begin{example}{Icecream production}{}
	Each month $i$ has a certain icecream demand $d_i$.
	Changing the prodution of icecream has a cost $a_1$ and the cost of storing icecream we pay $a_2$.
	What is the optimal amount of icecream to be produced?
\end{example}

\begin{proof}[Solution]
	For each month $i$ we want
	\begin{equation}
		x_i + s_{i-1} - s_i = d_i
	\end{equation}
	where $x_i$ is the production of icecream and $s_i$ is the amount we have in storage.

	Our objective will be
	\begin{equation}
		\min a_1 \sum \abs{x_i - x_{i-1}} + a_2 \sum s_i
	\end{equation}
	but this is not a linear funtion!

	Instead we introduce two new variables $y_i$ which represents the increase in the prodution
	and $z_i$ which represents the decrease in the production.
	We can now add the constraint
	\begin{align}
		x_i - {x_{i-1}} & = y_i - z_i \\
		y_i             & \geq 0      \\
		z_i             & \geq 0
	\end{align}
	and our objective becomes
	\begin{equation}
		\min a_1 \sum y_i + a_1 \sum z_i + a_2 \sum s_i
	\end{equation}
	which is indeed linear.
	Moreover, we notice that an optimal solution either has $y_i = 0$ or $z_i = 0$,
	hence the problem is equivalent to the first one.
\end{proof}

\end{document}
