\documentclass[12pt]{extarticle}

\title{Decision Theory - Problem Set 1}
\author{Czapran, Ellero, Porta, Tan}
\date{a.y. 2024/2025}

\usepackage{parskip} % for no indentation
\usepackage[a4paper,margin=1.5cm]{geometry} % to adjust margins
\usepackage{bookmark} % for pdf bookmarks

\usepackage{enumitem} % for custom lists

% We want numbers on subsections too
\setcounter{secnumdepth}{3}

\usepackage{amssymb}   % for varnothing and other weirds symbols
\usepackage{amsmath}   % basically everything
\usepackage{amsthm}    % for proof environment
\usepackage{mathtools} % for underbrace, arrows, and a lot of other things
\usepackage{mathrsfs}  % for "mathscr"
\usepackage{bm}        % for bold math symbols
\usepackage{physics}   % for derivatives and lots of operators

\numberwithin{table}{section}
\numberwithin{figure}{section}

\usepackage{cancel}    % for canceling terms
\renewcommand{\CancelColor}{\color{red}}

% Operators
\newcommand{\gt}{>}
\newcommand{\lt}{<}
\newcommand{\indep}{\perp \!\!\! \perp}

% Sets
\newcommand{\C}{\mathbb{C}}
\newcommand{\R}{\mathbb{R}}
\newcommand{\N}{\mathbb{N}}
\newcommand{\Q}{\mathbb{Q}}
\newcommand{\Z}{\mathbb{Z}}

\numberwithin{equation}{section}

\begin{document}

\maketitle

\section*{Problem 1}
\stepcounter{section}

\begin{enumerate}[label=\alph*.)]
	\item In this context $\succ$ means that
	      \begin{equation}
		      \max \{ x_1, x_2 \} \geq \max \{ y_1, y_2 \} \land  x_1 + x_2 > y_1 + y_2
	      \end{equation}
	      or
	      \begin{equation}
		      \max \{ x_1, x_2 \} > \max \{ y_1, y_2 \} \land  x_1 + x_2 \geq y_1 + y_2
	      \end{equation}
	      while $\sim$ means that
	      \begin{equation}
		      \max \{ x_1, x_2 \} = \max \{ y_1, y_2 \} \land  x_1 + x_2 = y_1 + y_2
	      \end{equation}

	\item This preference is \textbf{\emph{not} complete}.

	      Completeness would imply that $\forall x, y$ either $x \succsim y$ or $y \succsim x$ or both.

	      \begin{proof}
		      We can use as counterexample $x = (1, 9)$ and $y = (7,8)$: we have that
		      \begin{gather}
			      \max\{1, 9\} = 9 \geq \max \{7,8\} = 8\\
			      1+9 = 10 \leq 7+8 = 15
		      \end{gather}
		      which means neither $x \succsim y$ or $y \succsim x$.
	      \end{proof}

	      This preference is \textbf{transitive}.
	      Transitivity means given $x, y, z$, if $x \succsim y$ and $y \succsim z$, then $x \succsim z$.
	      \begin{proof}
		      We have that both conditions for preference are transitive, indeed:
		      \begin{itemize}
			      \item $(\max {x_1, x_2} \geq \max{y_1, y_2}) \land (\max{y_1, y_2} \geq \max{z_1, z_2}) \implies \max{x_1, x_2}  \geq \max{z_1, z_2}$.
			      \item $(x_1 + x_2 \geq y_1 + y_2) \land (y_1 + y_2 \geq z_1 + z_2) \implies x_1 + x_2 \geq z_1 + z_2$.
		      \end{itemize}
		      Therefore $\succsim$ is transitive.
	      \end{proof}


	\item This preference is \textbf{strictly monotone}.

	      Strict monotonicity implies that if $x > y \implies x \succ y$.
	      In this scenario we have
	      \begin{equation}
		      x > y \iff (x_1 > y_1 \land x_2 \geq y_2) \lor (x_1 \geq y_1 \land x_2 > y_2)
	      \end{equation}

	      \begin{proof}
		      Clearly $x > y \implies x_1 + x_2 > y_1 + y_2$, hence we are left to prove that $\max\{ x_1, x_2 \} \geq \max\{ y_1, y_2 \}$:
		      \begin{itemize}
			      \item If $x_1 > y_1 \land x_2 > y_2$ then $\max\{ x_1, x_2 \} > \max\{ y_1, y_2 \}$
			      \item If $x_1 \geq y_1\land x_2 > y_2$ (or viceversa) we will have
			            $\max\{ x_1, x_2 \} \geq \max\{ y_1, y_2 \}$
		      \end{itemize}
	      \end{proof}

	\item This preference is \textbf{\emph{not} convex} and \textbf{\emph{not} affine}.

	      Consider $x, y$ distinct such that $x \sim y$. If $\succsim$ is convex it holds
	      \begin{equation}
		      \alpha x + (1-\alpha)y \succsim x \quad \forall \alpha \in (0, 1)
	      \end{equation}
	      while if $\succsim$ is affine it holds
	      \begin{equation}
		      \alpha x + (1-\alpha)y \sim x \quad \forall \alpha \in (0, 1)
	      \end{equation}

	      \begin{proof}
		      We can use as counterexample $x = (1, 3), y = (3, 1), \alpha = 0.5$.
		      Since $\max x = 3 = \max y$ and $x_1 + x_2 = 4 = y_1 + y_2$ we have that $x \sim y$.
		      But then
		      \begin{gather}
			      0.5 \begin{bmatrix}1 \\ 3\end{bmatrix} + 0.5 \begin{bmatrix}3 \\ 1\end{bmatrix} \succsim \begin{bmatrix}1 \\ 3\end{bmatrix} \\
			      \begin{bmatrix}2 \\ 2\end{bmatrix} \not \succsim \begin{bmatrix}1 \\ 3\end{bmatrix}
		      \end{gather}
		      since $\max\{2,2\} = 2 < \max\{ 1, 3 \} = 3$.

		      This means that $\succsim$ is neither convex nor affine.
	      \end{proof}
\end{enumerate}

\section*{Problem 2}
\stepcounter{section}

$\succsim$ on $\mathbb{R}^2$ is said to be a weak order if it satisfies \textbf{completeness} and \textbf{transitivity}.

\begin{description}
	\item[Completeness]
	      For every $x, y \in \mathbb{R}^2$, there are three possibilities:
	      \begin{itemize}
		      \item $x_1 > y_1$, which implies that $x \succsim y$
		      \item $y_1 > x_1$, which implies that $y \succsim x$
		      \item $x_1 = y_1$. In this case, either:
		            \begin{itemize}
			            \item $x_2 \ge y_2$, which implies $x \succsim y$
			            \item $y_2 \ge x_2$, which implies $y \succsim x$
			            \item Both, which implies both $x \succsim y$ and $y \succsim x$
		            \end{itemize}
	      \end{itemize}
	      Since we either have $x \succsim y$, $y \succsim x$, or both, $\succsim$ is \textbf{complete}.
	\item[Transitivity]
	      For every $x, y, z \in \mathbb{R}^2$, if $x \succsim y$ and $y \succsim z$, then
	      \begin{equation}
		      x_1 > y_1 \lor (x_1 = y_1 \land x_2 \ge y_2)
	      \end{equation}
	      and
	      \begin{equation}
		      y_1 > z_1 \lor (y_1 = z_1 \land y_2 \ge z_2)
	      \end{equation}

	      In the case of $x_1 > y_1$, we either have:
	      \begin{itemize}
		      \item $x_1 > y_1 > z_1$
		      \item $x_1 > y_1 = z_1$ and $y_2 = z_2$
	      \end{itemize}

	      In both cases, we have that $x_1 > z_1$, so $x \succsim z$.
	      In the other case of $x_1 = y_1$ and $x_2 \ge y_2$, we either have:
	      \begin{itemize}
		      \item $x_1 = y_1 > z_1$
		      \item $x_1 = y_1 = z_1$ and $x_2 \ge y_2 \ge z_2$
	      \end{itemize}

	      So either
	      \begin{equation}
		      x_1 > z_1 \implies x \succsim z
	      \end{equation}
	      or
	      \begin{equation}
		      x_1 = z_1 \land x_2 \ge z_2 \implies x \succsim z
	      \end{equation}

	      Since for every $x, y, z \in \mathbb{R}^2$, $x \succsim y$ and $y \succsim z$ implies that $x \succsim z$, then $\succsim$ is \textbf{transitive}.
	\item[Antisymmetry]
	      Take $x, y \in \mathbb{R}^2$. Let $x \succsim y$.
	      Then either
	      \begin{enumerate}
		      \item $x_1 > y_1$
		      \item $x_1 = y_1$ and $x_2 \ge y_2$
	      \end{enumerate}
	      Also let $y \succsim x$. Then either
	      \begin{enumerate}
		      \setcounter{enumi}{2}
		      \item $y_1 > x_1$
		      \item $y_1 = x_1$ and $y_2 \ge x_2$
	      \end{enumerate}
	      Assume by contradiction that it is case (1).
	      Then neither $y_1 > x_1$ nor $y_1 = x_1$, which contradicts that $y \succsim x$. Therefore, it is case (2).
	      Since $x_1 = y_1$, (3) is not the case, leaving (4).
	      From (2) and (4), we have that $x_1 = y_1$, $x_2 \ge y_2$, and $y_2 \ge x_2$, which implies that $x_2 = y_2$, thus $x = y$. Since for any $x, y \in \mathbb{R}^2$, $x \succsim y$ and $y \succsim x$ implies that $x = y$, $\succsim$ is \textbf{antisymmetric}.
\end{description}

\section*{Problem 3}
\stepcounter{section}

\begin{enumerate}[label=\alph*.)]
	\item The statement is true.
	      \begin{proof}
		      By contradiction, assume $x \notin C(X)$, this would imply that $\exists y \in X$ such that $u(x) + 1 < u(y)$.
		      Since $X \subseteq Y$ and $y \in X$ we have that $y \in Y$.
		      But this would mean that also in $Y$ $u(x) + 1 < u(y)$ would hold implying that $x \notin C(Y)$ and we have reached a contradiction.
	      \end{proof}
	\item The statement is false.
	      \begin{proof}
		      We choose as counterexample $\mathcal D = \mathcal P(\N)$, $X = \{0,1,2,3\}$, $Y = \{0,1,2,3, 4, 5\}$ and $u(x) = x/2$.
		      Consider $x = 2 \in X, y = 3 \in X, z = 5 \in Y$.

		      We have that $x \in C(X)$ since $u(x) + 1 = 2 > \max_{x \in X} u(x) = 1.5$
		      Similarly $y \in C(X)$ because $u(y) + 1 = 2.5 > \max_{x \in X} u(x) = 1.5$.

		      We also have that $y \in C(Y)$ since $u(y) + 1 = 2.5 \geq \max_{x \in Y} u(x) = 2.5$, but $x \notin C(Y)$ because $u(x) + 1 = 2 < u(z) = 2.5$.
	      \end{proof}
\end{enumerate}

\section*{Problem 4}
\stepcounter{section}

Let $\succsim$ be a \textbf{weak order}, i.e. transitive and complete. We want to prove that for $X \in \mathcal{D}$, $X \succeq Y$ for every $Y \in \mathcal{D}$ \textit{if and only if} there exists $x \in \sigma(X)$ such that $x \succsim z$ for every $z \in \mathbf{X}$.
\begin{equation}
	\sigma(X) = \{\hat x \in X\colon \forall x \in X, \hat x \succsim x\}
\end{equation}

\begin{description}
	\item[$\implies$]
	      With $X \in \mathcal{D}$, suppose there exists $x \in \sigma(X)$ such that $x \succsim z$ for every $z \in \mathbf{X}$.
	      By contradiction, suppose $Y \succeq X$ for $Y \in \mathcal{D}$.
	      By definition,
	      \begin{equation}
		      \forall x \in X, \exists y \in Y\colon\quad y \succsim x
	      \end{equation}
	      This contradicts that $x \succsim z$, which should hold for every $z \in \mathbf{X}$, so $X \succeq Y$ due to completeness.
	\item[$\impliedby$]
	      Let $X \succeq Y$.
	      Then
	      \begin{equation}
		      \forall y \in Y, \exists x \in X\colon\quad x \succsim y
	      \end{equation}
	      Suppose that $z \succsim x$ for $z \in \mathbf{X}$.
	      This contradicts that
	      \begin{equation}
		      x \succsim y \qquad \forall y \in Y
	      \end{equation}
	      so we have $x \succsim z$. Since $\succsim$ is complete, this means that
	      \begin{equation}
		      \exists x \in \sigma(X) \colon\qquad x \succsim z \quad z \in \mathbf{X}
	      \end{equation}
	      because it is always preferred, thus optimal.
\end{description}

\end{document}