\documentclass[17pt]{extarticle}
\title{Linear Algebra Notes}
\author{Giacomo Ellero}

\usepackage{amsfonts}
\usepackage{amsthm}
\usepackage{amssymb}
\usepackage{amsmath}
\usepackage{mathtools}
\usepackage{commath}
\usepackage{dirtytalk}
\usepackage{parskip}
\usepackage[a4paper,margin=1.5cm]{geometry}

\newcommand{\C}{\mathbb{C}}
\newcommand{\R}{\mathbb{R}}
\newcommand{\F}{\mathbb{F}}
\newcommand{\citeladr}[1]{
    \begin{quotation}
        This references \say{Linear Algebra Done Right}, #1
    \end{quotation}
}
\newcommand{\innerprod}[1]{
    \langle #1 \rangle
}
\renewcommand{\Re}{\operatorname{Re}}
\renewcommand{\Im}{\operatorname{Im}}

\begin{document}

\maketitle
\tableofcontents
\clearpage

\section{Inner product spaces}

\citeladr{Chapter 6}

\subsection{Inner product and norms}

\citeladr{Chapter 6.A}

\subsubsection{Definition of complex conjugate}

Let $z \in \C$, then

$$
    \overline z = \overline{a + bi} = a - bi
$$

\subsubsection{Definition of complex absolute value}

Let $z \in \C$, then

$$
    |z| = \sqrt{a^2 + b^2}
$$

and

$$
    |z|^2 = z \overline z
$$


\subsubsection{Definition of norm}

\citeladr{6.8}

Let $v \in V$, then

$$
    \norm v = \sqrt{\langle v, v \rangle}
$$

\subsubsection{Definition of orthogonal}

\citeladr{6.11}

Two vectors $u, v$ are orthogonal if

$$
    \langle v, u \rangle = 0
$$

This is just a fancy way to say they are perpendicular since, for example, in $\R^2$, $\langle v, u \rangle = \norm v \norm u \cos \theta$.

\subsubsection{Pythagorean theorem}

\citeladr{6.13}

Let $v, u \in V$ be orthogonal, then

$$
    \norm{u+v}^2 = \norm u^2 + \norm v^2
$$

\subsubsection{Orthogonal Decomposition}

\citeladr{6.14}

Let $v, u \in V$ and $w = u - cv$ orthogonal to $v$.

We want to find $c \in \F$ and such that

$$
    u = cv + w
$$

To solve this problem we set

$$
    c = \frac{\langle u, v \rangle}{\norm v^2},
    \quad
    w = u - \frac{\langle u, v \rangle}{\norm v^2} v
$$

In this way we have that $u = cv + w$ and $\langle v, w \rangle = 0$

\subsubsection{Cauchy-Schwarz Inequality}

\citeladr{6.15}

Let $u, v \in V$. Then

$$
    |\langle u, v \rangle \le \norm u \norm v
$$

\begin{proof}
    Consider the orthogonal decomposition $u = cv + w$.
    Since $cv$ is orthogonal to $w$, by the Pythagorean Theorem we have that

    \begin{align*}
        \norm u ^2 & = \norm {c v}^2 + \norm w^2                                                 \\
                   & = \norm {\frac{\langle u, v \rangle}{\norm v^2} v}^2 + \norm w^2            \\
                   & = \left(\frac{\langle u, v \rangle}{\norm v^2} \norm v\right)^2 + \norm w^2 \\
                   & = \frac{|\langle u, v \rangle |^2}{\norm v^2} + \norm w^2                   \\
                   & \ge \frac{|\langle u, v \rangle |^2}{\norm v^2}                             \\
    \end{align*}

    Now we take square roots on both sides and rearrange the terms to get our result.

\end{proof}

\subsubsection{Triangle Inequality}

\citeladr{6.18}

Let $u, v \in V$. Then

$$
    \norm{u+v} \le \norm u + \norm v
$$

\begin{proof}
    \begin{align*}
        \norm{u+v}^2 & = \innerprod{u+v, u+v}                                                                 \\
                     & = \innerprod{u, u} + \innerprod{v,v} + \innerprod{u, v} + \innerprod{v, u}             \\
                     & = \innerprod{u, u} + \innerprod{v,v} + \innerprod{u, v} + \overline {\innerprod{u, v}} \\
                     & = \norm u^2 + \norm v^2 + 2 \Re \innerprod{u, v}                                       \\
                     & \le \norm u^2 + \norm v^2 + 2|\innerprod{u, v}|                                        \\
                     & \le \norm u^2 + \norm v^2 + 2 \norm u \norm v                                          \\
                     & = (\norm u + \norm v)^2
    \end{align*}
\end{proof}

\subsubsection{Parallelogram Equality}

\citeladr{6.22}

Let $u, v \in V$. Then

$$
    \norm {u+v}^2 + \norm{u-v}^2 = 2 (\norm u^2 + \norm v^2)
$$

This is quite easy to prove just with the definition of norm and inner product.

\clearpage

\subsection{Orthonormal Bases}

\citeladr{Chapter 6.B}

\subsubsection{Definition of Orthonormal List}

\citeladr{6.23}

A list of vectors is orthonormal if
each vector has norm 1 and
is orthogonal to the other vectors in the list.

This means that in an orthonormal list $x_1, \dots, x_n$ of vectors in $V$

$$
    \innerprod{x_i, x_j} = \begin{cases*}
        1  \text{ if } j = k, \\
        0  \text{ if } j \ne k
    \end{cases*}
$$

\end{document}
