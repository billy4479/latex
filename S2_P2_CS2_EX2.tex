\documentclass[12pt]{extarticle}

\usepackage{preamble}
\usepackage{bytefield}
\usepackage{preamble_code}

\title{Computer Science 2, Exercise 2}
\date{Semester 2, 2023/2024}

\setlength{\headheight}{15pt} % ??? we do what fancyhdr tells us to do
\renewcommand{\headrulewidth}{0pt}

\begin{document}

\section*{Problem 2}

\subsubsection*{Problem statement}

Solve the knapsack problem in $\O(nV)$ time, where $V = v_1 + v_2 + \ldots + v_n$ denotes the sum of item values.
Note that the running time cannot depend on the item weights nor the knapsack capacity.

\subsection*{Solution}

\subsubsection*{The idea}

The idea behind the solution is the same as the one that was presented in class by the professor, but with some slight modifications.

In this case $M$ is a matrix of size $V \times n$ where $n$ is the number of items and $V$ is the sum of all item values.
Each cell $M[i][j]$ will store the minimum weight needed to achieve a value of at least $i$ using the first $j$ items.

To fill each cell $M[i][j]$, we have two options: either we take the $j$-th item or we don't.
\begin{itemize}
    \item
          If we don't take the $j$-th item, then the minimum weight needed to achieve a value of at least $i$ using the first $j$ items is the same as the minimum weight needed to achieve a value of at least $i$ using the first $j - 1$ items.
          That is $M[i][j] = M[i][j - 1]$.
    \item
          If we take the $j$-th item, then we are left with the problem of achieving a value of at least $i - v_j$ using the first $j - 1$ items, but we have already solved this problem: we have the result stored in $M[i - v_j][j - 1]$.
\end{itemize}

Then we choose the minimum of these two options.

While writing this solution we have to be careful about the case in which $i < v_j$, that is, the value of the $j$-th item is greater than the value we are trying to achieve.
In this case, we choose the item only if the weight of the item is less than the minimum weight needed to achieve a value of at least $i$ using the first $j - 1$ items.

Now that we have filled the matrix $M$, we can find the solution to the problem by iterating over the rows of the matrix starting from the last row and stopping when we find a row where the minimum weight needed to achieve a value of at least $i$ is less than or equal to the maximum weight allowed.
The solution is guaranteed to be in the last column, as we are always choosing the minimum weight needed to achieve a value of at least $i$ using all the items.

It is easy to see that the time complexity of this solution is $\O(nV)$, as we are filling a matrix of size $V \times n$ and each operation takes $\O(1)$ time.

\clearpage

\subsubsection*{The code}

\begin{minted}{python}
def knapsack(items, max_weight):

    # Calculate V
    total_value = 0
    for item in items:
        total_value += item.value

    # Initialize M
    M = [[None for _ in range(len(items) + 1)] for _ in range(total_value + 1)]

    for item_index in range(len(items) + 1):
        for value in range(total_value + 1):
            if value == 0:
                # If value is 0, then we choose 0 items and the weight is 0
                M[value][item_index] = 0
            elif item_index == 0:
                # If there are no items, we cannot achieve any value
                # so we set the weight to infinity
                M[value][item_index] = float("inf")
            else:
                # If i < v_j
                if value < items[item_index - 1].value:
                    # Choose the minimum between the weight of the current item
                    # and the weight we needed without the current item
                    M[value][item_index] = min(
                        M[value][item_index - 1], items[item_index - 1].weight
                    )
                else:
                    # Choose the minimum between the weight we need
                    # without the current item and the weight we need
                    # with the current item, that is,the weight needed
                    # to achieve a value of at least i - v_j plus
                    # the weight of the current item
                    M[value][item_index] = min(
                        M[value][item_index - 1],
                        M[value - items[item_index - 1].value][item_index - 1]
                        + items[item_index - 1].weight,
                    )

    # Find the best value
    best_value = 0
    for value in range(total_value, 0, -1):
        if M[value][len(items)] <= max_weight:
            best_value = value
            break

    return M, best_value
\end{minted}

\end{document}