\documentclass[12pt]{extarticle}

\title{Decision Theory - Problem Set 1}
\author{Czapran, Ellero, Porta, Tan}
\date{a.y. 2024/2025}

\usepackage{parskip} % for no indentation
\usepackage[a4paper,margin=1.5cm]{geometry} % to adjust margins
\usepackage{bookmark} % for pdf bookmarks

\usepackage{enumitem} % for custom lists

% We want numbers on subsections too
\setcounter{secnumdepth}{3}

\usepackage{amssymb}   % for varnothing and other weirds symbols
\usepackage{amsmath}   % basically everything
\usepackage{amsthm}    % for proof environment
\usepackage{mathtools} % for underbrace, arrows, and a lot of other things
\usepackage{mathrsfs}  % for "mathscr"
\usepackage{bm}        % for bold math symbols
\usepackage{physics}   % for derivatives and lots of operators

\numberwithin{table}{section}
\numberwithin{figure}{section}

\usepackage{cancel}    % for canceling terms
\renewcommand{\CancelColor}{\color{red}}

% Operators
\newcommand{\gt}{>}
\newcommand{\lt}{<}
\newcommand{\indep}{\perp \!\!\! \perp}

% Sets
\newcommand{\C}{\mathbb{C}}
\newcommand{\R}{\mathbb{R}}
\newcommand{\N}{\mathbb{N}}
\newcommand{\Q}{\mathbb{Q}}
\newcommand{\Z}{\mathbb{Z}}

\numberwithin{equation}{section}

\begin{document}

\maketitle

\section{Exercise 1}

\begin{enumerate}[label=\alph*.)]
    \item $\succ$ means that
          \begin{equation}
              \max \{ x_1, x_2 \} \geq \max \{ y_1, y_2 \} \land  x_1 + x_2 > y_1 + y_2
          \end{equation}
          or
          \begin{equation}
              \max \{ x_1, x_2 \} > \max \{ y_1, y_2 \} \land  x_1 + x_2 \geq y_1 + y_2
          \end{equation}
          while $\sim$ means that
          \begin{equation}
              \max \{ x_1, x_2 \} = \max \{ y_1, y_2 \} \land  x_1 + x_2 = y_1 + y_2
          \end{equation}

    \item This preference is not complete.
          Completeness would imply that $\forall x, y \enspace xRy \lor yRx$ or both.
          We can use as counterexample $x = (1, 9)$ and $y = (7,8)$: we have that
          \begin{gather}
              \max\{1, 9\} = 9 \geq \max \{7,8\} = 8\\
              1+9 = 10 \leq 7+8 = 15
          \end{gather}
          which means neither $xRy$ or $yRx$.

    \item This preference is strictly monotone.
          Strict monotonicity implies that if $x > y \implies x \succ y$.
          In this scenario we have
          \begin{equation}
              x > y \iff (x_1 > y_1 \land x_2 \geq y_2) \lor (x_1 \geq y_1 \land x_2 > y_2)
          \end{equation}
          Clearly $x > y \implies x_1 + x_2 > y_1 + y_2$, hence we are left to prove that $\max\{ x_1, x_2 \} \geq \max\{ y_1, y_2 \}$:
          \begin{itemize}
              \item If $x_1 > y_1 \land x_2 > y_2$ then $\max\{ x_1, x_2 \} > \max\{ y_1, y_2 \}$
              \item If $x_1 \geq y_1\land x_2 > y_2$ (or viceversa) we will have
                    $\max\{ x_1, x_2 \} \geq \max\{ y_1, y_2 \}$
          \end{itemize}
    \item This preference is \emph{not} convex.
          Convexity implies that, for $x, y$ distinct such that $x \sim y$, it holds
          \begin{equation}
              \alpha x + (1-\alpha)y \succsim x \quad \forall \alpha \in (0, 1)
          \end{equation}

          We can use as counterexample $x = (1, 3), y = (3, 1), \alpha = 0.5$.
          Since $\max x = 3 = \max y$ and $x_1 + x_2 = 4 = y_1 + y_2$ we have that $x \sim y$.
          But then
          \begin{gather}
              0.5 \begin{bmatrix}1 \\ 3\end{bmatrix} + 0.5 \begin{bmatrix}3 \\ 1\end{bmatrix} \succsim \begin{bmatrix}1 \\ 3\end{bmatrix} \\
              \begin{bmatrix}2 \\ 2\end{bmatrix} \not \succsim \begin{bmatrix}1 \\ 3\end{bmatrix}
          \end{gather}
          since $\max\{2,2\} = 2 < \max\{ 1, 3 \} = 3$.
\end{enumerate}

\section{Exercise 2}

\section{Exercise 3}

\section{Exercise 4}

\begin{enumerate}[label=\alph*.)]
    \item The statement is true.
          \begin{proof}
              By contradiction, assume $x \notin C(X)$, this would imply that $\exists y \in X$ such that $u(x) + 1 < u(y)$.
              Since $X \subseteq Y$ and $y \in X$ we have that $y \in Y$.
              But this would mean that also in $Y$ $u(x) + 1 < u(y)$ would hold implying that $x \notin C(Y)$ and we have reached a contradiction.
          \end{proof}
    \item The statement is false.
          \begin{proof}
              We choose as counterexample $\mathcal D = \mathcal P(\N)$, $X = \{0,1,2,3\}$, $Y = \{0,1,2,3, 4, 5\}$ and $u(x) = x/2$.
              Consider $x = 2 \in X, y = 3 \in X, z = 5 \in Y$.

              We have that $x \in C(X)$ since $u(x) + 1 = 2 > \max_{x \in X} u(x) = 1.5$
              Similarly $y \in C(X)$ because $u(y) + 1 = 2.5 > \max_{x \in X} u(x) = 1.5$.

              We also have that $y \in C(Y)$ since $u(y) + 1 = 2.5 \geq \max_{x \in Y} u(x) = 2.5$, but $x \notin C(Y)$ because $u(x) + 1 = 2 < u(z) = 2.5$.
          \end{proof}
\end{enumerate}

\end{document}