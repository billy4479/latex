\documentclass[10pt]{extarticle}
\title{Probability Notes}
\author{Giacomo Ellero}

\usepackage{amsfonts}
\usepackage{amsthm}
\usepackage{amssymb}
\usepackage{amsmath}
\usepackage{mathtools}
\usepackage{commath}
\usepackage{dirtytalk}
\usepackage{parskip}
\usepackage{mathrsfs}
\usepackage[many]{tcolorbox}
\usepackage{xparse}
\usepackage[a4paper,margin=1.5cm]{geometry}
\usepackage{bookmark}

\newcommand{\C}{\mathbb{C}}
\newcommand{\R}{\mathbb{R}}
\newcommand{\F}{\mathcal{F}}
\renewcommand{\Re}{\operatorname{Re}}
\renewcommand{\Im}{\operatorname{Im}}

\newenvironment{absolutelynopagebreak}
  {\par\nobreak\vfil\penalty0\vfilneg
   \vtop\bgroup}
  {\par\xdef\tpd{\the\prevdepth}\egroup
   \prevdepth=\tpd}

\newtcolorbox{examplebox}[1]{colback=green!5!white,colframe=green!40!black,title={#1},fonttitle=\bfseries,parbox=false}
\newtcolorbox{notebox}[1]{colback=green!5!white,colframe=blue!40!black,title={Note: #1},fonttitle=\bfseries,parbox=false}

\begin{document}

\maketitle
\tableofcontents
\clearpage

\section{Class of 05/02/2024}

\subsection{Interpretations of probability}

Probability has had many interpretations over the years, here's some of them:
\begin{enumerate}
    \item \textbf{Classical interpretation}: Games of chances, all the outcomes are equaly likely.
          We can calculate probability as the number of favourable outcomes over the total.
          This only works on finite spaces with perfect simmetries.
    \item \textbf{Frequency interpretation}: This is the probability calculated in the long run by repeating the experiment infinitely many times.
          This only works if the experiment is repeatable. What if we cannot take the limit?
    \item \textbf{Subjective interpretation}: How much are people willing to bet on something to happen?
\end{enumerate}

\subsection{Sample spaces and outcomes}

In an \textbf{experiment} we call \textbf{sample space} $\Omega$ the set of all possible outcomes,
while a singular outcome is denoted as $\omega$.
$\Omega$ can be finite, countable or uncountable.

\begin{examplebox}{Example}
    \begin{itemize}
        \item Rolling a die: $\Omega = \{1, 2, 3, 4, 5, 6\}$
              \begin{itemize}
                  \item This is sample space is finite.
              \end{itemize}
        \item Tossing a coin until heads comes up: $\Omega = \{H, TH, TTH, \dots, TTT\dots\}$
              \begin{itemize}
                  \item This is sample space is not finite, but it is countable.
              \end{itemize}
        \item Dart in a circular target of radius $r$: $\Omega = \{(x, y) \in \R^2 : x^2 + y^2 \le r\}$
              \begin{itemize}
                  \item This is sample space is uncountable.
              \end{itemize}
    \end{itemize}
\end{examplebox}

\begin{notebox}{impossible $\ne$ probability 0}
    Despite some elements having probability 0, it doesn't mean they are impossible.

    \begin{itemize}
        \item In the dart example, the probability of hitting a single point is 0 since $\Omega$ is uncountable, but it's not impossible.
        \item In the coin example, the probability of getting an infinite number of tails is 0, but it's not impossible.
    \end{itemize}

    In particular note that if $\Omega$ is uncountable every singular outcome has probability 0.
\end{notebox}

\subsection{Events}

An \textbf{event} is a subset of the sample space of all the outcomes such that the event occurs.
We denote it as $A \subseteq \Omega$.

In particular:
\begin{itemize}
    \item The \textbf{empty set} $\varnothing$ is the impossible event.
    \item The \textbf{whole sample space} $\Omega$ is the certain event.
\end{itemize}

We can define the following operations on events:
\begin{itemize}
    \item \textbf{Complement}: $A^c$, the event occurs if $A$ doesn't.
    \item \textbf{Union}: $A \cup B$, the event occurs if $A$ \textit{or} $B$ occurs.
    \item \textbf{Intersection}: $A \cap B$, the event occurs if $A$ \textit{and} $B$ occur.
    \item \textbf{Difference}: $A \setminus B$, the event occurs if $A$ occurs but $B$ doesn't.
    \item \textbf{Symmetric difference}: $A \triangle B$, the event occurs if $A$ or $B$ occur but not both.
\end{itemize}

Moreover we say that $A$ \textbf{depends} of $B$ if $A \subseteq B$. This means that if $B$ occurs, then $A$ also occurs.

\begin{examplebox}{Example}
    Say we toss a coin twice, then the sample space is $\Omega = \{HH, HT, TH, TT\}$.

    We define the following events:
    \begin{itemize}
        \item $A = \{\text{first toss is a head}\} = \{HH, HT\}$
        \item $B = \{\text{2 times heads}\} = \{HH\}$
    \end{itemize}

    We have that $A \subseteq B$, so $A$ depends on $B$.
\end{examplebox}

\begin{notebox}{singleton events}
    Events that contain only one outcome are called \textbf{singleton events}.
\end{notebox}

\subsection{Sigma-algebras}

We usually denote a sigma-algebra as $\F$ and it's a subset of the power set of $\Omega$.
Every sigma-algebra, by definition, has the following properties:
\begin{enumerate}
    \item $\Omega \in \F$
    \item If $A \in \F$, then $A^c \in \F$
    \item If $A_1, A_2, \dots \in \F$, then $\bigcup_{i=1}^\infty A_i \in \F$
\end{enumerate}

\begin{notebox}{notation}
    We introduce the following notation:
    $$
        \bigcap_{i=1}^\infty A_i \qquad \text{ and } \qquad \bigcup_{i=1}^\infty A_i
    $$

    These symbols just denote the intersection and the union respectively for a infinite, countable set of events.
\end{notebox}

\subsubsection{Proprieties of sigma-algebras}

Moreover, we can prove the following properties:
\begin{enumerate}
    \item $\varnothing \in \F$
    \item $\F$ is closed under finite unions
    \item $\F$ is closed under finite and countable intersections
\end{enumerate}

\begin{proof}
    According to the properties of sigma-algebras we have that:

    \begin{enumerate}
        \item $\Omega \in \F$ by $(1)$, so $\Omega^c = \varnothing \in \F$, by $(2)$.
        \item Let $A_1 \cup A_2 \cup \dots \cup A_n$, we extend it by adding $\varnothing$ to the end of the sequence.
              Then this is in $\F$ by $(3)$.
        \item According to De Morgan's laws we have:
              $$
                  \bigcap_{i=1}^n A_i = \left(\left(\bigcap_{i=1}^n A_i\right)^c\right)^c = \left(\bigcup_{i=1}^n A_i^c\right)^c
              $$
              Since the intersection is in $\F$ by $(3)$ and the complement is in $\F$ by $(2)$, then the union is also in $\F$.
    \end{enumerate}
\end{proof}

\begin{notebox}{Reminder of De Morgan's Laws}
    $$
        \left(\bigcup_{i=1}^n A_i\right)^c = \bigcap_{i=1}^n A_i^c
        \qquad \text{ and } \qquad
        \left(\bigcap_{i=1}^n A_i\right)^c = \bigcup_{i=1}^n A_i^c
    $$
\end{notebox}

\subsubsection{Borel sigma-algebra}

Suppose $\Omega$ is uncountable and we want to define a sigma-algebra on it.

We start by choosing a class $\mathcal C$ of subsets of $\Omega$ that we want to include in the sigma-algebra.
Then we define the sigma-algebra generated by $\mathcal C$ as the smallest sigma-algebra that contains $\mathcal C$.

In particular Borel sigma-algebra is the sigma-algebra generated by the open sets of $\R$:
$$
    \mathcal B (\R) = \sigma(\{(a, b) : -\infty < a < b < \infty\})
$$

This definition can be extended to $\R^n$ by taking the gaussian product sigma-algebra:
$$
    \mathcal B (\R^n) = \sigma(\{(a_1, b_1) \times \dots \times (a_n, b_n) : -\infty < a_i < b_i < \infty\})
$$

\end{document}