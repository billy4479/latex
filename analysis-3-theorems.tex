\documentclass[12pt]{extarticle}

\title{Analysis 3 Theorems}
\author{Giacomo Ellero}
\date{a.y. 2024/2025}

\usepackage{preamble_base}
\usepackage{preamble_math}

% \renewcommand{\vec}[1]{\uvec{#1}}
\numberwithin{equation}{section}

\begin{document}

\section{First partial}

\subsection{Maximal solutions either coincide or do not intersect}

\begin{theorem}{}{}
    In the same setting of the Peano-Picard theorem,
    consider the maximal solution $u: I \to \R$
    for the i.v.p. $u(t_0) = \lambda_0$.
    Given $t_1 \in I$ and $\lambda_1 = u(t_1)$,
    consider $\tilde{u}: \tilde I \to \R$,
    the solution to the same ODE with initial condition $\tilde{u}(t_1) = \lambda_1$.
    Then
    \begin{equation}
        I = \tilde I \quad \text{and} \quad u = \tilde u
    \end{equation}
\end{theorem}

\begin{proof}
    Since, by assumption, $u$ also solves the i.v.p. with initial condition $(t_1, \lambda_1)$,
    by Peano-Picard, $I \subseteq \tilde I$ and $u = \left. \tilde u \right|_I$.

    Since $t_0 \in I$ it means $t_0 \in \tilde I$, meaning that $\tilde u(t_0) = \lambda_0$
    (since $u(t_0) = \lambda_0$), but now we can apply Peano-Picard again for the initial conditions $(t_0, \lambda_0)$
    and prove the claim.
\end{proof}


\begin{theorem}{}{}
    Under the assumptions of Peano-Picard, let $(t_0, \lambda_0)$ and $(t_0, \mu_0)$ with $\lambda_0 \ne \mu_0$.
    Consider $u: I' \to \R$ and $v: I'' \to \R$ to be solutions for the ODE with initial conditions as above.
    Then
    \begin{equation}
        u(t) \neq v(t) \quad \forall t \in I' \cap I''
    \end{equation}
\end{theorem}

\begin{proof}
    Assume that there exists $t_1 \in I' \cap I''$ such that $u(t_1) = v(t_1) = \lambda_1$.
    Let $w: I \to \R$ be the maximal solution for initial condition $(t_1, \lambda_1)$,
    then we must have that
    \begin{equation}
        I'\subseteq I, \quad I'' \subseteq I, \quad u = \left. w\right|_{I'}, \quad v = \left. w\right|_{I''}
    \end{equation}

    In particular, we must have $t_0 \in I \implies u(t_0) = w(t_0) = v(t_0)$, which is a contradiction.
\end{proof}

\subsection{\texorpdfstring{$t^j e^{\alpha_\ell t}$ form a basis of solutions}{Basis of solutions}}

\begin{theorem}{}{}
    Assume that the roots of the characteristic polynomial $p(z)$ are all real.
    Let $\alpha_1, \dots, \alpha_n$ be the distinct roots of $p(z)$ and
    $M_1, \dots, M_n$ be the algebraic multiplicity of each $\alpha_\ell$.

    For each root, consider the following set of functions:
    \begin{equation}
        S = \left\{ t^j e^{\alpha_l t} : l \in 1, \dots, n; j \in 0, \dots, m_k -1 \right\}
    \end{equation}
    each one of them solves the ODE and $S$ is a basis of the space of solutions.
\end{theorem}

\begin{proof}
    We will show that all the functions in $S$ solve the ODE.
    Then, to show that they form a basis, it is sufficient to note that
    $\abs{S} = \sum M_i = k$ where $k$ is the degree of $p(z)$,
    hence we just need to show that they are linearly independent.
    We will not prove linear independence.

    Introduce the operator $a(D) : X \to X$, where $a$ is a polynomial and
    $X$ is the set of functions of class $C^\infty$.
    Given $a(z) = \sum_{j= 0}^\ell \mu_j z^j$, the operator is defined as
    \begin{equation}
        a(D)f = \sum^{\ell}_{j = 0} \mu_j \dv[j]{f(t)}{t}
    \end{equation}
    Note that this operator is linear and behaves like a polynomial,
    notably if $c(z) = a(z)b(z)$ then $c(D) = a(D) \circ b(D)$ is commutative.

    Assume the characteristic polynomial has leading coefficient of $1$,
    then $u$ is a solution to the ODE (up to multiplying it by the inverse of the leading coefficient) if
    \begin{equation}
        p(D)u = 0
    \end{equation}

    Let us factorize $p(z)$ as
    \begin{equation}
        p(z) = (z-\alpha_1)^{M_1} \dots (z-\alpha_n)^{M_n}
    \end{equation}
    now isolate $r(z) = (z-\alpha_\ell)^{M_\ell}$ for some $\ell \in 1, \dots, n$
    and let $q(z) = \prod_{j \neq \ell} (z-\alpha_j)^{M_j}$
    such that $p(z) = q(z)r(z)$.
    Then, to show that $u$ solves the ODE it is enough to show that
    \begin{equation}
        r(D)u = (D - \alpha_\ell)^{M_\ell}u = 0
    \end{equation}

    We want to show this holds for each candidate solution.
    We can write a general candidate as $h(t) e^{\alpha_\ell t}$
    where $h(t)$ is a polynomial of degree less than $M_\ell$.
    Hence our proof reduces to showing that
    \begin{equation}
        (D - \alpha_\ell)^{M} [h(t) e^{\alpha_\ell t}] = 0
    \end{equation}
    with $M = M_\ell$.

    We can now proceed by induction on $M$:
    \begin{description}[font=\normalfont\itshape\space]
        \item[Base case]
              Since $M = 1$ the degree of $h(t)$ has to be $0$, hence it is constant.
              We have
              \begin{equation}
                  (D - \alpha_\ell)[h e^{\alpha_\ell t}] = h \dv{t} e^{\alpha_\ell t} - h\alpha e^{\alpha_\ell t} = 0
              \end{equation}
        \item[Inductive step]
              \begin{align}
                  (D - \alpha_\ell)^M[h(t) e^{\alpha_\ell t}] & = (D - \alpha_\ell)^{M-1}\left[ (D - \alpha_\ell) \left[h(t) e^{\alpha_\ell t}\right]\right]                                                               \\
                                                              & = (D - \alpha_\ell)^{M-1}\left[ h'(t) e^{\alpha_\ell t} + \cancel{h(t)\alpha_\ell e^{\alpha_\ell t}} - \cancel{h(t) \alpha_\ell e^{\alpha_\ell t}} \right] \\
                                                              & = 0
              \end{align}
              where in the last step we notice that since $h'(t)$ has degree less than $M -1$
              hence the inductive hypothesis applies.
    \end{description}
\end{proof}

\subsection{\texorpdfstring{$m_\gamma \leq M_\gamma$}{Geometric multiplicity less than or equal to algebraic}}

\begin{theorem}{}{}
    For all eigenvalues $\gamma \in \R$ there holds $1 \leq m \leq M$,
    where $m$ is geometric multiplicity of $\gamma$ and $M$ is the algebraic multiplicity of $\gamma$.
\end{theorem}

\begin{proof}
    Since $\gamma$ is an eigenvalue there exists an eigenvector $v \neq 0$,
    hence $\ker (A - \gamma \mathds{1}) \ne 0$ and $m \geq 1$.

    Let $f: V \to V$ be the linear map associated to $A$: $f(v) = A v$, where $V = \R^n$.
    Let $\{v_1, \dots, v_m\}$ be the basis of the eigenspace of $\gamma$
    and complete it to $\{v_1, \dots, v_n\}$ a basis of $V$.
    Then let $P = \begin{pmatrix}v_1 & \dots & v_n\end{pmatrix}$ be the change of basis matrix
    such that
    \begin{equation}
        A = P B P^{-1}
    \end{equation}

    Since $f(v_j) = \gamma v_j$ for all $j \in 1, \dots, m$, $B$ has the form of
    \begin{equation}
        B = \left(\begin{array}{c|c}
            \begin{matrix}
                \gamma &        & 0      \\
                       & \ddots &        \\
                0      &        & \gamma
            \end{matrix} & *_1          \\
            \hline
            0                           & *_2
        \end{array}
        \right)
    \end{equation}
    Notice that $z \mathds{1} - B$ is a triangular block matrix, hence its characteristic polynomial is
    \begin{equation}
        \det(z \mathds 1 - B) = (z-\gamma)^m \det(z \mathds 1_{n-m} - *_2)
    \end{equation}
    but since $A$ and $B$ are similar they have the same characteristic polynomial,
    hence $(z-\gamma)^m$ divides the characteristic polynomial of $A$,
    which means $M \geq m$.
\end{proof}

\subsection{Diagonalizable \texorpdfstring{$\iff m_\gamma = M_\gamma$}{iff geometric equals to algebraic}}

\begin{theorem}{}{}
    A square matrix is diagonalizable over $\C$ if and only if
    for each root $\gamma$ of the characteristic polynomial
    $m_\gamma = M_\gamma$.

    Moreover, it is diagonalizable over $\R$ if and only if all the roots are also real.
\end{theorem}

\begin{proof}
    We will just show the statement over $\R$ for simplicity.

    We will also use lemma which reads that non-zero eigenvector corresponding
    to distinct eigenvalues are linearly independent without proof.

    \begin{description}
        \item[$\implies$] If a matrix is diagonalizable over $\R$ we can find a basis of eigenvector,
              $\mathcal B = \{v_1, \dots, v_n \} \in \R^n$: call $\gamma_j$ their eigenvalues.
              Fix a specific $\gamma_j = \gamma$, then let $n_\gamma$ be the number of elements of $\mathcal B$
              which are eigenvectors of $\gamma$.
              These vectors belong to the eigenspace $\ker (A - \gamma \mathds 1)$ and,
              since they are part of a basis, they are linearly independent.
              So far we have that
              \begin{equation}
                  n_\gamma \leq m_\gamma \leq M_\gamma
              \end{equation}
              Moreover, since the vectors $v_j$ form a basis, necessarily $\sum n_\gamma = n$.
              On the other hand, the sum of the multiplicities of the real roots of a polynomial is at most its degree,
              which means $\sum M_\gamma \leq n$, and we obtain
              \begin{equation}
                  n = \sum n_\gamma \leq \sum m_\gamma \leq \sum M_\gamma \leq n
              \end{equation}
              therefore we must have $n_\gamma = m_\gamma = M_\gamma$ for all real roots $\gamma$.
              It also follows that, since $\sum M_\gamma = n$ (where $n$ was just the sum over the real roots),
              $p(z)$ has only real roots (there is no \say{space} for complex ones).
        \item[$\impliedby$] If all roots are real and $m_\gamma = M_\gamma$, then $\sum m_\gamma = \sum M_\gamma = n$.
              This means that we can find a basis of $\ker(A - \gamma \mathds{1})$ consisting of $m_\gamma$ vectors
              for each $\gamma$, which together form a collection of $n$ vectors.
              To conclude they are a basis we are just left to check that they are linearly independent.
              We now just use the fact that basis of eigenspaces are indeed linearly independent among the same eigenspace
              and combine it with the lemma.
    \end{description}
\end{proof}

\section{Second partial}

\subsection{Completeness of the metric space of functions}

\begin{definition}{Metric space of bounded continuous functions}
    Let $(X, d)$ be a metric space where $X$ is the set
    \begin{equation}
        X = \{ g:I \to \R : g \text{ bounded and continuous} \}
    \end{equation}
    and $d$ is induced from the \emph{supremum norm}:
    \begin{equation}
        d(g_1, g_2) = \norm{g_1 - g_2} = \sup_{t \in I} \abs{g_1 - g_2}
    \end{equation}
    Note that the supremum is always finite since $g$ is bounded.
\end{definition}


\begin{theorem}{}{}
    The metric space $(X, d)$, as defined above, is complete.
\end{theorem}
\begin{proof}
    Let $g_k$ be a Cauchy sequence in $(X, d)$,
    we want to show that there exists $g \in X$ such that $d(g_k, g) \to 0$ (i.e. $\norm{g_k - g} \to 0$) as $k \to \infty$.

    Since $g_k$ is a Cauchy sequence $\exists K \in \N$ such that $\norm{g_k - g_l} < \varepsilon$ for $k, l \geq K$ and, for an arbitrary $x \in I$ we have
    \begin{equation}
        \abs{g_k(x)-g_l(x)} \leq \norm{g_k - g_l} < \varepsilon \quad \forall k, l \geq K
    \end{equation}
    by our definition of the norm.
    This tells us that the sequence of numbers $(g_k(x))_{k \in \N}$ is a Cauchy sequence over $\R$ and, by completeness of $\R$ this sequence converges.
    We can therefore write
    \begin{equation}
        g(x) = \lim_{k \to \infty} g_k(x)
    \end{equation}
    for each $x \in I$.

    We now have a definition for $g$, we are left to prove the following:
    \begin{itemize}
        \item \emph{$g_k$ converges to $g$ uniformly}.

              Consider the definition of Cauchy sequence, fix $k \geq K$ and let $l \to \infty$.
              We get that
              \begin{equation}
                  \abs{g_k(x) - g(x)} = \lim_{l \to \infty} \abs{g_k(x) - g_l(x)} \leq \varepsilon
              \end{equation}
              Since $\varepsilon > 0$ is arbitrary all functions $g_k$ converge uniformly to $g$.

        \item \emph{$g$ is bounded}

              Let $\varepsilon = 1$.
              We have
              \begin{equation}
                  \abs{g(x)} = \abs{g(x) - g_K(x) + g_K(x)} \leq \abs{g(x) - g_K(x)} + \abs{g_K(x)} \leq 1 + \abs{g_K(x)}
              \end{equation}
              and since $g_K$ is bounded we get the result.

        \item \emph{$g$ is continuous}

              Since all $g_k$ are continuous and $g_k \to g$, $g$ is continuous function too.
              (Given for granted in the exam).
    \end{itemize}
\end{proof}

\subsection{Banach fixed-point theorem}

\begin{theorem}{}{banach}
    Let $(X, d)$ be a \emph{complete} metric space and $F$ a contraction.
    Then, there exists an unique $\bar x \in X$ such that
    \begin{equation}
        F(\bar x) = \bar x
    \end{equation}
\end{theorem}

\begin{proof}
    First let us show that $\bar x$ is unique.
    Assume that $F(x) = x$ and $F(y) = y$, hence
    \begin{equation}
        d(x, y) = d(F(x), F(y)) \leq \alpha d(x, y) \implies (1-\alpha) d(x, y) \leq 0
    \end{equation}
    and since distances are non-negative we get that $d(x, y) = 0$ but this implies $x = y$.

    We now want to prove the existence of $\bar x$.
    For the first step we define a sequence of points recursively as follows:
    \begin{equation}
        x_{k + 1} = F(x_k)
    \end{equation}
    and claim that
    \begin{equation}
        d(x_k, x_{k+1}) \leq \alpha^k d(x_0, x_1) \quad \forall k \in \N .
    \end{equation}
    We proceed by induction: the base case of $k = 0$ is trivial since $\alpha^0 = 1$ (assume that $0^0=1$);
    for the inductive step we have
    \begin{equation}
        d(x_{k+1}, x_{k+2}) = d(F(x_k), F(x_{k+1})) \leq \alpha d(x_k, x_{k+1}) \leq \alpha \alpha^k d(x_0, x_1)
    \end{equation}
    as desired.

    Next, we let $C = d(x_0, x_1)$ and claim that
    \begin{equation}
        d(x_k, x_{k+1}) \leq C \sum_{j = k}^{l-1} \alpha^j \quad \forall k, \ell \text{ s.t. } 0 \leq k < \ell.
    \end{equation}
    We again proceed by induction on $\ell > j$:
    the base case will be $\ell = j + 1$ which reduces to $d(x_k, x_{k+1}) \leq C \alpha^k$, which is the result of the previous claim;
    at $\ell + 1$ we have
    \begin{equation}
        d(x_k, x_{l+1}) \leq d(x_k, x_\ell) + d(x_\ell, x_{l + 1}) \leq d(x_k, x_\ell) + C \alpha^\ell \leq C\sum_{j = k}^{l-1} \alpha^j + C \alpha^\ell
    \end{equation}
    thanks to the triangle inequality, the previous claim and the inductive hypothesis.

    Now we can combine the non-negativity of $d$, the claim we just proved and some algebra to get that for $0 \leq k < \ell$
    \begin{equation}
        0 \leq d(x_k, x_\ell) \leq C \sum_{j = k}^{l-1} \alpha^j = C \frac{\alpha^k - \alpha^\ell}{1-\alpha}
    \end{equation}
    but the sequence on the right goes to $0$ as $k, j \to \infty$ (since $\alpha \in [0, 1)$) hence
    \begin{equation}
        \lim_{k, j \to \infty} d(x_k, x_{\ell}) = 0
    \end{equation}
    which means that $(x_k)_{k \in \N}$ is a Cauchy sequence.

    Since $(X, d)$ is complete by hypothesis, $(x_d)$ must converge to some $\bar x \in X$.
    We have
    \begin{equation}
        \bar x = \lim_{k \to \infty} x_k = \lim_{k \to \infty} x_{k + 1} = \lim_{k \to \infty} F(x_k).
    \end{equation}
    Since $F$ is continuous because it is a contraption (why?), for any converging sequence $x_k \to \bar x$ we have
    \begin{equation}
        d(F(\bar x), F(x_k)) \leq \alpha d(\bar x, x_k) \to 0
    \end{equation}
    as $k \to \infty$.
    Therefore $d(F(\bar x), F(x_k)) \to 0$ and thus $F(x_k) \to F(\bar x)$.
\end{proof}

\subsection{Lipschitz perturbation of the identity}

\begin{theorem}{}{}
    Let $f: U \to \R^d$ with $U \subseteq \R^p$ open be defined as
    \begin{equation}
        f(x) = x + g(x) \quad \text{with } \abs{g(a) - g(b)} \leq \alpha \abs{a - b}, \enspace \alpha \in [0,1)
    \end{equation}
    that is, a Lipschitz perturbation of the identity.
    Then $f(U)$ is open, $f: U \to f(U)$ is a bijection, and it has a continuous inverse $f^{-1}$
    which satisfies
    \begin{equation}
        \abs{f^{-1}(a)- f^{-1}(b)} \leq \frac{1}{1-\alpha} \abs{a-b}
    \end{equation}
\end{theorem}

\begin{proof}
    For the exam we will just be asked to show that $f(B_r(x_0))$ includes $B_{(1-\alpha)r}(f(x))$.

    Consider $x \mapsto f(x_0 + x) - f(x_0)$, then if a claim holds for this new function at $0$, it also holds for $f$ at $x_0$.
    This means that we can assume $x_0 = 0$ and $f(x_0) =0$ without loss of generality.

    Since $f(x) = x + g(x)$, for a fixed $y \in B_{(1-\alpha)r}(0)$ we want to solve $x + g(x) = y$ or equivalently
    \begin{equation}
        F(x) = x = y-g(x)
    \end{equation}
    in other words this means we want to find a fixed point of $F$.
    Since $B_r(0)$ is not complete, we can take a closed ball $\bar B_s(0)$ with $0 < s< r$ giving us
    $\bar B_s(0) \subset B_r(0) \subset U$.
    Since closed balls constitute closed sets and $\bar B_s(0)$ is a subset of a complete metric space,
    $(\bar B_s(0), d)$ is a complete metric space.

    In order to be able to apply \cref{thm:banach} we are left to check that $F$ maps $X = \bar B_s(0)$ to itself
    for a suitable $s$.
    Given $x \in \bar B_s(0)$ we have
    \begin{equation}
        \abs{F(x)} = \abs{y - g(x)} \leq \abs{y} + \abs{g(x)} =  \abs{y} + \abs{g(x) - g(0)} \leq \abs{y} + \alpha \abs{x - 0} \leq \abs{y} + \alpha s
    \end{equation}
    where we used the fact that $g(0) = f(0) - 0 = 0$.
    In order to deduce that $F(x) \in \bar B_s (0)$ we need to have $\abs{y} + \alpha s \leq s$, that is
    \begin{equation}
        \frac{\abs{y}}{1-\alpha} \leq s < r
    \end{equation}
    and since we are are assuming $\abs{y} < (1-\alpha) r$ such radius $s$ exists.

    Finally we check that $F$ is a contraption:
    \begin{equation}
        \abs{F(x) - F(x')} = \abs{(y-g(x')) - (y-g(x'))} = \abs{g(x)-g(x')} \leq \alpha \abs{x-x'}
    \end{equation}

    Then the existence of the fixed point is guaranteed by \cref{thm:banach}.
\end{proof}

\subsection{Local Peano-Picard}

Consider the following set with $T, M > 0$
and the initial condition $(t_0, \lambda_0) \in A$ of some initial value problem:
\begin{equation}
    X_{T, M} = \{ v: [t_0 - T, t_0 + T] \to \R^n : v \text{ continuous and } \norm{v-\lambda_0} \leq M \}
\end{equation}
where
\begin{equation}
    \norm{v-\lambda_0} = \max_{t \in [t_0 - T, t_0 + T]} \abs{v(t) - \lambda_0}
\end{equation}
is the usual supremum norm, which means that
\begin{equation}
    \norm{v - \lambda_0} \leq M \iff \abs{v(t) - \lambda_0} \enspace \forall t \in [t_0 - T, t_0 + T]
\end{equation}

We then define a metric space $(X_{T, M}, d)$ with
\begin{equation}
    d(v, w) = \max_{t \in [t_0 - T, t_0 + T]} \abs{v(t) - w(t)}
\end{equation}

We can prove that the $(X_{T,M}, d)$ is complete.

\begin{theorem}{Peano-Picard local version - EXAM}{}
    For $M > 0$ small enough and, once $M$ is fixed, for $T>0$ small enough, the set $X_{T, M}$ contains exactly one solution
    for the initial value problem.
\end{theorem}

Since $A$ is open, it includes an open ball around the point $p = (t_0, \lambda_0)$:
for some $R > 0$ we have $B_R(p) \subseteq A$.
Moreover, by shrinking $R$ we can assume that $\eval{f}_{B_R(p)}$ is $L$-Lipschitz.
We will always take $T, M \leq R/2$ from now on, which guarantees that points in the graph of a function $v \in X_{T, M}$ belong to the domain $A$ of $f$.

Recall that $v \in X_{T, M}$ solves the initial value problem iff
\begin{equation}
    v \mapsto F(v), \quad F(v)(t) = \lambda_0 + \int^t_{t_0} f(s, v(s)) \dd s
\end{equation}

In order to prove the theorem we want to show that
\begin{enumerate}
    \item $F(X_{T, M}) \subseteq X_{T, M}$ for a suitable $T, M >0$
    \item $\eval{F}_{X_{T, M}}$ is a contraption on a complete metric space for some suitable $T, M > 0$
\end{enumerate}
and then apply \cref{thm:banach}.

\begin{proposition}{$F$ is a contraption}{}
    If $M, T > 0$ are small enough, then the function $F: X_{T, M} \to X_{T, M}$
    is a contraption.
    Specifically, this will hold as long as
    \begin{equation}
        0 < M \leq \frac{R}{2} \quad \text{and} \quad 0 < T \leq \min \left\{ \frac{R}{2}, \frac{M}{\Lambda}, \frac{1}{2L} \right\}
    \end{equation}
    where $L$ is the Lipschitz constant for $F$ and
    $\Lambda$ is such that $\abs{F(q)} \leq \Lambda \enspace \forall q \in B_R(p)$.
\end{proposition}

\begin{proof}
    Take for granted that $F(X_{T, M}) \subseteq X_{T, M}$.
    To show that $F$ is a contraption with $\alpha = 0.5$ we want to show that
    \begin{equation}
        \norm{F(v) - F(w)} = \frac{1}{2} \norm{v-w}
    \end{equation}

    With $t \in [t_0 - T, t_0 + T]$ we compute that
    \begin{align}
        \abs{F(v)(t) - F(w)(t)} & = \abs{\int_{t_0}^{t} f(s,v(s)) \dd s - \int_{t_0}^{t} f(s,w(s)) \dd s} \\
                                & \leq \int_{t_0}^{t} \abs{f(s,v(s)) - f(s,w(s))} \dd s                   \\
                                & \leq \int_{t_0}^{t} L\norm{v - w} \dd s                                 \\
                                & \leq L \norm{v - w} (t - t_0) \leq LT \norm{v - w}                      \\
    \end{align}
    therefore as long as $LT \leq 0.5$ the claim is true.
\end{proof}

\end{document}