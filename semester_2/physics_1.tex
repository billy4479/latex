\documentclass[14pt]{extarticle}
\title{Physics 1 Notes}
\author{Giacomo Ellero}

\usepackage{amsfonts}
\usepackage{amsthm}
\usepackage{amssymb}
\usepackage{amsmath}
\usepackage{mathtools}
\usepackage{commath}
\usepackage{dirtytalk}
\usepackage{parskip}
\usepackage{mathrsfs}
\usepackage[many]{tcolorbox}
\usepackage{xparse}
\usepackage[a4paper,margin=1.5cm]{geometry}
\usepackage{bookmark}
\usepackage{physics}
\usepackage{tikz}
\usepackage{cancel}
\usepackage{siunitx}

\newcommand{\C}{\mathbb{C}}
\newcommand{\R}{\mathbb{R}}
\newcommand{\F}{\mathcal{F}}
\renewcommand{\Re}{\operatorname{Re}}
\renewcommand{\Im}{\operatorname{Im}}

\newenvironment{absolutelynopagebreak}
  {\par\nobreak\vfil\penalty0\vfilneg
   \vtop\bgroup}
  {\par\xdef\tpd{\the\prevdepth}\egroup
   \prevdepth=\tpd}

\newtcolorbox{examplebox}[1]{colback=green!5!white,colframe=green!40!black,title={#1},fonttitle=\bfseries,parbox=false}
\newtcolorbox{notebox}[1]{colback=blue!5!white,colframe=blue!40!black,title={Note: #1},fonttitle=\bfseries,parbox=false}
\newtcolorbox{bluebox}[1]{colback=blue!5!white,colframe=blue!40!black,title={#1},fonttitle=\bfseries,parbox=false}
\newtcolorbox{warningbox}[1]{colback=orange!5!white,colframe=orange!90!black,title={Warning: #1},fonttitle=\bfseries,parbox=false}
   
\begin{document}

\maketitle
\tableofcontents
\clearpage

\section{Class of 06/02/2024 - Introduction}

\subsection{How does science work?}

\subsubsection{Dimension, measure and uncertainty}

We will use SI units. Defining such units is a hard task.

When we measure something we need to take into account the uncertainty of the measurement:
the tool we use to measure has a precision, hence we cannot measure something smaller than the precision of the tool.

Moreover we have to check that the measurements are compatible with the errors in each of them.

\subsubsection{Dimensionsional analysis}

When we do a calculation we need to check that the units are consistent.

This can also be done before the actual calculation to check that the theory is truthful.

\subsubsection{Theory vs reality}

In our theory we take in account way less phenomena than what happens in reality.
We need to understand what we can neglect and what we can't.

\begin{enumerate}
  \item Identify the key ingredients
  \item Build a simplified model of reality
  \item Solve it (if you can)
  \item Add additional ingredients
\end{enumerate}

\subsection{Introduction to kinematics}

In this course we will discuss \textbf{kinematics} and \textbf{dynamics} which are two branches of mechanics:
\begin{itemize}
  \item \textbf{Kinematics}: How did the object move?
  \item \textbf{Dynamics}: Why did the object move?
\end{itemize}

\subsubsection{Position vector}

We will start describing the motion of a point mass or a particle
and to do so we wll need to introduce vectors from linear algebra.

We have two ways to represent a vector.
Here we use the position vector as an example.

\textbf{Position vector} $\vec{r}$:
\begin{itemize}
  \item $\vec{r} = (x, y, z)$
  \item $\vec{r} = x \hat{i} + y \hat{j} + z \hat{k}$
\end{itemize}

In the second representation we have the \textbf{unit vectors} $\hat{i}, \hat{j}, \hat{k}$ which are the basis of the space.

Moreover we can represent the position vector as a function of time:
$$
  \vec{r}(t) = \begin{pmatrix} x(t) \\ y(t) \\ z(t) \end{pmatrix}
$$

$\vec {r}(t)$ is called trajectory.

\subsubsection{Degrees of freedom}

We call \textbf{degree of freedom} the number of independent coordinates needed to describe the position of a point.
Generally we need 3 degrees of freedom.

\begin{examplebox}{Example}
  \begin{itemize}
    \item Consider two points connected by a rigid rod of length $l$ (of negligible cross section).

          The position of the first point $A$ requires all 3 degrees of freedom,
          but for the second point $B$ we only need 2 degrees of freedom since the distance from $A$ is fixed.

    \item A rigid body has 6 degrees of freedom: 3 for the position of the center of mass and 3 for the orientation.
  \end{itemize}

\end{examplebox}

\subsubsection{Velocity}

We define the \textbf{displacement vector} as
$$
  \Delta \vec{r} = \vec{r}(t') - \vec{r}(t) \text{ with } t' > t
$$

Note that the displacement vector could be zero even if the particle moved.

We define the \textbf{average velocity} as $\vec{v}_{\text{avg}} = \frac{\Delta \vec{r}}{\Delta t}$.

We define the \textbf{instantaneous velocity} as
\begin{align*}
  \vec{v}(t) & = \lim_{\Delta t \to 0} \frac{\Delta \vec{r}}{\Delta t}                                            \\
             & = \dv{\vec{r}}{t} \text{ this is the derivative of a vector}                                       \\
             & = \begin{pmatrix} \dv{x(t)}{t} \hat{i} & \dv{y(t)}{t} \hat{j} & \dv{z(t)}{t} \hat{k} \end{pmatrix}
\end{align*}

\subsubsection{Acceleration}

Similarly to the velocity, we define the \textbf{average acceleration} as $\vec{a}_{\text{avg}} = \frac{\Delta \vec{v}}{\Delta t}$
and we pass to the limit to get

\begin{align*}
  \vec{a}(t) & = \lim_{\Delta t \to 0} \frac{\Delta \vec{v}}{\Delta t} \\
             & = \dv{\vec{v}}{t} = \dv[2]{\vec{r}}{t}
\end{align*}

We see that the acceleration is the second derivative of the position vector.

We will be able to reconstruct the whole trajectory starting from the acceleration.

\section{Class of 08/02/2024 - Kinematics}

\subsection{Decomposing vectors}

Consider a 2D position vector, let $\theta$ be the angle between the vector and the $x$ axis,
let $r = \norm{\vec{r}}$ be the magnitude of the vector,
and let $\hat{r} = \vec{r} / r$ be the unit vector in the direction of $\vec{r}$.

\begin{align*}
  \vec{r} & = x \hat{i} + y \hat{j}                         \\
          & = r \cos \theta \hat{i} + r \sin \theta \hat{j} \\
          & = r \hat{r}
\end{align*}

This decomposition works for any vector in 2D, like the velocity and the acceleration.

\subsection{Changing the direction of motion is an acceleration}

According to these definitions we have that the acceleration, which is the derivative of the velocity, can be written as

\begin{align*}
  \vec{a} & = \dv{\vec{v}}{t} = \dv{t} (v \hat{v}) \\ &=
  \underbrace{ \dv{v}{t} \hat{v} }_{\text{change in speed}} +
  \underbrace{ v \dv{\hat{v}}{t} }_{\text{change in direction}}
\end{align*}

We see how changing the direction of the velocity will cause an acceleration.

\subsection{Recovering the trajectory from the acceleration}

We can use differential equations to get that

\begin{align*}
  \vec{v}(t) & = \int \vec{a}(t) \dd{t} + c_1 \\
  \vec{r}(t) & = \int \vec{v}(t) \dd{t} + c_2
\end{align*}

In the result we get two constants of integration $c_1$ and $c_2$
which can be determined by knowing the position or velocity at a certain time.

For example if $v(t_0)$ and $r(t_0)$ are known we have that, at $t_1$,

\begin{align*}
  \vec{v}(t_1) & = \vec{v}(t_0) + \int_{t_0}^{t_1} \vec{a}(t) \dd{t} \\
  \vec{r}(t_1) & = \vec{r}(t_0) + \int_{t_0}^{t_1} \vec{v}(t) \dd{t}
\end{align*}

\subsubsection{Uniformly accelerated motion}

If we consider the case where $\vec{a}$ is constant we get
the high-school equation for uniformly accelerated motion:

\begin{align*}
  \vec{v}(t) & = \vec{a} t - v(t_0)                                \\
  \vec{r}(t) & = \frac{1}{2} \vec{a} t^2 - v(t_0) t + \vec{r}(t_0)
\end{align*}

Moreover, we can substitute one into the other to get an equation for the trajectory:
$$
  r = \frac{v^2(t) - v^2(t_0)}{2a}
$$

\subsection{Relative motion}

Multiple observers may have different reference frames,
we need a way to convert from a reference frame to the other.

\begin{center}
  \begin{tikzpicture}
    % Normal axis
    \draw[black, ->] (-3,0) -- (5,0) node[anchor=north]{$\hat{i}$};
    \draw[black, ->] (0,-3) -- (0,5) node[anchor=east]{$\hat{j}$};

    % Rotated axis
    \draw[dashed, black, ->] (-3,-1.8) -- (5,3) node[anchor=south]{$\hat{i'}$};
    \draw[dashed, black, ->] (1.8,-3) -- (-3,5) node[anchor=east]{$\hat{j'}$};

    % Vector
    \draw[blue, thick, ->] (0,0) -- (3,3) node[anchor=south]{$\vec{r}$};

    % Angle between axis
    \draw[red, thick] (0.5,0) arc (0:31:0.5) node[right]{$\theta$};
  \end{tikzpicture}
\end{center}

We can write $\hat{i}$ and $\hat{j}$ in terms of $\hat{i'}$ and $\hat{j'}$:

\begin{align*}
  \hat{i'} & = \left( \hat{i'} \cdot \hat{i} \right) \hat{i}
  + \left( \hat{i'} \cdot \hat{j} \right) \hat{j}            \\
           & = \cos \theta \hat{i} + \sin \theta \hat{j}     \\
  \hat{j'} & = \left( \hat{j'} \cdot \hat{i} \right) \hat{i}
  + \left( \hat{j'} \cdot \hat{j} \right) \hat{j}            \\
           & = -\sin \theta \hat{i} + \cos \theta \hat{j}
\end{align*}

Then, $\vec{r} = x \hat{i} + y \hat{j}$ can be written as

\begin{align*}
  \vec{r} & = x \left( \cos \theta \hat{i'} + \sin \theta \hat{j'} \right)  + y \left( -\sin \theta \hat{i'} + \cos \theta \hat{j'} \right) \\
          & = \underbrace{ (x \cos \theta - y \sin \theta) }_x \hat{i'}
  + \underbrace{ (x \sin \theta + y \cos \theta) }_y \hat{j'}
\end{align*}

We just performed a linear transformation on the position vector,
we can write the matrix representing this transformation:

$$
  R = \begin{pmatrix}
    \cos \theta & -\sin \theta \\
    \sin \theta & \cos \theta
  \end{pmatrix}
$$

We have
$$
  \begin{pmatrix}
    x \\
    y
  \end{pmatrix}
  =
  \begin{pmatrix}
    \cos \theta & -\sin \theta \\
    \sin \theta & \cos \theta
  \end{pmatrix}
  \begin{pmatrix}
    x' \\
    y'
  \end{pmatrix}
$$

The inverse transformation is given by the inverse of the matrix:

$$
  R^{-1} = \begin{pmatrix}
    \cos \theta  & \sin \theta \\
    -\sin \theta & \cos \theta
  \end{pmatrix}
$$

\subsubsection{Frames of reference in relative motion}

Using vectors we can abstract away the complexity derived from the fact that the reference frames are moving.

Consider the classic example of a train moving at constant speed, with two observers, one on the train and one on the ground,
and a ball which is moving inside the train.

We have that the trajectory of the ball for the observer on the ground is given by
$$
  \vec{r}_{gb} = \vec{r}_{gt} + \vec{r}_{tb}
$$

We don't need to worry about the direction of the train, we can just sum the vectors.

Similar considerations can be made for the velocity and the acceleration.

\section{Class of 12/02/2024}

\subsection {Uniform circular motion}

In this kind of motion a particle spins in circle of radius $R$
and with a velocity of constant module:

$$
  |\vec v| = \text{const}
$$

This means that the particle spans equal angles $\theta$ in equal times.

We introduce the angular velocity $\omega$ defined as
$$
  \omega = \frac{\theta}{t} = \text{const}
$$

We can then write the law of motion in terms of angles:

$$
  \theta(t) = \theta_0 + \omega t
$$

In general

$$
  \omega (t) = \dv{\theta t}{t}
$$

After one turn we have that $\Delta \theta (t) = 2 \pi$
We call period $T$ the time it takes to do a full turn, then
$$
  T = \frac{2 \pi}{\omega}
$$

We also define the frequency as

$$
  \nu = \frac{1}{T}
$$

We note the relationship between $v$ and $\omega$:

$$
  v = \frac{2 \pi R}{T} = R \omega
$$

\subsubsection{Using position vector}

We decompose the position vector in its components:

$$
  \vec{r}(t) = x (t) \hat i  + y(t) + \hat j
$$

Note that $|\vec r (t)| = R \enspace \forall t$, hence

\begin{align*}
  \begin{cases}
    x(t) & = R \cos (\theta (t)) \\
    y(t) & = R \sin (\theta (t))
  \end{cases}
\end{align*}

We have that the velocity is

\begin{align*}
  \vec v & = \dv {r}{t} = \dv{x}{t} \hat i + \dv {y}{t} \hat j                     \\
         & = - R \sin (\theta(t)) \dv{\theta(t)}{t} \hat i
  + R \cos (\theta(t)) \dv{\theta(t)}{t} \hat j                                    \\
         & = - R \sin (\theta(t)) \omega \hat i + R \cos (\theta(t)) \omega \hat j \\
         & = v_x (t) \hat i + v_y (t) \hat j
\end{align*}

The module of the velocity is

\begin{align*}
  |\vec v| & = \sqrt{v_x^2(t) + v_y^2(t)}                 \\
           & = \sqrt{R^2 \omega^2 \underbrace{(\dots)}_1} \\
           & = R\omega
\end{align*}

\subsubsection{Acceleration}

\begin{align*}
  \vec a & = \dv {\vec v}{t} = \dv {t} (v \hat v) \\
         & = \dv {v}{t}\hat v + \dv{\hat v}{t} v  \\
         & = \dv {\hat v}{t} v \ne 0
\end{align*}

since the modulus doesn't change.

We can also compute this quantity by differentiating the velocity equation from above

\begin{align*}
  \vec a & = \dv{\vec v}{t}                                                                          \\
         & = - R \omega ^2 \cos (\theta(t)) \hat i - R \omega^2 \sin (\theta(t)) \hat j              \\
         & = - \omega^2 \underbrace{(R \cos (\theta(t)) \hat i +  \sin (\theta(t)) \hat j)}_{\vec r} \\
         & = -\omega^2 \vec r
\end{align*}

It's easy to compute the module of this vector since $|\vec r| = R$, hence

$$
  a = \omega^2  |\vec r| = \omega^2 R = \frac{v ^2}{r}
$$

\subsection{Polar coordinates}

Depending on the exercise, using cartesian coordinates can be difficult.
We can introduce the polar coordinates to mitigate this issue.

Instead of specifying $x$ and $y$, we define a radius $R$ and an angle $\theta$.

This is quite helpful in the case of circular motion since we have that $R$ is constant
and only $\theta$ is a function of time.

We can define a function to map $(x, y) \to (r, \theta)$:

\begin{align*}
  \begin{cases}
    r      & = \sqrt{x^2 + y^2}                \\
    \theta & = \arctan\left(\frac{y}{x}\right)
  \end{cases}
\end{align*}

We note that this function is a bijection between the two coordinate spaces.

Moreover, we denote the have that the angle vector $\vec \theta$
and the radius vector $\vec r$ are orthogonal. This makes the polar coordinates \textbf{orthonormal}.

\begin{center}
  \begin{tikzpicture}
    \draw[black, ->] (0,0) -- (5,0) node[anchor=north]{$x$};
    \draw[black, ->] (0,0) -- (0,5) node[anchor=west]{$y$};

    \draw[blue, thick] (0,0) circle (4);
    \draw[red, thick, ->] (0,0) -- (4.5,4.5) node[anchor=west]{$\vec r$};

    \filldraw [gray] (2.82,2.82) circle (2pt);
    \filldraw [gray] (0,2.82) circle (2pt) node[anchor=east]{$y(t)$};
    \filldraw [gray] (2.82,0) circle (2pt) node[anchor=north]{$x(t)$};

    \draw[gray, dashed] (2.82,0) -- (2.82,2.82);
    \draw[gray, dashed] (0,2.82) -- (2.82,2.82);

    \draw[green, ->] (2.82,2.82) -- (3.82,3.82) node[anchor=north]{$\hat r$};
    \draw[green, ->] (2.82,2.82) -- (2,3.66) node[anchor=south]{$\hat \theta$};
  \end{tikzpicture}
\end{center}

We can write the position vector in polar coordinates as follows

$$
  \vec r = x \hat i + y \hat j = r \hat r + \theta \hat \theta
$$

The terms $\hat r$ and $\hat \theta$ can change direction depending on $\theta$:

\begin{align*}
  \begin{cases}
    \hat r(\theta)      & = \cos \theta \hat i + \sin \theta \hat j   \\
    \hat \theta(\theta) & = - \sin \theta \hat i + \cos \theta \hat j
  \end{cases}
\end{align*}

or, in matrix form,

$$
  \begin{pmatrix}
    \vec r \\
    \vec \theta
  \end{pmatrix}
  =
  \begin{pmatrix}
    \cos \theta  & \sin \theta \\
    -\sin \theta & \cos \theta
  \end{pmatrix}
  \begin{pmatrix}
    \hat i \\
    \hat j
  \end{pmatrix}
$$

\subsubsection{Application in circular motion}

This is quite useful, for example, to calculate the velocity

$$
  \vec v = \dv {\vec r}{t} = \dv{t} (R \hat r + \cancel{\theta \hat \theta})
  = R \dv{\hat r}{t}
$$

(we cancelled $\theta \hat \theta$ because the $\theta$ component is 0, $\vec r$ it's just the radius.)

To compute $\dv{\hat r}{t}$ we need to pass back to cartesian coordinates and get the result from above

\begin{align*}
  \vec{v} & = - R \sin (\theta(t)) \omega \hat i + R \cos (\theta(t)) \omega \hat j                   \\
          & = R \omega \underbrace{[\sin (\theta(t)) \hat i + \cos (\theta(t)) \hat j]}_{\hat \theta} \\
          & = R \omega \hat \theta
\end{align*}

\begin{bluebox}{EXERCISE}
  Derive the same relation for acceleration.

  $$
    \vec{a} = \dv {t} \vec v = \dv{t}(R \omega \hat \theta) = ?
  $$
\end{bluebox}

\section{Class of 13/02/2024 - Dynamics}

Dynamics is the study of the \textit{causes} of motion.

\subsection{Newton's laws}

\begin{enumerate}
  \item \textbf{First law} (or law of inertia): A body that is subject to no external forces maintains in uniform velocity.
  \item \textbf{Second law}: The acceleration of a body is proportional to the force acting on it.
  \item \textbf{Third law}: For every action, there is an equal and opposite reaction.
\end{enumerate}

Note that that these laws cannot be proven mathematically, they are just a good model of reality.

\subsection{First law}

By looking at the laws it might look like the first law is a special case of the second law,
but it is actually very important by itself because it tells us that we need to specify the reference frame:
only some frame of reference can be \say{put in contact} and describe the same motion; we call such frames \textbf{inertial frames}.

\begin{examplebox}{Example}
  Consider an object that is steady from the point of view of some observer.

  Now consider another observer that is moving in circle.
  To this second observer will see the object moving in circle, hence he will see an acceleration.

  These two observers are in different frames of reference, and only the first one is inertial.
\end{examplebox}

\subsubsection{Galilean group}

$$
  \begin{cases}
    \vec {r'} = \vec r + \vec{r_b} \\
    \vec {r'} = R \vec r           \\
    \vec {r'} = \vec r + \vec{v_b} t
  \end{cases}
$$

These equations describe transformations between inertial frames of reference.

The first equation is a translation, the second is a scaling, the third one is uniform motion.

We can see how, based on the third equation, we cannot know if we are moving or if the other frame is moving.

\subsection{Second law}

We can write the second law as

$$
  F \propto a
$$

which simply means that $F$ is proportional to $a$.
We can be more precise and write the equation in vector form:

$$
  \vec F = m \vec a
$$

where $m$ is the mass of the object.

This $\vec F$ is the sum of all forces acting on the object.

The unit of the force is

$$
  [F] = [M][L][T]^{-2} = \si{\kilogram \meter \per \second \squared} = \si{\newton}
$$

Even though the equation seams simple it is actually quite difficult to show it is true.
For example, what is the mass of an object? This is a concept that Newton had to introduce.

To answer this question we need to find a constant force, measure the acceleration it causes and define the mass according to the law.

Still, when we measure mass (or any dimension really), we are just comparing a mass to another mass.

\subsubsection{Hooke's law}

This law describes the force exerted by a spring.

$$
  F = - k \Delta x
$$

We cannot prove this law mathematically, but experimentally we can see that for small displacements it holds (after too much pulling we break the spring).
We call the displacement range where the law holds \textit{linear regime}.

The constant $k$ is called \textit{spring constant} and it depends on the material and the shape of the spring.

\subsubsection{First cardinal law of motion}

We define a new quantity called \textit{momentum} $\vec p$.

$$
  \vec p = m \vec v = m \dv{\vec r}{t}
$$

its unit is $[p] = [M][L][T]^{-1} = \si{\kilogram \meter \per \second}$.

We can write the second law in terms of momentum:

$$
  \vec F = \dv{\vec p}{t} = \dv{t} (m \vec v) = m \dv{\vec v}{t} + \dv{m}{t} \vec v
$$

This is very interesting because we see how the force changes if, while the object is moving, the mass changes (for example a rocket using fuel).

Usually though this doesn't happen and we can rely on the special case described above.

Generally we see how force can be a function of many different things:

$$
  m \dv[2]{\vec r}{t} = \vec F(\vec r, \dv{\vec r}{v}, t)
$$

This is a second order differential equation: it is quite difficult to solve and there is no standard method to solve it.

To attempt to solve this equation we need to consider some initial conditions, 2 for each degree of freedom.

\subsection{Third law}

We can write the third law mathematically as

$$
  \vec{F}_{1 \to 2} = - \vec{F}_{2 \to 1}
$$

There is not much more to say regarding this law.

\subsection{Gravitational force}

Consider the following system:

\begin{center}
  \begin{tikzpicture}
    \draw[black, ->] (0,0) -- (5,0) node[anchor=north]{$x$};
    \draw[black, ->] (0,0) -- (0,5) node[anchor=west]{$y$};

    \filldraw[red] (2,3) circle (2pt) node[anchor=south]{$m_1$};
    \filldraw[red] (4,4) circle (2pt) node[anchor=east]{$m_2$};

    \draw[black, ->] (2,3) -- (4,4) node[anchor=west]{$\vec r_2 - \vec r_1$};
    \draw[black, ->] (0,0) -- (2,3) node[anchor=north]{$\vec r_1$};
    \draw[black, ->] (0,0) -- (4,4) node[anchor=north]{$\vec r_2$};

  \end{tikzpicture}
\end{center}

We have that the gravitational force is given by

$$
  \vec F = G \frac{ m_1 m_2}{|\vec r_2 - \vec r_1|^2} \hat r
$$

Note that if we put $F = ma$ we get that the acceleration for $m_1$ is

$$
  \vec a = G \frac{m_2}{|\vec r_2 - \vec r_1|^2} \hat r
$$

If we are on the surface of the Earth we can use the approximation $|\vec r_2 - \vec r_1| \approx R$, since the position on the planet is negligible compared to the radius of the planet.

Then we get that the acceleration of any object on the surface of Earth is

$$
  \vec a = - G \frac{M}{R^2} = g \approx \SI{9.81}{\meter \per \second \squared}
$$

\subsection{Examples}

\subsubsection{Example 1}

Consider the very basic case of a still box of mass $m$ on a table.

We have two forces acting on the box: the gravitational force (pulling the box down) and the normal force (keeping the box on the table).

$$
  \vec N = - m \vec g
$$

\subsubsection{Example 2}

Consider two boxes one next to the other on a surface with no friction, one of mass $m_1$ and the other of mass $m_2$.

\begin{center}
  \begin{tikzpicture}
    \draw[black] (0,0) -- (5,0);
    \draw[blue] (2,0) rectangle (3,1) node[midway]{$m_1$};
    \draw[red] (3,0) rectangle (4,3) node[midway]{$m_2$};
  \end{tikzpicture}
\end{center}

Intuitively we can just consider the two boxes as a single object of mass $m_1 + m_2$ but we will solve the problem as if we didn't notice this.

For each body we can draw a \textbf{free body diagram}:

\begin{center}
  \begin{tikzpicture}
    \draw[blue] (0,0) rectangle (3,3) node[midway]{$m_1$};

    \draw[black, ->] (1.5,0) -- (1.5,-1.5) node[anchor=west]{$-m_1 g$};
    \draw[black, ->] (1.5,3) -- (1.5,4.5) node[anchor=west]{$\vec N_1$};
    \draw[black, ->] (-1.5,1.5) -- (0,1.5) node[anchor=east]{$\vec F$};
    \draw[black, ->] (3,1.5) -- (4.5,1.5) node[anchor=west]{$\vec f_1$};
  \end{tikzpicture}


  \begin{tikzpicture}
    \draw[red] (0,0) rectangle (3,3) node[midway]{$m_2$};

    \draw[black, ->] (1.5,0) -- (1.5,-1.5) node[anchor=west]{$-m_2 g$};
    \draw[black, ->] (1.5,3) -- (1.5,4.5) node[anchor=west]{$\vec N_2$};
    \draw[black, ->] (-1.5,1.5) -- (0,1.5) node[anchor=east]{$-\vec f$};
  \end{tikzpicture}
\end{center}

We know that the normal forces are equal and opposite to the gravitational forces, and since there are no other forces acting in those directions we can cancel them out.

Now we focus on the horizontal direction.
We write the second law for each box:

\begin{align*}
  m_1 a_1 & = F - f \\
  m_2 a_2 & = f
\end{align*}

and

$$
  (m_1 + m_2) a = F
$$

\subsubsection{Example 3}

Consider two boxes connected by a rope.
We can define \textbf{tension} as the force exerted by the rope on the boxes.
By the third law we have that the tension is the same for both boxes.

\section{Class of 15/02/2024}

\subsection{Equivalence principle}

We have two different kind of masses: the \textbf{inertial mass} and the \textbf{gravitational mass}.

The first one is the mass that appears in the second law, the second one is the mass that appears in the gravitational force.

The equivalence principle (proposed by Einstein) states that the two masses are the same.

\subsubsection{Example: elevator}

Consider an elevator which is accelerating upwards with an acceleration $a$.

We have that

$$
  N - m g = m a
$$

Where $N$ is the normal force. We get that $N = m(a + g)$

(Note that we are ignoring tidal forces.)

This principle says that there is no way to distinguish between the two forces, in the example we see how we cannot know if the elevator is accelerating or if the gravitational force is stronger.

\subsection{Harmonic oscillator}

Consider a mass $m$ attached to a spring with spring constant $k$. Let $\Delta x$ be the displacement from the equilibrium position $x_0$. For simplicity we will consider a system with origin in $x_0$. Suppose there is no gravity and no friction.

From the equation of motion we have that

$$
  m a_x = m \dv[2]{t}x(t) = F(t) = - k x(t)
$$

We need to solve the differential equation

$$
  \dv[2]{t}x(t) = - \frac{k}{m} x(t)
$$

\subsubsection{Solving the differential equation}

To solve this we have to find a function that is its own second derivative but with the sign changed.

We can work on the function $e^t$ since it is similar to the result we want to get. We can obtain the minus sign by introducing the imaginary unit $i$.

$$
  \dv[2]{t}e^{i t} = - e^{i t}
$$

Although this would work we have the problem of the dimension of time: when we do a taylor expansion of a $e^t$ we get a number that has dimension $[T] + [T]^2 + \dots$ which is not what we want.

To get rid of the dimension of $t$ we introduce a coefficient $\omega$ with dimension $[T]^{-1}$.

$$
  \dv[2]{t}e^{i \omega t} = - \omega^2 e^{i \omega t}
$$

From complex analysis we know Euler's formula:

$$
  e^{i \omega t} = \cos (\omega t) + i \sin (\omega t)
$$

Now, since we are doing physics here, we want to get rid of the imaginary unit.

We notice that both $\cos(\omega t)$ and $\sin(\omega t)$ solves the equation. When we solve differential equations (we know ?) that the general solution is a linear combination of 2 particular solutions.

$$
  x(t) = A \cos(\omega t) + B \sin(\omega t)
$$

Now we reparametrize the function:

\begin{align*}
  A & = R \cos(\gamma) \\
  B & = R \sin(\gamma)
\end{align*}

We can write the function as

\begin{align*}
  x(t) & = R \cos(\gamma) \cos(\omega t) + R \sin(\gamma) \sin(\omega t)              \\
       & = R \left( \cos(\gamma) \cos(\omega t) + \sin(\gamma) \sin(\omega t) \right) \\
       & = R \cos(\omega t - \gamma)
\end{align*}

\subsubsection{Fixing the constants}

We know that $\omega^2 = \frac{k}{m} \implies \omega = \pm \sqrt{\frac{k}{m}}$.

Now we have to fix the constants $R$ and $\gamma$ and to do so we will use the initial conditions of the system.

\begin{align*}
  x(0) & = x_0 = R \cos(\gamma)        \\
  v(0) & = v_0 = R \omega \sin(\gamma)
\end{align*}

We obtain these results by setting $t = 0$ in the function and in its derivative and exploiting the fact that sine is an even function.

\subsection{Friction}

Friction is quite a complex force that depends on many complex factors. For the purpose of this course we will consider a simple model of friction which describes well enough the behavior we see in real life.

The first thing we notice is that the friction depends on the two surfaces in contact, we will model this through a coefficient $\mu$.

Moreover we see how if we push a mass with a force $F$ the friction will keep the mass still until the force is greater than a certain value, after that threshold the mass will start moving.
We call this value $f_{\text{max}}$ and say that $\norm{f} \leq f_{\text{max}}$.

We also notice that the friction is proportional to the normal force. Combining all the above we get that

$$
  f = \mu N
$$

The direction of this force is opposite to the direction of the force that is trying to move the mass.

Moreover the coefficient $\mu$ changes if the mass is moving or not: if the mass is still we use the static coefficient of friction $\mu_s$, if the mass is moving we use the kinetic coefficient of friction $\mu_k$.

\subsubsection{Example: Inclined plane}

\begin{center}
  \begin{tikzpicture}
    \draw[black] (0,3) -- (5,0);
    \draw[black] (0,0) -- (5,0) node[midway, below]{$b$};
    \draw[black] (0,0) -- (0,3) node[midway, left]{$h$};

    % \draw[blue, rotate=60] (5,1) rectangle (7,2) node[midway]{$m$};
    % TODO: finish the drawing
  \end{tikzpicture}
\end{center}

Consider a mass $m$ on an inclined plane with angle $\theta$.

We have that $\theta = \arctan\left(\frac{h}{b}\right)$.

The gravitational force is always pointing towards $-\hat j$, hence it will try to pull the mass down the plane.

Since the plane is inclined we need to find the two components of the gravitational force in order to find which one is causing the body to move and which one is creating a normal force.

The friction force, then, will be pointing in the opposite direction of motion, hence the opposite direction of the component of the gravitational force that is causing the motion.

We will consider as frame of reference the one that is parallel $\hat u_{\parallel}$ and perpendicular $\hat u_{\perp}$ to the plane.

Consider the forces along each direction:

\begin{itemize}
  \item $\hat u_{\parallel}$:
        Here we have the gravitational force $F_{g\parallel} = mg \sin \theta$ and the friction $f = \mu N$
  \item $\hat u_{\perp}$: Here we have the normal force $N$ and the gravitational force $F_{g\perp} = mg \cos \theta = - N$
\end{itemize}

The body will start moving only after $F_{g\parallel} > f$. Thus

\begin{align*}
  F_{g\parallel} & < f       \\
  mg \sin \theta & < \mu_s N
\end{align*}

We call $\theta_c$ ($\theta$ critical) the angle at which the body starts moving. We can calculate it as follows:

\begin{align*}
  mg \sin \theta_c & = \mu_s N                \\
  mg \sin \theta_c & = \mu_s mg \cos \theta_c \\
  \tan \theta_c    & = \mu_s
\end{align*}

This is very important because in this way we have a way to measure the coefficient of friction.

\end{document}