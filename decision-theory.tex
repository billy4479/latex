\documentclass[12pt]{extarticle}

\setlength{\headheight}{15pt} % ??? we do what fancyhdr tells us to do  

\title{Decision Theory and Human Behavior}
\author{Giacomo Ellero}
\date{a.y. 2024/2025}

\usepackage{preamble}

\renewcommand{\vec}[1]{\uvec{#1}}

\begin{document}

\firstpage

\section{Binary relations}

\begin{definition}{Proposition}{proposition}
    A proposition is a statement that can be either true or false.
\end{definition}

\begin{definition}{Logical predicate}{predicate}
    A logical predicate $p(x)$ is an expression that contains an argument $x$ that varies in a given domain $X$.
\end{definition}

Note that if we take a predicate and choose an $x$ and fix it we get a proposition.

To turn a predicate into a proposition we can use the universal $\forall$ and existential $\exists$ quantifier.

We can also identify a predicate with the set $A$ of all the elements $x \in X$ such that the proposition $p(x)$ is true.
\begin{equation}
    \label{eq:extension_set}
    A = \{ x \in X : p(x) \text{ is true} \}
\end{equation}

Moreover we can also have predicates with two or more variables that may have different domains.
We call a predicate with two variables a \textbf{binary predicate}. We can again define a set like the one in \cref{eq:extension_set}:
\begin{equation}
    R = \{ (x,y) \in X \times Y : p(x,y) \text{ is true} \}
\end{equation}
we call this the \textbf{extension set} of $p$.

For our purposes we will always consider $X = Y$.

\begin{definition}{Binary relation}{binary-relation}
    Let $X$ be a domain. A binary relation is a subset $R$ of $X \times X$.
\end{definition}

Therefore a binary relation is the extension of a binary predicate.

\begin{remark}{Set-theoretic notation}{set-theoretic-notation}
    Instead of $(x, y) \in R$ we will instead just write
    \begin{equation}
        xRy
    \end{equation}
    It reads as \say{$x$ is in the relation $R$ with $y$}.
\end{remark}

\begin{proposition}{Properties of binary relations}{prop-binary-relations}
    Given a binary relation $R$ on a set $X$ we say that $R$ is
    \begin{enumerate}[label=\roman*.]
        \item \textbf{Reflexive} if $xRx \enspace \forall x \in X$
        \item \textbf{Transitive} if $xRy \land yRz \implies xRz$
        \item \textbf{Complete} if $\forall x, y \in R$ either $xRy$ or $yRx$ or both
        \item \textbf{Symmetric} if $xRy \implies yRx \enspace \forall x,y \in X$
        \item \textbf{Asymmetric} if $xRy \implies \lnot yRn \enspace \forall x,y \in X$
        \item \textbf{Antisymmetric} if $xRy \land yRx \implies x = y \enspace \forall x,y \in X$
    \end{enumerate}
\end{proposition}

\begin{definition}{The $I$ relation}{i-relation}
    Given a binary relation $R$ and $x, y \in X$ we define $I$ as
    \begin{equation}
        xIy \iff xRy \land yRx
    \end{equation}
\end{definition}

\begin{proposition}{Properties of $I$}{props-i}
    Let $R$ be reflexive and transitive (a \textit{preorder}) on a set $X$.
    Then the induced binary relation $I$ is reflexive, symmetric and transitive.
\end{proposition}

\begin{proof}
    We want to show that $I$ is transitive.
    Let $xIy$ and $yIz$, this is equivalent, by definition, to $xRy \land yRx$ and $yRz \land zRy$.
    By transitivity of $R$ we have that $xRy \land yRz \implies xRz$ and $yRx \land zRy \implies zRx$, meaning $xIz$.
\end{proof}

\begin{definition}{Equivalence relation}{equivalence-relation}
    A binary relation $R$ is an equivalence relation if it satisfies \textit{reflexivity}, \textit{symmetry}, and \textit{transitivity}.
\end{definition}

\begin{corollary}{}{}
    $I$ is an equivalence relation.
\end{corollary}

\begin{definition}{Equivalence class}{equivalence-class}
    Given an equivalence relation $R$ and $x \in X$ we define the equivalence class of $x$ as
    \begin{equation}
        [x] = \{y \in X : xRy\}
    \end{equation}
    that is, the set of all elements of $X$ equivalent to $x$ according to $R$.
\end{definition}

\begin{lemma}{}{}
    If $y \in [x]$, then $[y] = [x]$.
\end{lemma}

\begin{proposition}{Equivalence classes a partition}{eq-classes-partition}
    If $R$ is an equivalence relation on $X$ define
    \begin{equation}
        X/R = \{ [x] : x\in X \}
    \end{equation}
    as the set all equivalence classes in $X$ w.r.t. $R$.

    $X/R$ form a partition on $X$.
\end{proposition}

\begin{definition}{Partial order}{partial-order}
    A binary relation $R$ is a partial order if it satisfies \textit{reflexivity}, \textit{transitivity}, and \textit{antisymmetry}.
\end{definition}

A partial order is a preorder (as seen in \cref{prop:props-i}) with antisymmetry added on top.

\begin{example}{Partial order for functions}{}
    Consider the set $\R^\Omega$, the set of functions $f: \Omega \to \R$.
    One can define a partial order on $\R^\Omega$ as
    \begin{equation}
        f \geq g \implies f(\omega) \geq g(\omega) \enspace \forall \omega \in \Omega
    \end{equation}

\end{example}

\begin{example}{Partial order for sets}{}
    The easiest partial order for sets is inclusion:
    \begin{equation}
        A \geq B \iff A \subseteq B
    \end{equation}
\end{example}

If we are able to define a partial order $\geq$ on $X$ we call $(X, \geq)$ a \textbf{partially ordered set}.

Moreover, if $\forall x, y \in X$ either $x \geq y$ or $y \geq x$ we say that $\geq$ is a \textbf{total order}.

\end{document}
