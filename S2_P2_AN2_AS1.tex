\documentclass[12pt]{extarticle}

\usepackage{preamble_base}
\usepackage{preamble_math}

\title{Analysis Assignment}
\date{Semester 2, 2023/2024}

\renewcommand{\vec}[1]{\underline{\mathbf{#1}}}
\newcommand{\Hg}{{\operatorfont Hg}}
\newcommand{\tmin}{t_{\text{min}}}

\begin{document}

\section*{Analysis Assignment}

\subsection*{Problem 2}

Let $z(t) = (1-t) \vec x + t \vec y$.

\begin{enumerate}[label=\alph*.]
    \item $t_{\text{min}} \leq 1$ since if $t = 1$ we have $z(1) = \vec{y}$ which we know $\in E$.
    \item Consider the sequence $a_n = t_{\text{min}} + \frac{1}{n}$. The sequence $z(a_n)$ converges to $\vec z$ and $a_n \in E \enspace \forall n$, then $z \in \overline{E}$, but since $E$ is closed $E = \overline{E} \implies z \in E$.

          Moreover, $\tmin > 0$ since if $\tmin = 0$ we would have $z(0) = \vec x \notin E$.
    \item Since $E$ is open we have that for every $\vec a \in E \enspace \exists \varepsilon_0 > 0 : B(\vec z, \varepsilon_0) \subseteq E$, then we want to find some $\varepsilon$ such that $\norm {z(\tmin) - z(\tmin- \varepsilon)} < \varepsilon_0$.
          We have
          $$
              z(\tmin - \varepsilon) = (1 - t +\varepsilon)\vec x + (t-\varepsilon)\vec y = z(\tmin) + \varepsilon(\vec x - \vec y)
          $$
          and substituting we get
          $$
              \varepsilon < \frac{\varepsilon_0}{\norm{\vec x - \vec y}}
          $$
          then $z(\tmin - \varepsilon) \in B(z(\tmin), \varepsilon_0) \subseteq E$, that is $z(\tmin - \varepsilon) \in E$.

    \item We encountered a contradiction: the fact that $z(\tmin - \varepsilon) \in E$ contradicts the fact that $\tmin$ is an infimum.
\end{enumerate}

\subsection*{Problem 5}

\begin{enumerate}[label=\alph*.]
    \item Checking that $f(0,0) = 0$ is trivial since any integral whose bounds of integrations are equal is 0.
          To prove that $\pdv{f}{y}()(x,y) = h(x, y)$ we can use the fact that, as $x$ is fixed, $\int_0^x g(t,0) \dd{t}$ is a constant and therefore it vanishes in the derivation process; for $\int_0^y h(x, s) \dd{s}$ we can use the fundamental theorem of calculus that tells us that the derivative of this second term is indeed $h(x, y)$.
    \item First, recall that the fundamental theorem of calculus tells us that $F(b) - F(a) = \int_a^b f(u) \dd{u}$ where $F'(x) = f(x)$. In particular, for functions $p: \R \to \R$ of class $C^1$ we can write $\int_a^b \dv{u} p(u) \dd{u} = p(b) - p(a)$.
          This is possible only for functions of class $C^1$, otherwise $\dv{u} p(u)$ might be undefined.

          Now we can prove the main result.
          We will write Fubini's theorem once using $\pdv{g}{y}$ and once using $\pdv{h}{x}$:
          \begin{gather*}
              \int_0^x \left(\int_0^y \pdv{g}{y}()(t, s) \dd{s} \right) \dd{t} = \int_0^y \left(\int_0^x \pdv{h}{x}()(t, s) \dd{t} \right) \dd{s} \\
              \int_0^x \left(g(t, y) - g(t,  0)\right) \dd{t} = \int_0^y \left( h(x, s) - h(0, s)\right) \dd{s} \\
              \int_0^x g(t, y) \dd{t} - \int_0^x g(t,  0) \dd{t} = \int_0^y h(x, s) \dd{s} - \int_0^y h(0, s) \dd{s} \\
          \end{gather*}
    \item We use the result from the previous part to write
          $$
              \int_0^x g(t, 0) \dd{t} + \int_0^y h(x, s) \dd{s} = \int_0^y h(0, s) \dd{s} + \int_0^x g(t, y) \dd{t}
          $$
          we recognize on the left side the initial definition of $f(x, y)$, therefore
          $$
              f(x, y) = \int_0^y h(0, s) \dd{s} + \int_0^x g(t, y) \dd{t}
          $$

          To calculate $\pdv{f}{x}()(x, y)$ we can apply the same reasoning as in the first part:
          the first term vanishes as it is constant in $x$, while the second term becomes $g(x, y)$ by the fundamental theorem of calculus.

          Thus $\pdv{f}{x}()(x, y) = g(x, y)$.
\end{enumerate}

\end{document}