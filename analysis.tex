\documentclass[17pt]{extarticle}
\title{Analysis Notes - Part 2}
\author{Giacomo Ellero}

\usepackage{amsfonts}
\usepackage{amsthm}
\usepackage{amssymb}
\usepackage{amsmath}
\usepackage{mathtools}
\usepackage{commath}
\usepackage{dirtytalk}
\usepackage{parskip}
\usepackage{mathrsfs}
\usepackage[many]{tcolorbox}
\usepackage{xparse}
\usepackage[a4paper,margin=1.5cm]{geometry}
\usepackage{bookmark}

\newcommand{\C}{\mathbb{C}}
\newcommand{\R}{\mathbb{R}}
\newcommand{\F}{\mathbf{F}}
\newcommand{\rk}{\operatorfont{rk} }
\renewcommand{\Re}{\operatorname{Re}}
\renewcommand{\Im}{\operatorname{Im}}

\newenvironment{absolutelynopagebreak}
  {\par\nobreak\vfil\penalty0\vfilneg
   \vtop\bgroup}
  {\par\xdef\tpd{\the\prevdepth}\egroup
   \prevdepth=\tpd}

\newtcolorbox{statementbox}[1]{colback=green!5!white,colframe=green!40!black,title={#1},fonttitle=\bfseries,parbox=false}

\NewDocumentEnvironment{statement}{ O{Theorem} m}{
\begin{absolutelynopagebreak}
\subsubsection{#2}
\label{#2}
\begin{statementbox}{#1}
}
{
\end{statementbox}
\end{absolutelynopagebreak}
}


\begin{document}

\maketitle
\tableofcontents
\clearpage

\section {Differential calculus}
\subsection{Derivative and affine functions}


\begin{statement}[Definition]{Affine and linear functions}
  Affine functions are of form $l(x) = px + q$ with $p, q \in \R$.
\end{statement}

Linear functions are of form $l(x) = px$.
Affine functions respect linearity $l(\alpha x_0 + \beta x_1) = \alpha l(x_1) + \beta l(x_2)$
iff $\alpha + \beta = 1$.

\begin{statement}[Definition]{Derivative}
  Let $f: \operatorfont{dom} f \to \R$ be a function defined in the neighbourhood of $x_0$.
  We say $f$ is differentiable at $x_0$ if the limit

  $$
    \lim_{x_1 \to x_0} \frac{f(x_1) - f(x_0)}{x_1 - x_0}
  $$

  exists and is a finitite real number.
\end{statement}

We note how if $f$ is differentiable at $x_0$ its tangent at that point has the equation

$$
  y = f'(x_0)(x-x_0) + f(x_0)
$$

We can formulate the limit from the definition in the following equivalent form

$$
  \lim_{h \to 0} \frac{f(x_0 + h) - f(x_0)}{h}
$$

\begin{statement}{Right and left derivatives}
  Considering the limit from the previous definition we say that
  the limit as $h \to 0^-$ is the \textbf{left derivative}
  and the limit as $h \to 0^+$ is the \textbf{right derivative}.

  A function is differentiable at $x_0$ iff its left and right derivative exist, are finite and equal.
\end{statement}

\begin{statement}{Differentiability implies continuity *}
  If $f$ is derivable at $x_0$ then it is also continous at $x_0$.
\end{statement}

\begin{proof}
  Suppose $f$ is differentiable.
  We need to prove that $\lim_{x \to x_0} f(x) = f(x_0)$ which is equivalent to $\lim{x \to x_0} (f(x) - f(x_0)) = 0$

  \begin{align*}
     & \lim_{x \to x_0} \left(f(x) - f(x_0)\right) = 0          \\
     & \lim_{x \to x_0} \frac{f(x) - f(x_0)}{x-x_0} (x-x_0) = 0 \\
     & \lim_{x \to x_0} f'(x_0) (x-x_0) = 0                     \\
     & \lim_{x \to x_0} f'(x_0) 0 = 0                           \\
  \end{align*}

  Which is indeed true iff $f'(x_0)$ exists and is a finite number.
\end{proof}

\subsection{Differentiability as best local approximation by linear function}

We want to find the best approximation of $f$ in terms of an affine function $l$ in a neightbourhood of $x_0$.
We then need to choose $p, q$ for our $l(x) = p(x-x_0) + q$.

We need define the error $E(x) = f(x) - l(x) = f(x) - p(x-x_0) - q$.

\begin{enumerate}
  \item We want that $E(x_0) = \lim_{x \to x_0} E(x) = 0$, hence $q := f(x_0)$
  \item Now we need to choose $p$. We choose $p$ such that $E(x) = o(x-x_0)$
\end{enumerate}

We can now elaborate on our second contraint.
By the definition of $o$ we have that
$\lim{x \to x_0} \omega = 0$.

We can define
$$
  \omega (x) := \begin{cases}
    \frac{E(x)}{x-x_0} \text{ if } x \ne x_0 \\
    0  \text{ if } x = x_0
  \end{cases}
$$

Then $E(x) = \omega (x)(x - x_0)$ and

\begin{align*}
  f(x) & = f(x_0) + p(x - x_0) + E(x)              \\
       & = f(x_0) + p(x - x_0) + \omega (x)(x-x_0) \\
       & = f(x_0) + (p + \omega(x))(x-x_0)         \\
       & = f(x_0) + (p + o(1))(x-x_0)              \\
       & = f(x_0) + p(x - x_0) + o(x-x_0)
\end{align*}

\end{document}