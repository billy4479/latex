\documentclass[12pt]{extarticle}

\title{Advanced Programming and Optimization Algorithms \\ Homework 03}
\author{Giacomo Ellero}
\date{16/04/2025}

\usepackage{preamble_base}
\usepackage{preamble_math}

\renewcommand{\vec}[1]{\bvec{#1}}

\numberwithin{equation}{section}

\begin{document}

\maketitle

\section*{Exercise 1}
\stepcounter{section}

Let $A$ be the incidence matrix of $G$ and assume that the odd cycle $C$ is made of the vertices
$v_1, \dots v_n$.
If $A$ is totally unimodular, then every square submatrix of $A$ is also TU.

Consider $A'$, the submatrix of $A$ containing the vertices and the edges which are part of the
cycle.
Since all the rows in $A'$ represent a node in the cycle, each row and each column has exactly two
$1$s.
We can rearrange the rows of the matrix so that
\begin{equation}
	A' = \begin{pmatrix}
		1      & 0 & \dots  & 0 & 1 \\
		1      & 1 & \dots  & 0 & 0 \\
		0      & 1 & \dots  & 0 & 0 \\
		\vdots &   & \ddots &   &   \\
		0      & 0 & \dots  & 1 & 1
	\end{pmatrix}
\end{equation}

We now want to compute the determinant of will now perform some row operations on $A'$: iterate over
$k \in \{2, \dots, n\}$ and add set $R_1$ to $R_1 + (-1)^k R_k$. In this manner, since $n$ is odd,
we will obtain a matrix such that
\begin{equation}
	\det A' =
	\begin{vmatrix}
		2      & 0 & \dots  & 0 & 0 \\
		1      & 1 & \dots  & 0 & 0 \\
		0      & 1 & \dots  & 0 & 0 \\
		\vdots &   & \ddots &   &   \\
		0      & 0 & \dots  & 1 & 1
	\end{vmatrix} =
	2 \cdot
	\begin{vmatrix}
		1      & 0 & \dots  & 0 & 0 \\
		1      & 1 & \dots  & 0 & 0 \\
		0      & 1 & \dots  & 0 & 0 \\
		\vdots &   & \ddots &   &   \\
		0      & 0 & \dots  & 1 & 1
	\end{vmatrix} = 2 \cdot \det(\mathds{1} ) = 2
\end{equation}
were we have used the $1$ in $R_1$ to remove the first $1$ in $R_2$, the $1$ in $R_2$ for the first
$1$ in $R_3$ and so on until we obtained the identity matrix.

Which means that $A$ cannot be TU because the determinant of its submatrix $A'$ is not in
$\{\pm 1 , 0\}$.

\section*{Exercise 2}
\stepcounter{section}

We model the problem as a min-cost flow.

Consider a directed bipartite graph $G = (V, E)$ where $V = M \cup R$: each node in $M$ represents
a mechanic and each node in $R$ represents a repair. We connect each node in $M$ to each node in
$R$, as each mechanic can perform any repair, however, each edge has a cost $c_{(u, v)}$.

Then consider another graph $G' = (V', E')$ such that $V' = V \cup \{s, t\}$ and
\begin{equation}
	E' = E \cup \{ (s, v) \mid v \in M \} \cup \{ (u, t) \mid u \in R\}
\end{equation}
The cost for these new edges is $0$ so that we form a new cost vector $\vec \kappa$ where the cost
for the original edges in $E$ is the one indicated by $\vec c$ while the missing spots are filled
with zeroes.

Set the capacity of all edges to $1$, then the max $s$-$t$ flow in $G'$ is
$\alpha = \abs{M} = \abs{R} = n$.
Let $M$ be the incidence matrix of $G$ and $M'$ the incidence matrix of $G'$.
Then we can write the problem as an ILP:
\begin{equation}
	\begin{array}{rrcl}
		\min              & \vec \kappa^T \vec x                    &      &        \\
		\text{subject to} & M^T \vec x                              & =    & 0      \\
		                  & \vec x                                  & \leq & 1      \\
		                  & \vec x                                  & \geq & 0      \\
		                  & \displaystyle \sum_{u \in M} x_{(s, u)} & \geq & \alpha
	\end{array}
\end{equation}
where we have one variable for each edge: the first constraint is the conservation of flow, the
second one is the capacity constraint (which is $1$ for all edges), the third one is the
non-negativity of the flow, and the last one tells us that we have to find a flow of at least
$\alpha$.

This program will always match all mechanics to a repair since this is the only way to obtain a flow
of at least $\alpha$.
Moreover, since $M$ is TU and $\alpha$ is already an integer solution coming from a face of an
integer polyhedron, as shown in class, the optimal solution to this program will be an integer (as
long as the cost vector is positive).

\section*{Exercise 3}
\stepcounter{section}

Consider $b = n$, $c_i = 1$ and $a_i = 2$ for all $i \in \{ 1, \dots, n\}$.
We can write the first constraint as
\begin{equation}
	\sum_{i = 1}^n x_i \leq \frac{n}{2}
\end{equation}

If $n$ is even the optimal face will have all integer vertices as we are able to fill the knapsack
up to the very limit.

If $n$ is odd we will need to leave some space empty in the knapsack, which will trick the Branch
and Bound algorithm into doing a lot of work.

Indeed, every LP relaxation will have a better objective than any ILP, since it will try to fill the
empty space with an \say{half} item.
Since the variable are between 0 and 1, when we choose one to bound, we are effectively
\say{freezing} it to be either 0 or 1.

In this manner, the branching will halt only when we have
$k = \floor{\frac{n}{2}} + 1 = \frac{n+1}{2}$ variables
frozen \emph{with the same value}:
\begin{itemize}
	\item If we have frozen $k$ variables with value $1$, we will have that the problem is
	      infeasible;
	\item If we have frozen $k$ variables with value $0$, the LP relaxation will give an integer
	      solution, which is kind of \say{forced} as we need to set to $1$ all the other variables
	      which are not frozen.
\end{itemize}

Note, however, that even when we reach a leaf of the tree, either because we have found an integer
solution or we have an unfeasible problem, the algorithm will continue on other branches because the
best integer solution will always be worse than the solution given by the LP relaxations.
This, in turns, will ensure that the Branch and Bound algorithm always builds a full binary tree,
where the leaves have $k$ variables fixed of the same type.

We now want to count how many leaves our tree has. To do this we count how many combinations of with
$k$ frozen variables of the same type we have.
\begin{itemize}
	\item First we count the number of combinations with $k$ zeroes: $\binom{n}{\frac{n+1}{2}}$;
	\item Then we count the number of combinations where we froze $n - k = \frac{n-1}{2}$ ones:
	      $\binom{n}{\frac{n+1}{2}}$.
\end{itemize}
Calling $\ell$ the total number of leaves we get
\begin{equation}
	\ell = \binom{n}{\frac{n+1}{2}} + \binom{n}{\frac{n-1}{2}}
\end{equation}

Then, since we have shown that we have a full binary tree, we can compute the number of nodes as
$2 \ell -1$, giving the final result of
\begin{equation}
	2 \left(\binom{n}{\frac{n+1}{2}} + \binom{n}{\frac{n-1}{2}} \right) -1
\end{equation}

We can prove that the number of nodes $m$ in a full binary tree is $2 \ell + 1$ by induction on the
number of leaves $\ell$
\begin{description}[font=\normalfont\itshape]
	\item[Base case] If $\ell = 1$ then there is just one node, therefore $m = 1$ and our formula
	      holds: $2(1) -1 = 1 = m$.
	\item[Inductive step] At $\ell + 1 > 1$ we have that the root node will have two children.
	      This will allow us to split the tree in two smaller ones, so that the children of the root
	      node are the new roots of the new trees. Each one of these new trees will have at least $1$
	      leaf (since it is not possible to create a tree without leaves) and they are still going to
	      be full binary trees.
	      The number of leaves of the original tree $\ell + 1$ is the sum of the number of leaves of the
	      subtrees $\ell_1 + \ell_2$. But these trees will necessarily have $\ell_1 < \ell + 1$, and
	      $\ell_2 < \ell + 1$, therefore we can apply the inductive hypothesis to get
	      \begin{equation}
		      m = (2 \ell_1 - 1) + (2 \ell_2 - 1) + 1 = 2 (\ell_1 + \ell_2) - 1 = 2 (\ell + 1) -1
	      \end{equation}
	      as desired. We have used the fact that the total number of nodes is $m = m_1 + m_2 + 1$.
\end{description}

\end{document}
