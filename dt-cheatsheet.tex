\documentclass[landscape, 12pt]{extarticle}

\title{Decision Theory Cheat Sheet}
\author{Giacomo Ellero}
\date{a.y. 2024/2025}

\usepackage{preamble_base}
\usepackage{preamble_math}
\usepackage{multicol}
\geometry{letterpaper}

\renewcommand{\vec}[1]{\bm{#1}}

\newcommand{\dsuccsim}{\ \dot \succsim \ }
\newcommand{\ddsuccsim}{\ \ddot \succsim \ }
\newcommand{\astsuccsim}{\succsim^*}

\begin{document}

\begin{center}
	\huge
	\bf
	Decision Theory
\end{center}

\setlength{\columnsep}{1cm}
\setlength{\columnseprule}{1pt}
\def\columnseprulecolor{\color{blue}}

\vspace{0.25cm}

\section{Notation}

\begin{multicols}{2}

	\paragraph{Actions} what the DM does
	\begin{itemize}
		\item Preference: $\succsim$
		\item Space: $\bm A$
		\item Collections of subsets $A$ of $\bm A$: $\mathcal A$
	\end{itemize}


	\paragraph{Consequences} what happens
	\begin{itemize}
		\item Preference: $\dot \succsim$
		\item Space: $\bm C$
	\end{itemize}


	\paragraph{State} everything else we cannot control
	\begin{itemize}
		\item Each state is called $s$
		\item Space: $S$
		\item Algebra on $S$: $\bm \Sigma$
		\item Each element of $\bm \Sigma$ is an event $E$ ???
	\end{itemize}

	\paragraph{Act} what happens in each state
	\[
		f: S \to \bm C = \sum^n_{i = 1} c_i \mathds 1_{E_i}
	\]
	\begin{itemize}
		\item $c_i$ is the consequence of event $E_i$
		\item Preference: $\mathring \succsim$
		\item Space: $\bm F$
		\item $\bm C \subseteq \bm F$: consequences can be seen as state independent acts
		\item We use simple acts i.e. where $\Im(f)$ is finite
		\item Every action gives an act $\rho(a) = \gamma_a: S \to \bm C$
		      where $\rho$ is the consequence function.
		\item There is a bijection between acts and actions given by
		      \[
			      a \succsim b \iff \rho(a) \, \mathring \succsim \, \rho(b)
		      \]
	\end{itemize}

	\paragraph{Lotteries}
	\begin{itemize}
		\item Space over consequences $\bm C$: $\bm \Delta(\bm C) = \bm L$
		\item Preference: $\ddot \succsim$
		\item Sometimes acts do not output consequences but lotteries:
		      $\gamma: \bm A \cross S \to \bm \Delta(\bm C)$ and $\gamma(a, s)(c)$ is the probability
		      of the consequence $c$ when action $a$ is taken in the state $s$
	\end{itemize}

	\paragraph{Charge}
	\begin{itemize}
		\item $\mu : \bm \Sigma \to \R$ s.t. for $A, B \in \bm \Sigma$ if $A \cap B = \varnothing$
		      then $\mu(A \cup B) = \mu(A) + \mu(B)$.
		\item $\mathrm{ba}(\bm \Sigma)$ is the space of bounded charges, that is
		      \[
			      \mathrm{ba}(\bm \Sigma) = \left\{
			      \mu \in \R^{\bm \Sigma}:
			      \substack{
				      \sup_{A \in \bm \Sigma} \abs{\mu(A)} < \inf \\
				      \mu(A \cup B) = \mu(A) + \mu(B) \enspace \forall A, B \in \Sigma \text{ s.t. } A \cap B = \varnothing
			      }
			      \right\}
		      \]
	\end{itemize}

\end{multicols}

\clearpage

\section{Axioms}

\begin{multicols}{2}
	\subsection{General}

	\paragraph{A.1 Transitivity}
	For all $x, y, z \in X$ if $x \succsim y$ and $y \succsim z$, then $x \succsim z$.

	\paragraph{A.2 Completeness}
	For all $x, y \in X$ we have $x \succsim y$ or $y \succsim x$ (or both).

	\paragraph{A.3 Monotonicity}
	For all $x, y \in X$, if
	\begin{itemize}
		\item \textit{Strict}: $x>y$ (at least one is bigger bigger)
		\item \textit{Strong}: $x \gg y$ (all components are bigger)
		\item \textit{Weak}: $x \geq y$
	\end{itemize}
	we have $x \succ y$ (or $x\succsim y$ for weak)

	\paragraph{A.4 Archimedean}
	For all $x \succ y \succ z \in X$ there exists $\alpha, \beta \in (0, 1)$ such that
	\[
		\alpha x + \overline \alpha z \succ y \succ \beta x + \overline \beta z
	\]

	\paragraph{A.5 Convexity}
	For all $x, y \in X$, there exists $\alpha \in [0, 1]$ such that $x \sim y$ implies
	that for the convex combination
	\[
		c = \alpha x + \overline \alpha y
	\]
	\begin{center}
		\begin{tabular}{cc|cc|cc}
			\textit{Weak}     & $c \succsim x$ &
			\textit{Strict}   & $c \succ x$    &
			\textit{Affinity} & $c \sim x$       \\
		\end{tabular}
	\end{center}

	\paragraph{A.8 Independence}
	For all $x, y, z \in X$ and $\alpha \in (0,1)$
	if $x \succ y$ then
	\[
		\alpha x + \overline \alpha z \succ \alpha y + \overline \alpha z
	\]

	\paragraph{A.16 Monotonicity (of acts)}
	For any $f, g \in \bm F$ if $f(s) \ddsuccsim g(s)$ for all $s \in S$
	then $f \dsuccsim g$.

	%\subsection{Lotteries}
	%\paragraph{B.1 Weak order}
	%$\succsim$ is complete and transitive (A.1 and A.2)
	%
	%\paragraph{B.2 Independence}
	%For all $l, l', l'' \in \bm L$ and $p \in (0,1)$
	%\[
	%	l \succ l' \implies pl + \overline p l'' \succ p l' + \overline p l''
	%\]
	%\textit{Strong}: replace $\succ$ with $\succsim$ and $\sim$.
	%
	%\paragraph{B.3 Archimedean}
	%For all $l \succ l' \succ l'' \in \bm L$ there exists $p,q \in (0,1)$
	%\[
	%	pl + \overline p l'' \succ l' \succ ql + \overline q l''
	%\]
	%
	%\paragraph{B.4 Monotonicity}
	%For all $c, c' \in \bm C$ then $c \gg c' \implies c\succ c'$ and $c \geq c' \implies c \succsim c'$.
	%
	%\paragraph{B.5 Certainty equivalent}
	%For every $l \in \bm L$ there exists $c_l \in \bm C$ such that $l \sim c_l$.

	\subsection{De Finetti}
	These are acioms over events $E \in \bm \Sigma$,
	where $\succsim^*$ is the \textit{qualitatibe probability}:
	\say{event $E$ is at least as likely as event $F$}.

	\paragraph{DF.1 Weak order}
	$\succsim^*$ is complete and transitive (A.1 and A.2)

	\paragraph{DF.2 Normalization}
	For all events $E \in \bm \Sigma$
	we have $S \succsim^* E \succsim^* \varnothing$
	and $S \succsim^* \varnothing$.

	\paragraph{DF.3 Additivity}
	For all events $E, F, H$ such that $E \cap H = F \cap H = \varnothing$ we have
	\[
		E \succsim^* F \iff E \cup H \succsim^* F \cup H
	\]

	\paragraph{DF.4 Equidivisibility}
	For each $n \geq 1$ there exists a partition of $2^n$ equally likely events.

	A stronger version (\textit{Divisibility}) states that for all $E \succ^* F$ there ecists
	a partition of events $\{H_i\}^n_{i = 1}$
	such that $E \succ^* F \cup H_i$ for all $i = 1, \dots, n$.

	\subsection{Savage}
	We consider $\succsim$ on the set $\bm F$ of acts from $S$ to $\bm C$.

	Let $f, g \in \bm F$ and $E \in \bm \Sigma$: we introduce the following notation
	\[
		fEg = \begin{cases}
			f(s) & \text{if } s \in E    \\
			g(s) & \text{if } s \notin E
		\end{cases}
	\]

	\paragraph{P.1 Weak order}
	$\succsim$ is complete and transitive (A.1 and A.2).

	\paragraph{P.2 Sure-thing principle}
	For all acts $f, g, h, h'$ and any event $E$
	\[
		fEh \succsim gEh \iff fEh' \succsim gEh'
	\]
	from which we can define the \textit{conditional preference} $\succsim_E$ as
	\[
		f \succsim_E g \iff \exists h \in \bm F : f E h \succsim gEh
	\]

	An event is \textit{null} if $\forall f, g$ there holds $fEh \sim gEh$.

	\paragraph{P.3 State independece}
	Let $E$ be a non-null event and $c, d \in \bm C$, for any $h \in \bm F$
	\[
		cEh \succsim dEh \iff c\, \dot \succsim \, d
	\]

	\paragraph{P.4 Stake independance}
	For any $c \dot \succsim d$ and $c' \dsuccsim d' \in \bm C$
	and $E, F \in \bm \Sigma$
	\[
		cEd \succsim cFd \iff c'Ed'\succsim c'Fd'
	\]

	\paragraph{P.5 Non-triviality}
	There exists $c \, \dot \succ \, d \in \bm C$.

	\paragraph{P.6 Divisibility}
	Let $f \succ g \in \bm F$ and $c \in \bm C$,
	then there exists a parition of events $\{E_i\}^n_{i=1}$
	such that $cE_if \succ g$ and $f \succ cE_ig$ for all $i=1,\dots,n$.

	This tells us that we can divide $S$ in events so small that we can create
	a partition of events that does not influence the outcome at all (represented by $cE_i$).

	\subsection{Gilboa-Schmeidler}
	\paragraph{GS.3 Certainty independece}
	For every $f, g \in \bm F(X)$ and $l \in \bm X$
	\[
		f \succ g \iff \alpha f + \overline \alpha l \succ \alpha g + \overline \alpha x
	\]
	same for $\sim$.

	\paragraph{GS.6 Uncertainty Aversion}
	If $f \sim g \in \bm F$ then $\alpha f + \overline \alpha g \succsim f$.

\end{multicols}

\clearpage

\section{Representations}

\begin{multicols}{2}
	\subsection{Savage}

	\paragraph{Axioms} P.1-P.6

	\paragraph{Theorem} The following are equivalent:
	\begin{itemize}
		\item $\succsim$ over $\bm F$ respects the axioms
		\item $\exists \dot u$ (utility on $\bm C$ prices) and
		      an unique probability measure $P$ of $\bm \Sigma$
		      such that
		      \[
			      u(f) = \int_S \dot u(f(s)) dP(s)
		      \]
		      represents $\succsim$.
	\end{itemize}

	\paragraph{Derivation}
	\begin{enumerate}
		\item P.* axioms for $\succsim$ $\implies$ DF.* axioms over $\succsim^*$ (on events)
		      $\implies$ by De Finetti $\exists P$ that represents $\succsim^*$..
		\item $P$ induces a probability measure $p_f \in \bm \Delta(\bm C)$ for  given act $f$
		      as follows $p_f(c) = P(s \in S : f(s) = c)$.
		\item $p_f \ddsuccsim p_g \iff f \succsim g$.
		\item P.* axioms on $\succsim$ $\implies$ A.1, A.2, A.4, A.8 axioms over $\ddsuccsim$ $\implies$
		      $\ddsuccsim$ has a vN-M utility representation.
		\item Combine:
		      \begin{align*}
			      f \succsim g & \iff p_f \ddsuccsim p_g                                             \\
			                   & \iff \sum \dot u(c)p_f(c) \geq \sum \dot u(c)p_g(c)                 \\
			                   & \iff \int_S \dot u(f(s)) \dd P(s) \geq \int_S \dot u(g(s)) \dd P(s)
		      \end{align*}
	\end{enumerate}

	\subsection{Anscombe-Aumann}

	\paragraph{Setup}
	Very similar to Savage
	but this times acts are defined as $f: S \to \bm \Delta(\bm C)$
	such that they give out lotteries instead of straigh up consequences.
	We therefore have two stages of uncertainty: first the events, and then the lottery.

	Differences from Savage: $\bm \Delta(\bm C)$ is a convex vector space,
	therefore it has more structure than $\bm C$ in Savage's setup;
	meanwhile $\bm \Sigma$ is just an algebra	of $S$, not necessarily a $\sigma$-algebra
	and we no longer have P.6, which asked for a non-trivial structure of the event space.

	In the abstract setting we consider a convex subset $X$ of a vector space
	instead of $\bm \Delta(\bm C)$.

	\paragraph{Axioms} A.1, A.2, A.4, A.8 (same as vN-M), and A.16.

	\paragraph{Theorem}
	Consider $\succsim$ over $\bm F$.
	The following are equivalent:
	\begin{itemize}
		\item $\succsim$ satisfies the axioms
		\item There exists $\dot u: \bm C \to \R$ and an unique probability measure $P: \bm \to [0, 1]$
		      such that
		      \[
			      u(f) = \int_S \left( \sum_{c\in \text{supp } f(s)} \dot u(c) f(s)(c) \right) \dd P(s)
		      \]
	\end{itemize}
	In the generalized settings we write
	\[
		u(f) = \int_S v(f(s))\dd P(s)
	\]

	\paragraph{Derivation}
	\begin{enumerate}
		\item TODO: lecture notes are a mess since they miss Chiasini lemma,
		      giuseppe's notes are much better
	\end{enumerate}

	\subsection{Gilboa-Schmeidler}

	\paragraph{Setup}
	This time the problem is that we want to model ambiguity.
	DMs are more likely to choose acts from which they are aware
	of the distribution of the consequences.

	\paragraph{Axioms} A.1, A.2, A.4, A.16, GS.3, GS.6

	\paragraph{Theorem}
	Consider $\succsim$ on $\bm F$.
	\begin{itemize}
		\item $\succsim$ satisfies the axioms
		\item There exists an affine function	$u: X \to \bm F$
		      and a set of probabilities $\mathcal C \subseteq \bm \Delta(\bm \Sigma)$
		      such that
		      \[
			      U(f) = \min_{P\in \mathcal C} \int_S u(f(s)) dP(s)
		      \]
	\end{itemize}

	\paragraph{Derivation}
	TODO: see giuseppe's notes
\end{multicols}

\end{document}
